
% Default to the notebook output style

    


% Inherit from the specified cell style.




    
\documentclass[11pt]{article}

    
    
    \usepackage[T1]{fontenc}
    % Nicer default font (+ math font) than Computer Modern for most use cases
    \usepackage{mathpazo}

    % Basic figure setup, for now with no caption control since it's done
    % automatically by Pandoc (which extracts ![](path) syntax from Markdown).
    \usepackage{graphicx}
    % We will generate all images so they have a width \maxwidth. This means
    % that they will get their normal width if they fit onto the page, but
    % are scaled down if they would overflow the margins.
    \makeatletter
    \def\maxwidth{\ifdim\Gin@nat@width>\linewidth\linewidth
    \else\Gin@nat@width\fi}
    \makeatother
    \let\Oldincludegraphics\includegraphics
    % Set max figure width to be 80% of text width, for now hardcoded.
    \renewcommand{\includegraphics}[1]{\Oldincludegraphics[width=.8\maxwidth]{#1}}
    % Ensure that by default, figures have no caption (until we provide a
    % proper Figure object with a Caption API and a way to capture that
    % in the conversion process - todo).
    \usepackage{caption}
    \DeclareCaptionLabelFormat{nolabel}{}
    \captionsetup{labelformat=nolabel}

    \usepackage{adjustbox} % Used to constrain images to a maximum size 
    \usepackage{xcolor} % Allow colors to be defined
    \usepackage{enumerate} % Needed for markdown enumerations to work
    \usepackage{geometry} % Used to adjust the document margins
    \usepackage{amsmath} % Equations
    \usepackage{amssymb} % Equations
    \usepackage{textcomp} % defines textquotesingle
    % Hack from http://tex.stackexchange.com/a/47451/13684:
    \AtBeginDocument{%
        \def\PYZsq{\textquotesingle}% Upright quotes in Pygmentized code
    }
    \usepackage{upquote} % Upright quotes for verbatim code
    \usepackage{eurosym} % defines \euro
    \usepackage[mathletters]{ucs} % Extended unicode (utf-8) support
    \usepackage[utf8x]{inputenc} % Allow utf-8 characters in the tex document
    \usepackage{fancyvrb} % verbatim replacement that allows latex
    \usepackage{grffile} % extends the file name processing of package graphics 
                         % to support a larger range 
    % The hyperref package gives us a pdf with properly built
    % internal navigation ('pdf bookmarks' for the table of contents,
    % internal cross-reference links, web links for URLs, etc.)
    \usepackage{hyperref}
    \usepackage{longtable} % longtable support required by pandoc >1.10
    \usepackage{booktabs}  % table support for pandoc > 1.12.2
    \usepackage[inline]{enumitem} % IRkernel/repr support (it uses the enumerate* environment)
    \usepackage[normalem]{ulem} % ulem is needed to support strikethroughs (\sout)
                                % normalem makes italics be italics, not underlines
    

    
    
    % Colors for the hyperref package
    \definecolor{urlcolor}{rgb}{0,.145,.698}
    \definecolor{linkcolor}{rgb}{.71,0.21,0.01}
    \definecolor{citecolor}{rgb}{.12,.54,.11}

    % ANSI colors
    \definecolor{ansi-black}{HTML}{3E424D}
    \definecolor{ansi-black-intense}{HTML}{282C36}
    \definecolor{ansi-red}{HTML}{E75C58}
    \definecolor{ansi-red-intense}{HTML}{B22B31}
    \definecolor{ansi-green}{HTML}{00A250}
    \definecolor{ansi-green-intense}{HTML}{007427}
    \definecolor{ansi-yellow}{HTML}{DDB62B}
    \definecolor{ansi-yellow-intense}{HTML}{B27D12}
    \definecolor{ansi-blue}{HTML}{208FFB}
    \definecolor{ansi-blue-intense}{HTML}{0065CA}
    \definecolor{ansi-magenta}{HTML}{D160C4}
    \definecolor{ansi-magenta-intense}{HTML}{A03196}
    \definecolor{ansi-cyan}{HTML}{60C6C8}
    \definecolor{ansi-cyan-intense}{HTML}{258F8F}
    \definecolor{ansi-white}{HTML}{C5C1B4}
    \definecolor{ansi-white-intense}{HTML}{A1A6B2}

    % commands and environments needed by pandoc snippets
    % extracted from the output of `pandoc -s`
    \providecommand{\tightlist}{%
      \setlength{\itemsep}{0pt}\setlength{\parskip}{0pt}}
    \DefineVerbatimEnvironment{Highlighting}{Verbatim}{commandchars=\\\{\}}
    % Add ',fontsize=\small' for more characters per line
    \newenvironment{Shaded}{}{}
    \newcommand{\KeywordTok}[1]{\textcolor[rgb]{0.00,0.44,0.13}{\textbf{{#1}}}}
    \newcommand{\DataTypeTok}[1]{\textcolor[rgb]{0.56,0.13,0.00}{{#1}}}
    \newcommand{\DecValTok}[1]{\textcolor[rgb]{0.25,0.63,0.44}{{#1}}}
    \newcommand{\BaseNTok}[1]{\textcolor[rgb]{0.25,0.63,0.44}{{#1}}}
    \newcommand{\FloatTok}[1]{\textcolor[rgb]{0.25,0.63,0.44}{{#1}}}
    \newcommand{\CharTok}[1]{\textcolor[rgb]{0.25,0.44,0.63}{{#1}}}
    \newcommand{\StringTok}[1]{\textcolor[rgb]{0.25,0.44,0.63}{{#1}}}
    \newcommand{\CommentTok}[1]{\textcolor[rgb]{0.38,0.63,0.69}{\textit{{#1}}}}
    \newcommand{\OtherTok}[1]{\textcolor[rgb]{0.00,0.44,0.13}{{#1}}}
    \newcommand{\AlertTok}[1]{\textcolor[rgb]{1.00,0.00,0.00}{\textbf{{#1}}}}
    \newcommand{\FunctionTok}[1]{\textcolor[rgb]{0.02,0.16,0.49}{{#1}}}
    \newcommand{\RegionMarkerTok}[1]{{#1}}
    \newcommand{\ErrorTok}[1]{\textcolor[rgb]{1.00,0.00,0.00}{\textbf{{#1}}}}
    \newcommand{\NormalTok}[1]{{#1}}
    
    % Additional commands for more recent versions of Pandoc
    \newcommand{\ConstantTok}[1]{\textcolor[rgb]{0.53,0.00,0.00}{{#1}}}
    \newcommand{\SpecialCharTok}[1]{\textcolor[rgb]{0.25,0.44,0.63}{{#1}}}
    \newcommand{\VerbatimStringTok}[1]{\textcolor[rgb]{0.25,0.44,0.63}{{#1}}}
    \newcommand{\SpecialStringTok}[1]{\textcolor[rgb]{0.73,0.40,0.53}{{#1}}}
    \newcommand{\ImportTok}[1]{{#1}}
    \newcommand{\DocumentationTok}[1]{\textcolor[rgb]{0.73,0.13,0.13}{\textit{{#1}}}}
    \newcommand{\AnnotationTok}[1]{\textcolor[rgb]{0.38,0.63,0.69}{\textbf{\textit{{#1}}}}}
    \newcommand{\CommentVarTok}[1]{\textcolor[rgb]{0.38,0.63,0.69}{\textbf{\textit{{#1}}}}}
    \newcommand{\VariableTok}[1]{\textcolor[rgb]{0.10,0.09,0.49}{{#1}}}
    \newcommand{\ControlFlowTok}[1]{\textcolor[rgb]{0.00,0.44,0.13}{\textbf{{#1}}}}
    \newcommand{\OperatorTok}[1]{\textcolor[rgb]{0.40,0.40,0.40}{{#1}}}
    \newcommand{\BuiltInTok}[1]{{#1}}
    \newcommand{\ExtensionTok}[1]{{#1}}
    \newcommand{\PreprocessorTok}[1]{\textcolor[rgb]{0.74,0.48,0.00}{{#1}}}
    \newcommand{\AttributeTok}[1]{\textcolor[rgb]{0.49,0.56,0.16}{{#1}}}
    \newcommand{\InformationTok}[1]{\textcolor[rgb]{0.38,0.63,0.69}{\textbf{\textit{{#1}}}}}
    \newcommand{\WarningTok}[1]{\textcolor[rgb]{0.38,0.63,0.69}{\textbf{\textit{{#1}}}}}
    
    
    % Define a nice break command that doesn't care if a line doesn't already
    % exist.
    \def\br{\hspace*{\fill} \\* }
    % Math Jax compatability definitions
    \def\gt{>}
    \def\lt{<}
    % Document parameters
    \title{COURSEWORK}
    
    
    

    % Pygments definitions
    
\makeatletter
\def\PY@reset{\let\PY@it=\relax \let\PY@bf=\relax%
    \let\PY@ul=\relax \let\PY@tc=\relax%
    \let\PY@bc=\relax \let\PY@ff=\relax}
\def\PY@tok#1{\csname PY@tok@#1\endcsname}
\def\PY@toks#1+{\ifx\relax#1\empty\else%
    \PY@tok{#1}\expandafter\PY@toks\fi}
\def\PY@do#1{\PY@bc{\PY@tc{\PY@ul{%
    \PY@it{\PY@bf{\PY@ff{#1}}}}}}}
\def\PY#1#2{\PY@reset\PY@toks#1+\relax+\PY@do{#2}}

\expandafter\def\csname PY@tok@kp\endcsname{\def\PY@tc##1{\textcolor[rgb]{0.00,0.50,0.00}{##1}}}
\expandafter\def\csname PY@tok@ne\endcsname{\let\PY@bf=\textbf\def\PY@tc##1{\textcolor[rgb]{0.82,0.25,0.23}{##1}}}
\expandafter\def\csname PY@tok@ch\endcsname{\let\PY@it=\textit\def\PY@tc##1{\textcolor[rgb]{0.25,0.50,0.50}{##1}}}
\expandafter\def\csname PY@tok@bp\endcsname{\def\PY@tc##1{\textcolor[rgb]{0.00,0.50,0.00}{##1}}}
\expandafter\def\csname PY@tok@dl\endcsname{\def\PY@tc##1{\textcolor[rgb]{0.73,0.13,0.13}{##1}}}
\expandafter\def\csname PY@tok@gi\endcsname{\def\PY@tc##1{\textcolor[rgb]{0.00,0.63,0.00}{##1}}}
\expandafter\def\csname PY@tok@c1\endcsname{\let\PY@it=\textit\def\PY@tc##1{\textcolor[rgb]{0.25,0.50,0.50}{##1}}}
\expandafter\def\csname PY@tok@s1\endcsname{\def\PY@tc##1{\textcolor[rgb]{0.73,0.13,0.13}{##1}}}
\expandafter\def\csname PY@tok@mo\endcsname{\def\PY@tc##1{\textcolor[rgb]{0.40,0.40,0.40}{##1}}}
\expandafter\def\csname PY@tok@se\endcsname{\let\PY@bf=\textbf\def\PY@tc##1{\textcolor[rgb]{0.73,0.40,0.13}{##1}}}
\expandafter\def\csname PY@tok@vc\endcsname{\def\PY@tc##1{\textcolor[rgb]{0.10,0.09,0.49}{##1}}}
\expandafter\def\csname PY@tok@ni\endcsname{\let\PY@bf=\textbf\def\PY@tc##1{\textcolor[rgb]{0.60,0.60,0.60}{##1}}}
\expandafter\def\csname PY@tok@kt\endcsname{\def\PY@tc##1{\textcolor[rgb]{0.69,0.00,0.25}{##1}}}
\expandafter\def\csname PY@tok@err\endcsname{\def\PY@bc##1{\setlength{\fboxsep}{0pt}\fcolorbox[rgb]{1.00,0.00,0.00}{1,1,1}{\strut ##1}}}
\expandafter\def\csname PY@tok@c\endcsname{\let\PY@it=\textit\def\PY@tc##1{\textcolor[rgb]{0.25,0.50,0.50}{##1}}}
\expandafter\def\csname PY@tok@cp\endcsname{\def\PY@tc##1{\textcolor[rgb]{0.74,0.48,0.00}{##1}}}
\expandafter\def\csname PY@tok@vi\endcsname{\def\PY@tc##1{\textcolor[rgb]{0.10,0.09,0.49}{##1}}}
\expandafter\def\csname PY@tok@kc\endcsname{\let\PY@bf=\textbf\def\PY@tc##1{\textcolor[rgb]{0.00,0.50,0.00}{##1}}}
\expandafter\def\csname PY@tok@gp\endcsname{\let\PY@bf=\textbf\def\PY@tc##1{\textcolor[rgb]{0.00,0.00,0.50}{##1}}}
\expandafter\def\csname PY@tok@go\endcsname{\def\PY@tc##1{\textcolor[rgb]{0.53,0.53,0.53}{##1}}}
\expandafter\def\csname PY@tok@sc\endcsname{\def\PY@tc##1{\textcolor[rgb]{0.73,0.13,0.13}{##1}}}
\expandafter\def\csname PY@tok@sd\endcsname{\let\PY@it=\textit\def\PY@tc##1{\textcolor[rgb]{0.73,0.13,0.13}{##1}}}
\expandafter\def\csname PY@tok@gs\endcsname{\let\PY@bf=\textbf}
\expandafter\def\csname PY@tok@gu\endcsname{\let\PY@bf=\textbf\def\PY@tc##1{\textcolor[rgb]{0.50,0.00,0.50}{##1}}}
\expandafter\def\csname PY@tok@mb\endcsname{\def\PY@tc##1{\textcolor[rgb]{0.40,0.40,0.40}{##1}}}
\expandafter\def\csname PY@tok@sh\endcsname{\def\PY@tc##1{\textcolor[rgb]{0.73,0.13,0.13}{##1}}}
\expandafter\def\csname PY@tok@kd\endcsname{\let\PY@bf=\textbf\def\PY@tc##1{\textcolor[rgb]{0.00,0.50,0.00}{##1}}}
\expandafter\def\csname PY@tok@mi\endcsname{\def\PY@tc##1{\textcolor[rgb]{0.40,0.40,0.40}{##1}}}
\expandafter\def\csname PY@tok@mh\endcsname{\def\PY@tc##1{\textcolor[rgb]{0.40,0.40,0.40}{##1}}}
\expandafter\def\csname PY@tok@nt\endcsname{\let\PY@bf=\textbf\def\PY@tc##1{\textcolor[rgb]{0.00,0.50,0.00}{##1}}}
\expandafter\def\csname PY@tok@gh\endcsname{\let\PY@bf=\textbf\def\PY@tc##1{\textcolor[rgb]{0.00,0.00,0.50}{##1}}}
\expandafter\def\csname PY@tok@cm\endcsname{\let\PY@it=\textit\def\PY@tc##1{\textcolor[rgb]{0.25,0.50,0.50}{##1}}}
\expandafter\def\csname PY@tok@sx\endcsname{\def\PY@tc##1{\textcolor[rgb]{0.00,0.50,0.00}{##1}}}
\expandafter\def\csname PY@tok@vg\endcsname{\def\PY@tc##1{\textcolor[rgb]{0.10,0.09,0.49}{##1}}}
\expandafter\def\csname PY@tok@sr\endcsname{\def\PY@tc##1{\textcolor[rgb]{0.73,0.40,0.53}{##1}}}
\expandafter\def\csname PY@tok@nn\endcsname{\let\PY@bf=\textbf\def\PY@tc##1{\textcolor[rgb]{0.00,0.00,1.00}{##1}}}
\expandafter\def\csname PY@tok@gr\endcsname{\def\PY@tc##1{\textcolor[rgb]{1.00,0.00,0.00}{##1}}}
\expandafter\def\csname PY@tok@fm\endcsname{\def\PY@tc##1{\textcolor[rgb]{0.00,0.00,1.00}{##1}}}
\expandafter\def\csname PY@tok@s2\endcsname{\def\PY@tc##1{\textcolor[rgb]{0.73,0.13,0.13}{##1}}}
\expandafter\def\csname PY@tok@sa\endcsname{\def\PY@tc##1{\textcolor[rgb]{0.73,0.13,0.13}{##1}}}
\expandafter\def\csname PY@tok@kn\endcsname{\let\PY@bf=\textbf\def\PY@tc##1{\textcolor[rgb]{0.00,0.50,0.00}{##1}}}
\expandafter\def\csname PY@tok@ow\endcsname{\let\PY@bf=\textbf\def\PY@tc##1{\textcolor[rgb]{0.67,0.13,1.00}{##1}}}
\expandafter\def\csname PY@tok@il\endcsname{\def\PY@tc##1{\textcolor[rgb]{0.40,0.40,0.40}{##1}}}
\expandafter\def\csname PY@tok@ge\endcsname{\let\PY@it=\textit}
\expandafter\def\csname PY@tok@si\endcsname{\let\PY@bf=\textbf\def\PY@tc##1{\textcolor[rgb]{0.73,0.40,0.53}{##1}}}
\expandafter\def\csname PY@tok@gd\endcsname{\def\PY@tc##1{\textcolor[rgb]{0.63,0.00,0.00}{##1}}}
\expandafter\def\csname PY@tok@o\endcsname{\def\PY@tc##1{\textcolor[rgb]{0.40,0.40,0.40}{##1}}}
\expandafter\def\csname PY@tok@gt\endcsname{\def\PY@tc##1{\textcolor[rgb]{0.00,0.27,0.87}{##1}}}
\expandafter\def\csname PY@tok@nb\endcsname{\def\PY@tc##1{\textcolor[rgb]{0.00,0.50,0.00}{##1}}}
\expandafter\def\csname PY@tok@sb\endcsname{\def\PY@tc##1{\textcolor[rgb]{0.73,0.13,0.13}{##1}}}
\expandafter\def\csname PY@tok@w\endcsname{\def\PY@tc##1{\textcolor[rgb]{0.73,0.73,0.73}{##1}}}
\expandafter\def\csname PY@tok@na\endcsname{\def\PY@tc##1{\textcolor[rgb]{0.49,0.56,0.16}{##1}}}
\expandafter\def\csname PY@tok@nv\endcsname{\def\PY@tc##1{\textcolor[rgb]{0.10,0.09,0.49}{##1}}}
\expandafter\def\csname PY@tok@kr\endcsname{\let\PY@bf=\textbf\def\PY@tc##1{\textcolor[rgb]{0.00,0.50,0.00}{##1}}}
\expandafter\def\csname PY@tok@cpf\endcsname{\let\PY@it=\textit\def\PY@tc##1{\textcolor[rgb]{0.25,0.50,0.50}{##1}}}
\expandafter\def\csname PY@tok@nc\endcsname{\let\PY@bf=\textbf\def\PY@tc##1{\textcolor[rgb]{0.00,0.00,1.00}{##1}}}
\expandafter\def\csname PY@tok@nf\endcsname{\def\PY@tc##1{\textcolor[rgb]{0.00,0.00,1.00}{##1}}}
\expandafter\def\csname PY@tok@m\endcsname{\def\PY@tc##1{\textcolor[rgb]{0.40,0.40,0.40}{##1}}}
\expandafter\def\csname PY@tok@k\endcsname{\let\PY@bf=\textbf\def\PY@tc##1{\textcolor[rgb]{0.00,0.50,0.00}{##1}}}
\expandafter\def\csname PY@tok@nl\endcsname{\def\PY@tc##1{\textcolor[rgb]{0.63,0.63,0.00}{##1}}}
\expandafter\def\csname PY@tok@vm\endcsname{\def\PY@tc##1{\textcolor[rgb]{0.10,0.09,0.49}{##1}}}
\expandafter\def\csname PY@tok@mf\endcsname{\def\PY@tc##1{\textcolor[rgb]{0.40,0.40,0.40}{##1}}}
\expandafter\def\csname PY@tok@cs\endcsname{\let\PY@it=\textit\def\PY@tc##1{\textcolor[rgb]{0.25,0.50,0.50}{##1}}}
\expandafter\def\csname PY@tok@nd\endcsname{\def\PY@tc##1{\textcolor[rgb]{0.67,0.13,1.00}{##1}}}
\expandafter\def\csname PY@tok@no\endcsname{\def\PY@tc##1{\textcolor[rgb]{0.53,0.00,0.00}{##1}}}
\expandafter\def\csname PY@tok@ss\endcsname{\def\PY@tc##1{\textcolor[rgb]{0.10,0.09,0.49}{##1}}}
\expandafter\def\csname PY@tok@s\endcsname{\def\PY@tc##1{\textcolor[rgb]{0.73,0.13,0.13}{##1}}}

\def\PYZbs{\char`\\}
\def\PYZus{\char`\_}
\def\PYZob{\char`\{}
\def\PYZcb{\char`\}}
\def\PYZca{\char`\^}
\def\PYZam{\char`\&}
\def\PYZlt{\char`\<}
\def\PYZgt{\char`\>}
\def\PYZsh{\char`\#}
\def\PYZpc{\char`\%}
\def\PYZdl{\char`\$}
\def\PYZhy{\char`\-}
\def\PYZsq{\char`\'}
\def\PYZdq{\char`\"}
\def\PYZti{\char`\~}
% for compatibility with earlier versions
\def\PYZat{@}
\def\PYZlb{[}
\def\PYZrb{]}
\makeatother


    % Exact colors from NB
    \definecolor{incolor}{rgb}{0.0, 0.0, 0.5}
    \definecolor{outcolor}{rgb}{0.545, 0.0, 0.0}



    
    % Prevent overflowing lines due to hard-to-break entities
    \sloppy 
    % Setup hyperref package
    \hypersetup{
      breaklinks=true,  % so long urls are correctly broken across lines
      colorlinks=true,
      urlcolor=urlcolor,
      linkcolor=linkcolor,
      citecolor=citecolor,
      }
    % Slightly bigger margins than the latex defaults
    
    \geometry{verbose,tmargin=1in,bmargin=1in,lmargin=1in,rmargin=1in}
    
    

    \begin{document}
    
    
    \maketitle
    
    

    
    \section{Coursework}\label{coursework}

    Candidate number: 22817

Date submitted: 19/12/2019

    \subsubsection{Abstract}\label{abstract}

    Two exoplanets were identified to be orbiting around a Kepler star
system using the 'primary transit' detection method. This method
involves observing dips in the emitted flux by the star caused by an
exoplanet moving in front of the star. The paper demonstrated multiple
data reduction techniques and how they can be used in a varied of
situations. Aperture photometry was explored as a concept of measuring a
star's flux and magnitude. The GROND dataset was reduced using bias
frame subtraction and flat fielding. Using a standard star of known
absolute magnitude +19.50 the absolute magnitude of a star in the image
was determined to be +17.33 +/- 0.62.

Normalised lightcurves were created from the Kepler data by applying a
median filter. From a periodogram two exoplanets of period 19.237
(Planet 1) and 38.98 (Planet 2) days were discovered. Their radii
(Jupiter radii), semi major axis and densities (Jupiter densities) were
also calculated. For Planet 1 these were 0.738 +/-0.123, 0.141 AU and
0.340 +/-0.170 respectively. For Planet 2 these were 0.674 +/-0.135,
0.226 Au and 0.741 +/- 0.371.

The properties of the two exoplanets suggest that both exoplanets are
low density gas giants that have formed close to their host star. It was
concluded that the extrasolar system was unlike our own, with no
possibility of hosting in a form that we know. Spectroscopy was
suggested as was to follow up investigations to gain an understanding
into the exoplanet's atmospheres allowing us to determine the mass
compositions.

    \begin{Verbatim}[commandchars=\\\{\}]
{\color{incolor}In [{\color{incolor}1}]:} \PY{c+c1}{\PYZsh{}Necessary imports}
        \PY{k+kn}{from} \PY{n+nn}{astropy}\PY{n+nn}{.}\PY{n+nn}{io} \PY{k}{import} \PY{n}{fits} \PY{c+c1}{\PYZsh{}used to open fits files in which data are stored}
        \PY{k+kn}{import} \PY{n+nn}{pylab} \PY{c+c1}{\PYZsh{}plotting}
        \PY{k+kn}{import} \PY{n+nn}{glob}
        \PY{k+kn}{import} \PY{n+nn}{numpy} \PY{k}{as} \PY{n+nn}{np}
        \PY{k+kn}{import} \PY{n+nn}{matplotlib} \PY{k}{as} \PY{n+nn}{plt}
        \PY{k+kn}{import} \PY{n+nn}{photutils}
        \PY{k+kn}{import} \PY{n+nn}{pandas} \PY{k}{as} \PY{n+nn}{pd}
        \PY{k+kn}{import} \PY{n+nn}{scipy}\PY{n+nn}{.}\PY{n+nn}{signal}
        
        \PY{k+kn}{from} \PY{n+nn}{scipy}\PY{n+nn}{.}\PY{n+nn}{signal} \PY{k}{import} \PY{n}{savgol\PYZus{}filter}
        \PY{k+kn}{from} \PY{n+nn}{scipy} \PY{k}{import} \PY{n}{interpolate}
        \PY{k+kn}{from} \PY{n+nn}{scipy}\PY{n+nn}{.}\PY{n+nn}{optimize} \PY{k}{import} \PY{n}{curve\PYZus{}fit}
        \PY{k+kn}{from} \PY{n+nn}{scipy}\PY{n+nn}{.}\PY{n+nn}{signal} \PY{k}{import} \PY{n}{medfilt}
        \PY{k+kn}{from} \PY{n+nn}{photutils} \PY{k}{import} \PY{n}{aperture\PYZus{}photometry}
        \PY{k+kn}{from} \PY{n+nn}{imgdatareduction} \PY{k}{import} \PY{n}{ImageData}
        \PY{k+kn}{from} \PY{n+nn}{matplotlib}\PY{n+nn}{.}\PY{n+nn}{colors} \PY{k}{import} \PY{n}{LogNorm}
        \PY{k+kn}{from} \PY{n+nn}{scipy}\PY{n+nn}{.}\PY{n+nn}{signal} \PY{k}{import} \PY{n}{lombscargle}
        \PY{k+kn}{from} \PY{n+nn}{matplotlib} \PY{k}{import} \PY{n}{pyplot}
        
        \PY{k+kn}{import} \PY{n+nn}{seaborn} \PY{k}{as} \PY{n+nn}{sns}
        \PY{k+kn}{from} \PY{n+nn}{scipy} \PY{k}{import} \PY{n}{optimize}
        \PY{k+kn}{from} \PY{n+nn}{scipy}\PY{n+nn}{.}\PY{n+nn}{interpolate} \PY{k}{import} \PY{n}{interpn}
        \PY{k+kn}{from} \PY{n+nn}{CompSample} \PY{k}{import} \PY{n}{read\PYZus{}exopars}\PY{p}{,} \PY{n}{linfit}\PY{p}{,} \PY{n}{Zrecal}\PY{p}{,} \PY{n}{conflevels}\PY{p}{,} \PY{n}{density\PYZus{}scatter}
        
        \PY{n}{mykepler} \PY{o}{=} \PY{l+s+s1}{\PYZsq{}}\PY{l+s+s1}{1}\PY{l+s+s1}{\PYZsq{}} \PY{c+c1}{\PYZsh{}Data source number for object 1 is \PYZsq{}1\PYZsq{}}
\end{Verbatim}


    \subsubsection{1. Introduction}\label{introduction}

An exoplanet is any planetary body that exists outside of the Solar
System. Most exoplanets are detected in orbit around a host star or
stars. However, exoplanets can exist outside of extrasolar systems, but
detection of these exoplanets is difficult with current methods and
technology {[}1{]}. Exoplanets are difficult to directly image using
current telescopes because the luminosity of the host star is much
greater than that of the planet. Instead indirect methods of observing
the effects of the exoplanet on the star are used to confirm the
presence of an exoplanet. The `primary transit' and the `radial
velocity' are the most common methods of detecting exoplanets orbiting
around a host star. The purpose of this paper is to explore how the
primary transit method is used to detect an exoplanet. Using this method
one can determine the period and size of the exoplanet orbiting the star
{[}2{]}. When an exoplanet passes in front of a host star, the planet
blocks a small percentage of the light emitted from the star. This
change in flux can be detected if the planet's transit is in the plane
of observation. Knowing the baseline luminosity of the star from the
Kepler lightcurve one can calculate the change in flux, this is called
the transit depth. The transit depth is equal to the ratio areas of the
exoplanet and star {[}3{]}. Therefore, the radius of the exoplanet
cannot be measured from just the transit across the star. One must also
measure the radius of the star in order to calculate the exoplanet's
radius {[}3{]}. Rearranging equation 3, one can derive the exoplanets
semi major axis.

\(\text{Transit Depth} = (R_{exoplanet}/R_{star})^{2}\) (1)

\$ R\_\{exoplanet\} = R\_\{star\}*\sqrt{\text{Transit Depth}}\$ (2)

\$a\^{}\{3\} = \frac{G(M_{star}+M_{planet})}{4\pi ^{2}} P\^{}\{2\} (3)

In this paper one will demonstrate how to model a Kepler lightcurve and
calculate the transit depth, period, radius semi major axis and density
of an exoplanet using data taken from the Kepler space telescope. The
exoplanet and extrasolar system will be evaluated as to whether it could
sustain life in a form as we know it. Data reduction methods will be
implemented to create clear and comprehensible filtered data in order to
achieve the goals. Photometry is a common method of data reduction in
astrophysics. This involves calculating the signal to noise ratio for a
given image or data set {[}4{]}.

\(\text{SNR} = \text{Desired Signal}⁄Noise\) (4){[}4{]}.

In astronomy noise can stem from all manner of things. These range from
random fluctuations from the background sky to the in inherent noise in
the CCD chip and electronics {[}4{]}. These data reduction methods will
also be applied to a star in order to calculate its flux and magnitude.

    \subsubsection{2.1 Data Reduction - GROND}\label{data-reduction---grond}

Below are the steps that were taken to clean the GROND dataset. The
program stars by opening an image file. The image shows that the data
was taken over two separate CCD plates. Before any data reduction can be
done the two individual chip images need to be combined into one. In
order to do this the images were cropped to remove the borders between
and around the images. The combined CCD plate image shows that each chip
has a different bias level. A histogram of the subtraction bias for each
chip was created. It shows the average count per pixel. As expected, a
gaussian distribution of counts per pixel is recorded over each chip.
Correction for the bias was implemented by subtracting a bias frame
taken in the same filter as the raw image from the combined image. Now
that two images have the same bias, one can begin to reduce the
background noise in the image. One way of doing this is creating a flat
field image. The aim of this technique is to remove any variations in
pixel-to-pixel sensitivity or random noise from the image background. A
flat field image mask was applied to the GROND image data. After
subtracting the bias and the flat field image, one can begin to
calculate the flux and magnitude of a star in the image.

A function was written to calculate the flux and magnitude of a star in
the image. It used imported function from the photutils python library.
The first imported function is `CircularAperture', this function has the
purpose of plotting a circle of radius, r, centred at positions x and y,
where r, x, and y are all user defined variables. The second imported
function, `aperture\_photometry', is fed the user defined
`CircularAperture' and applies it to the cleaned data image. The outputs
of `aperture\_photometry' are the ID, the coordinate centre of the
aperture and the sum of the counts per pixel over the entire area, i.e.
flux.

Three areas and positions were defined, the first for the star being
measured, the second for a standard star of known magnitude and the
third for a measurement of the background flux. The first and second
fluxes given by the program are not the actual fluxes of the stars,
rather they are the flux emitted by the star plus the background noise
in the image. The program removed the background flux from each flux
measurement by taking the third circular aperture flux measurement and
dividing it by the circular aperture area. This gives average background
count per pixel (noise). The noise is then multiplied by the circular
aperture area for each star respectively. Finally, the program subtracts
the respective total noise per area from the unadjusted flux
measurements. The result is the emitted flux by each star. The program
then plots each respective circular aperture over the reduced data
image. This gives a visual representation of where each circle is
centred. The white circle is the aperture of the star being measured,
the blue circle is the standard star aperture and the red circle is the
background noise aperture.

Uncertainty was calculated on the flux measurements of each star and the
background noise. Using the standard error of the mean and the Poisson
noise.

\$ \sigma\_\{\text{aperture photometry}\} =
\sqrt{\sigma_{\text{source}}^{2} + \sigma_{\text{sky aperture}}^{2} + n_{\text{pixels}}*\text{(sky level error)}^{2} + n_{\text{pixels}}*\text{RON}^{2}}\$
(5)

\$ standard~error~of~the~mean~=~\frac{\sigma_{source}}{\sqrt N} \$ (6)

Using the uncertainty equations (5) and (6) the uncertainty in the flux
of each star was calculated by the program. The program then prints the
final background noise adjusted flux values for each star with their
uncertainties.

In order to find the magnitude of the star being measured, one needs to
know the Zero Point, this is the magnitude of an object with 1 count per
second on the CCD chip. There is a standard star in the image with a
known magnitude of +19.5. Using this and the standards stars flux
calculated earlier in the program, equation 7 was rearranged to find the
Zero Point. Now that the program has a value for the Zero Point it can
also calculate the magnitude of the star being measured. The stars
magnitude is calculated by the program and then printed along with a
value for the uncertainty. The uncertainty was derived from the
uncertainties calculated for the fluxes.

\$ Magnitude~=~Zero~Point~-~2.5\times\log\_\{10\}\{(Flux)\} \$ (7)

    \begin{Verbatim}[commandchars=\\\{\}]
{\color{incolor}In [{\color{incolor}2}]:} \PY{c+c1}{\PYZsh{}Opening the raw image fits data file}
        \PY{n}{raw} \PY{o}{=} \PY{l+s+s1}{\PYZsq{}}\PY{l+s+s1}{GROND\PYZus{}rband.246\PYZus{}0006.fits}\PY{l+s+s1}{\PYZsq{}}
        \PY{n}{image} \PY{o}{=} \PY{n}{ImageData}\PY{p}{(}\PY{n}{filen}\PY{o}{=}\PY{n}{raw}\PY{p}{)}
        \PY{c+c1}{\PYZsh{}Plotting the fits file}
        \PY{n}{pylab}\PY{o}{.}\PY{n}{imshow}\PY{p}{(}\PY{n}{image}\PY{o}{.}\PY{n}{raw}\PY{p}{,}\PY{n}{vmin}\PY{o}{=}\PY{l+m+mi}{500}\PY{p}{,}\PY{n}{vmax}\PY{o}{=}\PY{l+m+mi}{1000}\PY{p}{,}\PY{n}{norm}\PY{o}{=}\PY{n}{LogNorm}\PY{p}{(}\PY{p}{)}\PY{p}{)}
        \PY{n}{pylab}\PY{o}{.}\PY{n}{colorbar}\PY{p}{(}\PY{p}{)} \PY{c+c1}{\PYZsh{}colour shows visually how many counts per pixel (ie the flux)}
        \PY{n}{pylab}\PY{o}{.}\PY{n}{title}\PY{p}{(}\PY{l+s+s1}{\PYZsq{}}\PY{l+s+s1}{Raw Data Image}\PY{l+s+s1}{\PYZsq{}}\PY{p}{)}
\end{Verbatim}


\begin{Verbatim}[commandchars=\\\{\}]
{\color{outcolor}Out[{\color{outcolor}2}]:} Text(0.5,1,'Raw Data Image')
\end{Verbatim}
            
    \begin{center}
    \adjustimage{max size={0.9\linewidth}{0.9\paperheight}}{output_7_1.png}
    \end{center}
    { \hspace*{\fill} \\}
    
    \begin{Verbatim}[commandchars=\\\{\}]
{\color{incolor}In [{\color{incolor}3}]:} \PY{c+c1}{\PYZsh{}Cropping the image to remove borders}
        \PY{n}{pylab}\PY{o}{.}\PY{n}{imshow}\PY{p}{(}\PY{n}{image}\PY{o}{.}\PY{n}{rawtrim}\PY{p}{,}\PY{n}{vmin}\PY{o}{=}\PY{l+m+mi}{500}\PY{p}{,}\PY{n}{vmax}\PY{o}{=}\PY{l+m+mi}{1000}\PY{p}{,}\PY{n}{norm}\PY{o}{=}\PY{n}{LogNorm}\PY{p}{(}\PY{p}{)}\PY{p}{)}
        \PY{n}{pylab}\PY{o}{.}\PY{n}{colorbar}\PY{p}{(}\PY{p}{)}
        \PY{n}{pylab}\PY{o}{.}\PY{n}{title}\PY{p}{(}\PY{l+s+s1}{\PYZsq{}}\PY{l+s+s1}{Combined CCD image}\PY{l+s+s1}{\PYZsq{}}\PY{p}{)}
\end{Verbatim}


\begin{Verbatim}[commandchars=\\\{\}]
{\color{outcolor}Out[{\color{outcolor}3}]:} Text(0.5,1,'Combined CCD image')
\end{Verbatim}
            
    \begin{center}
    \adjustimage{max size={0.9\linewidth}{0.9\paperheight}}{output_8_1.png}
    \end{center}
    { \hspace*{\fill} \\}
    
    \begin{Verbatim}[commandchars=\\\{\}]
{\color{incolor}In [{\color{incolor}4}]:} \PY{c+c1}{\PYZsh{}Read in bias frame}
        \PY{n}{bias} \PY{o}{=} \PY{l+s+s1}{\PYZsq{}}\PY{l+s+s1}{bias\PYZus{}r.fits}\PY{l+s+s1}{\PYZsq{}} 
        \PY{n}{image}\PY{o}{.}\PY{n}{readbias}\PY{p}{(}\PY{n}{bias}\PY{p}{)}
        
        \PY{c+c1}{\PYZsh{} Let\PYZsq{}s plot the bias frame to see what it looks like}
        \PY{n}{pylab}\PY{o}{.}\PY{n}{imshow}\PY{p}{(}\PY{n}{image}\PY{o}{.}\PY{n}{bias}\PY{p}{,}\PY{n}{vmin}\PY{o}{=}\PY{l+m+mi}{150}\PY{p}{,}\PY{n}{vmax}\PY{o}{=}\PY{l+m+mi}{300}\PY{p}{)}
        \PY{n}{pylab}\PY{o}{.}\PY{n}{colorbar}\PY{p}{(}\PY{p}{)}
        \PY{n}{pylab}\PY{o}{.}\PY{n}{title}\PY{p}{(}\PY{l+s+s1}{\PYZsq{}}\PY{l+s+s1}{Image Bias}\PY{l+s+s1}{\PYZsq{}}\PY{p}{)}
\end{Verbatim}


\begin{Verbatim}[commandchars=\\\{\}]
{\color{outcolor}Out[{\color{outcolor}4}]:} Text(0.5,1,'Image Bias')
\end{Verbatim}
            
    \begin{center}
    \adjustimage{max size={0.9\linewidth}{0.9\paperheight}}{output_9_1.png}
    \end{center}
    { \hspace*{\fill} \\}
    
    \begin{Verbatim}[commandchars=\\\{\}]
{\color{incolor}In [{\color{incolor}5}]:} \PY{c+c1}{\PYZsh{}Chip 1, low\PYZhy{}level pixel to pixel level variation}
        \PY{n}{pylab}\PY{o}{.}\PY{n}{imshow}\PY{p}{(}\PY{n}{image}\PY{o}{.}\PY{n}{bias}\PY{p}{[}\PY{p}{:}\PY{p}{,}\PY{p}{:}\PY{o}{\PYZhy{}}\PY{l+m+mi}{1023}\PY{p}{]}\PY{p}{,}\PY{n}{vmin}\PY{o}{=}\PY{l+m+mi}{290}\PY{p}{,}\PY{n}{vmax}\PY{o}{=}\PY{l+m+mi}{300}\PY{p}{)}
        \PY{n}{pylab}\PY{o}{.}\PY{n}{colorbar}\PY{p}{(}\PY{p}{)}
        \PY{n}{pylab}\PY{o}{.}\PY{n}{title}\PY{p}{(}\PY{l+s+s1}{\PYZsq{}}\PY{l+s+s1}{Chip 1 pixel\PYZhy{}pixel variation}\PY{l+s+s1}{\PYZsq{}}\PY{p}{)}
\end{Verbatim}


\begin{Verbatim}[commandchars=\\\{\}]
{\color{outcolor}Out[{\color{outcolor}5}]:} Text(0.5,1,'Chip 1 pixel-pixel variation')
\end{Verbatim}
            
    \begin{center}
    \adjustimage{max size={0.9\linewidth}{0.9\paperheight}}{output_10_1.png}
    \end{center}
    { \hspace*{\fill} \\}
    
    \begin{Verbatim}[commandchars=\\\{\}]
{\color{incolor}In [{\color{incolor}6}]:} \PY{c+c1}{\PYZsh{} Plotting a histogram of the counts/pix along row 1000 for chip 1}
        \PY{n}{pylab}\PY{o}{.}\PY{n}{hist}\PY{p}{(}\PY{n}{image}\PY{o}{.}\PY{n}{bias}\PY{p}{[}\PY{l+m+mi}{1000}\PY{p}{:}\PY{l+m+mi}{1001}\PY{p}{,}\PY{p}{:}\PY{o}{\PYZhy{}}\PY{l+m+mi}{1023}\PY{p}{]}\PY{o}{.}\PY{n}{ravel}\PY{p}{(}\PY{p}{)}\PY{p}{)}
        \PY{n}{pylab}\PY{o}{.}\PY{n}{plot}\PY{p}{(}\PY{p}{[}\PY{n}{image}\PY{o}{.}\PY{n}{chip1\PYZus{}bl}\PY{p}{,}\PY{n}{image}\PY{o}{.}\PY{n}{chip1\PYZus{}bl}\PY{p}{]}\PY{p}{,}\PY{p}{[}\PY{l+m+mi}{0}\PY{p}{,}\PY{l+m+mi}{255}\PY{p}{]}\PY{p}{,}\PY{n}{ls}\PY{o}{=}\PY{l+s+s1}{\PYZsq{}}\PY{l+s+s1}{\PYZhy{}\PYZhy{}}\PY{l+s+s1}{\PYZsq{}}\PY{p}{,}\PY{n}{c}\PY{o}{=}\PY{l+s+s1}{\PYZsq{}}\PY{l+s+s1}{k}\PY{l+s+s1}{\PYZsq{}}\PY{p}{)}
        \PY{n}{pylab}\PY{o}{.}\PY{n}{ylim}\PY{p}{(}\PY{p}{[}\PY{l+m+mi}{0}\PY{p}{,}\PY{l+m+mi}{255}\PY{p}{]}\PY{p}{)}
        \PY{n}{pylab}\PY{o}{.}\PY{n}{title}\PY{p}{(}\PY{l+s+s1}{\PYZsq{}}\PY{l+s+s1}{Chip 1, row 1000 of the bias frame}\PY{l+s+s1}{\PYZsq{}}\PY{p}{)}
        \PY{n}{pylab}\PY{o}{.}\PY{n}{xlabel}\PY{p}{(}\PY{l+s+s1}{\PYZsq{}}\PY{l+s+s1}{Counts/pix}\PY{l+s+s1}{\PYZsq{}}\PY{p}{)}
        \PY{n}{pylab}\PY{o}{.}\PY{n}{ylabel}\PY{p}{(}\PY{l+s+s1}{\PYZsq{}}\PY{l+s+s1}{Number of pixels}\PY{l+s+s1}{\PYZsq{}}\PY{p}{)}
\end{Verbatim}


\begin{Verbatim}[commandchars=\\\{\}]
{\color{outcolor}Out[{\color{outcolor}6}]:} Text(0,0.5,'Number of pixels')
\end{Verbatim}
            
    \begin{center}
    \adjustimage{max size={0.9\linewidth}{0.9\paperheight}}{output_11_1.png}
    \end{center}
    { \hspace*{\fill} \\}
    
    \begin{Verbatim}[commandchars=\\\{\}]
{\color{incolor}In [{\color{incolor}7}]:} \PY{c+c1}{\PYZsh{} First let\PYZsq{}s plot a histogram of the counts/pix along row 1000 for chip 2}
        \PY{n}{pylab}\PY{o}{.}\PY{n}{hist}\PY{p}{(}\PY{n}{image}\PY{o}{.}\PY{n}{bias}\PY{p}{[}\PY{l+m+mi}{1000}\PY{p}{:}\PY{l+m+mi}{1001}\PY{p}{,}\PY{l+m+mi}{1200}\PY{p}{:}\PY{p}{]}\PY{o}{.}\PY{n}{ravel}\PY{p}{(}\PY{p}{)}\PY{p}{)}
        \PY{n}{pylab}\PY{o}{.}\PY{n}{plot}\PY{p}{(}\PY{p}{[}\PY{n}{image}\PY{o}{.}\PY{n}{chip2\PYZus{}bl}\PY{p}{,}\PY{n}{image}\PY{o}{.}\PY{n}{chip2\PYZus{}bl}\PY{p}{]}\PY{p}{,}\PY{p}{[}\PY{l+m+mi}{0}\PY{p}{,}\PY{l+m+mi}{300}\PY{p}{]}\PY{p}{,}\PY{n}{ls}\PY{o}{=}\PY{l+s+s1}{\PYZsq{}}\PY{l+s+s1}{\PYZhy{}\PYZhy{}}\PY{l+s+s1}{\PYZsq{}}\PY{p}{,}\PY{n}{c}\PY{o}{=}\PY{l+s+s1}{\PYZsq{}}\PY{l+s+s1}{k}\PY{l+s+s1}{\PYZsq{}}\PY{p}{)}
        \PY{n}{pylab}\PY{o}{.}\PY{n}{ylim}\PY{p}{(}\PY{p}{[}\PY{l+m+mi}{0}\PY{p}{,}\PY{l+m+mi}{255}\PY{p}{]}\PY{p}{)}
        \PY{n}{pylab}\PY{o}{.}\PY{n}{title}\PY{p}{(}\PY{l+s+s1}{\PYZsq{}}\PY{l+s+s1}{Chip 2, row 2000 of the bias frame}\PY{l+s+s1}{\PYZsq{}}\PY{p}{)}
        \PY{n}{pylab}\PY{o}{.}\PY{n}{xlabel}\PY{p}{(}\PY{l+s+s1}{\PYZsq{}}\PY{l+s+s1}{Counts/pix}\PY{l+s+s1}{\PYZsq{}}\PY{p}{)}
        \PY{n}{pylab}\PY{o}{.}\PY{n}{ylabel}\PY{p}{(}\PY{l+s+s1}{\PYZsq{}}\PY{l+s+s1}{Number of pixels}\PY{l+s+s1}{\PYZsq{}}\PY{p}{)}
\end{Verbatim}


\begin{Verbatim}[commandchars=\\\{\}]
{\color{outcolor}Out[{\color{outcolor}7}]:} Text(0,0.5,'Number of pixels')
\end{Verbatim}
            
    \begin{center}
    \adjustimage{max size={0.9\linewidth}{0.9\paperheight}}{output_12_1.png}
    \end{center}
    { \hspace*{\fill} \\}
    
    \begin{Verbatim}[commandchars=\\\{\}]
{\color{incolor}In [{\color{incolor}8}]:} \PY{c+c1}{\PYZsh{}Subtracting of image bias from the raw data image}
        \PY{n}{image}\PY{o}{.}\PY{n}{bias\PYZus{}subtract}\PY{p}{(}\PY{p}{)}
        
        \PY{c+c1}{\PYZsh{}Showing new bias subtraction image}
        \PY{n}{pylab}\PY{o}{.}\PY{n}{imshow}\PY{p}{(}\PY{n}{image}\PY{o}{.}\PY{n}{biassub}\PY{p}{,}\PY{n}{vmin}\PY{o}{=}\PY{l+m+mi}{300}\PY{p}{,}\PY{n}{vmax}\PY{o}{=}\PY{l+m+mi}{600}\PY{p}{,}\PY{n}{norm}\PY{o}{=}\PY{n}{LogNorm}\PY{p}{(}\PY{p}{)}\PY{p}{)}
        \PY{n}{pylab}\PY{o}{.}\PY{n}{colorbar}\PY{p}{(}\PY{p}{)}
        \PY{n}{pylab}\PY{o}{.}\PY{n}{title}\PY{p}{(}\PY{l+s+s1}{\PYZsq{}}\PY{l+s+s1}{Raw Image with Subtracted Bias}\PY{l+s+s1}{\PYZsq{}}\PY{p}{)}
\end{Verbatim}


\begin{Verbatim}[commandchars=\\\{\}]
{\color{outcolor}Out[{\color{outcolor}8}]:} Text(0.5,1,'Raw Image with Subtracted Bias')
\end{Verbatim}
            
    \begin{center}
    \adjustimage{max size={0.9\linewidth}{0.9\paperheight}}{output_13_1.png}
    \end{center}
    { \hspace*{\fill} \\}
    
    \begin{Verbatim}[commandchars=\\\{\}]
{\color{incolor}In [{\color{incolor}9}]:} \PY{c+c1}{\PYZsh{}Data reduction}
        \PY{n}{flat} \PY{o}{=} \PY{l+s+s1}{\PYZsq{}}\PY{l+s+s1}{flat\PYZus{}r.fits}\PY{l+s+s1}{\PYZsq{}}
        \PY{n}{image}\PY{o}{.}\PY{n}{readflat}\PY{p}{(}\PY{n}{flat}\PY{p}{)} \PY{c+c1}{\PYZsh{}readsremoves artifacting from pixel to pixel sensitivity variations}
        \PY{c+c1}{\PYZsh{}Showing the flat\PYZhy{}field frame}
        \PY{n}{pylab}\PY{o}{.}\PY{n}{imshow}\PY{p}{(}\PY{n}{image}\PY{o}{.}\PY{n}{flat}\PY{p}{,}\PY{n}{vmin}\PY{o}{=}\PY{l+m+mi}{49000}\PY{p}{,}\PY{n}{vmax}\PY{o}{=}\PY{l+m+mi}{50500}\PY{p}{)}
        \PY{n}{pylab}\PY{o}{.}\PY{n}{colorbar}\PY{p}{(}\PY{p}{)}
        \PY{n}{pylab}\PY{o}{.}\PY{n}{title}\PY{p}{(}\PY{l+s+s1}{\PYZsq{}}\PY{l+s+s1}{Flat Field Image}\PY{l+s+s1}{\PYZsq{}}\PY{p}{)}
\end{Verbatim}


\begin{Verbatim}[commandchars=\\\{\}]
{\color{outcolor}Out[{\color{outcolor}9}]:} Text(0.5,1,'Flat Field Image')
\end{Verbatim}
            
    \begin{center}
    \adjustimage{max size={0.9\linewidth}{0.9\paperheight}}{output_14_1.png}
    \end{center}
    { \hspace*{\fill} \\}
    
    \begin{Verbatim}[commandchars=\\\{\}]
{\color{incolor}In [{\color{incolor}10}]:} \PY{c+c1}{\PYZsh{} Normalisation of the flat field frame such that the counts = 1}
         \PY{n}{image}\PY{o}{.}\PY{n}{flat} \PY{o}{=} \PY{n}{image}\PY{o}{.}\PY{n}{flat}\PY{o}{/}\PY{n}{np}\PY{o}{.}\PY{n}{mean}\PY{p}{(}\PY{n}{image}\PY{o}{.}\PY{n}{flat}\PY{p}{)}
         
         \PY{c+c1}{\PYZsh{}Normalised flat field frame}
         \PY{n}{pylab}\PY{o}{.}\PY{n}{imshow}\PY{p}{(}\PY{n}{image}\PY{o}{.}\PY{n}{flat}\PY{p}{,}\PY{n}{vmin}\PY{o}{=}\PY{l+m+mf}{0.99}\PY{p}{,}\PY{n}{vmax}\PY{o}{=}\PY{l+m+mf}{1.02}\PY{p}{)}
         \PY{n}{pylab}\PY{o}{.}\PY{n}{colorbar}\PY{p}{(}\PY{p}{)}
         \PY{n}{pylab}\PY{o}{.}\PY{n}{title}\PY{p}{(}\PY{l+s+s1}{\PYZsq{}}\PY{l+s+s1}{Normalised Flat Field Image}\PY{l+s+s1}{\PYZsq{}}\PY{p}{)}
\end{Verbatim}


\begin{Verbatim}[commandchars=\\\{\}]
{\color{outcolor}Out[{\color{outcolor}10}]:} Text(0.5,1,'Normalised Flat Field Image')
\end{Verbatim}
            
    \begin{center}
    \adjustimage{max size={0.9\linewidth}{0.9\paperheight}}{output_15_1.png}
    \end{center}
    { \hspace*{\fill} \\}
    
    \begin{Verbatim}[commandchars=\\\{\}]
{\color{incolor}In [{\color{incolor}11}]:} \PY{c+c1}{\PYZsh{} Let\PYZsq{}s now flat\PYZhy{}field our data image}
         \PY{n}{image}\PY{o}{.}\PY{n}{flatfield}\PY{p}{(}\PY{p}{)}
         
         \PY{c+c1}{\PYZsh{} Final reduced data is bias\PYZhy{}subtracted and flat\PYZhy{}fielded }
         \PY{c+c1}{\PYZsh{} Showing reduced data fits image}
         \PY{n}{pylab}\PY{o}{.}\PY{n}{imshow}\PY{p}{(}\PY{n}{image}\PY{o}{.}\PY{n}{red}\PY{p}{,}\PY{n}{vmin}\PY{o}{=}\PY{l+m+mi}{250}\PY{p}{,}\PY{n}{vmax}\PY{o}{=}\PY{l+m+mi}{500}\PY{p}{,}\PY{n}{norm}\PY{o}{=}\PY{n}{LogNorm}\PY{p}{(}\PY{p}{)}\PY{p}{)}
         \PY{n}{pylab}\PY{o}{.}\PY{n}{colorbar}\PY{p}{(}\PY{p}{)}
         \PY{n}{pylab}\PY{o}{.}\PY{n}{title}\PY{p}{(}\PY{l+s+s1}{\PYZsq{}}\PY{l+s+s1}{Cleaned Data Image}\PY{l+s+s1}{\PYZsq{}}\PY{p}{)}
         \PY{c+c1}{\PYZsh{}Saving reduced image as a .fits file}
         \PY{n}{image}\PY{o}{.}\PY{n}{savefits}\PY{p}{(}\PY{n}{image}\PY{o}{.}\PY{n}{red}\PY{p}{,}\PY{l+s+s1}{\PYZsq{}}\PY{l+s+s1}{my\PYZus{}reddata}\PY{l+s+s1}{\PYZsq{}}\PY{p}{)}
\end{Verbatim}


    \begin{center}
    \adjustimage{max size={0.9\linewidth}{0.9\paperheight}}{output_16_0.png}
    \end{center}
    { \hspace*{\fill} \\}
    
    \subsubsection{2.2 GROND dataset - Star flux and magnitude
caluclations}\label{grond-dataset---star-flux-and-magnitude-caluclations}

    \begin{Verbatim}[commandchars=\\\{\}]
{\color{incolor}In [{\color{incolor}12}]:} \PY{k}{def} \PY{n+nf}{star\PYZus{}photometry}\PY{p}{(}\PY{p}{)}\PY{p}{:}
             
             
         
             \PY{c+c1}{\PYZsh{}CALCULATING FLUX AND MAGNITUDE OF A STAR FROM GROND}
         
             \PY{c+c1}{\PYZsh{}Defining the star area and finding its flux}
             \PY{n}{xpos} \PY{o}{=} \PY{l+m+mi}{1533} \PY{c+c1}{\PYZsh{}x\PYZhy{}position of object in image being measured}
             \PY{n}{ypos} \PY{o}{=} \PY{l+m+mi}{424}  \PY{c+c1}{\PYZsh{}y\PYZhy{}position of object in image being measured}
             \PY{n}{srcrad} \PY{o}{=} \PY{l+m+mi}{20} \PY{c+c1}{\PYZsh{}radius over which to measure the object\PYZsq{}s flux}
             \PY{n}{starsky} \PY{o}{=} \PY{n}{photutils}\PY{o}{.}\PY{n}{CircularAperture}\PY{p}{(}\PY{p}{(}\PY{n}{xpos}\PY{p}{,} \PY{n}{ypos}\PY{p}{)}\PY{p}{,} \PY{n}{srcrad}\PY{p}{)} \PY{c+c1}{\PYZsh{}plots circular aperture over image}
             \PY{n}{star\PYZus{}circle} \PY{o}{=} \PY{n}{aperture\PYZus{}photometry}\PY{p}{(}\PY{n}{image}\PY{o}{.}\PY{n}{red}\PY{p}{,} \PY{n}{starsky}\PY{p}{)} \PY{c+c1}{\PYZsh{}calculates the flux in a given area for a fits file}
             \PY{n}{starflux} \PY{o}{=} \PY{n+nb}{float}\PY{p}{(}\PY{n}{star\PYZus{}circle}\PY{p}{[}\PY{l+s+s1}{\PYZsq{}}\PY{l+s+s1}{aperture\PYZus{}sum}\PY{l+s+s1}{\PYZsq{}}\PY{p}{]}\PY{p}{)} \PY{c+c1}{\PYZsh{}converts the flux output to a float value for easy printing}
             \PY{n+nb}{print} \PY{p}{(}\PY{l+s+s1}{\PYZsq{}}\PY{l+s+s1}{The area over which the stars flux is measured is: }\PY{l+s+s1}{\PYZsq{}} \PY{l+s+s2}{\PYZdq{}}\PY{l+s+si}{\PYZob{}:.0f\PYZcb{}}\PY{l+s+s2}{\PYZdq{}}\PY{o}{.}\PY{n}{format}\PY{p}{(}\PY{n+nb}{float}\PY{p}{(}\PY{n}{starsky}\PY{o}{.}\PY{n}{area}\PY{p}{(}\PY{p}{)}\PY{p}{)}\PY{p}{)}\PY{p}{)}
             \PY{n+nb}{print} \PY{p}{(}\PY{l+s+s1}{\PYZsq{}}\PY{l+s+s1}{The total star flux (not ajusted for background noise) is: }\PY{l+s+s1}{\PYZsq{}}\PY{p}{,} \PY{l+s+s2}{\PYZdq{}}\PY{l+s+si}{\PYZob{}:.0f\PYZcb{}}\PY{l+s+s2}{\PYZdq{}}\PY{o}{.}\PY{n}{format}\PY{p}{(}\PY{n}{starflux}\PY{p}{)}\PY{p}{,} \PY{l+s+s1}{\PYZsq{}}\PY{l+s+s1}{counts}\PY{l+s+s1}{\PYZsq{}}\PY{p}{)}
         
         
             \PY{c+c1}{\PYZsh{}Defining the standard star area and finding its flux}
             \PY{n}{std\PYZus{}star\PYZus{}xpos}\PY{o}{=}\PY{l+m+mf}{1805.3} \PY{c+c1}{\PYZsh{}x\PYZhy{}position of object in image being measured}
             \PY{n}{std\PYZus{}star\PYZus{}ypos}\PY{o}{=}\PY{l+m+mf}{362.6}  \PY{c+c1}{\PYZsh{}y\PYZhy{}position of object in image being measured}
             \PY{n}{std\PYZus{}starrad} \PY{o}{=} \PY{l+m+mi}{20} \PY{c+c1}{\PYZsh{}radius over which to measure the object\PYZsq{}s flux}
             \PY{n}{std\PYZus{}starsky} \PY{o}{=} \PY{n}{photutils}\PY{o}{.}\PY{n}{CircularAperture}\PY{p}{(}\PY{p}{(}\PY{n}{std\PYZus{}star\PYZus{}xpos}\PY{p}{,} \PY{n}{std\PYZus{}star\PYZus{}ypos}\PY{p}{)}\PY{p}{,} \PY{n}{std\PYZus{}starrad}\PY{p}{)}
             \PY{n}{std\PYZus{}star\PYZus{}circle} \PY{o}{=} \PY{n}{aperture\PYZus{}photometry}\PY{p}{(}\PY{n}{image}\PY{o}{.}\PY{n}{red}\PY{p}{,} \PY{n}{std\PYZus{}starsky}\PY{p}{)}
             \PY{n}{std\PYZus{}starflux} \PY{o}{=} \PY{n+nb}{float}\PY{p}{(}\PY{n}{std\PYZus{}star\PYZus{}circle}\PY{p}{[}\PY{l+s+s1}{\PYZsq{}}\PY{l+s+s1}{aperture\PYZus{}sum}\PY{l+s+s1}{\PYZsq{}}\PY{p}{]}\PY{p}{)}
             \PY{n+nb}{print} \PY{p}{(}\PY{l+s+s1}{\PYZsq{}}\PY{l+s+s1}{The area over which the standard stars flux is measured is: }\PY{l+s+s1}{\PYZsq{}} \PY{l+s+s2}{\PYZdq{}}\PY{l+s+si}{\PYZob{}:.0f\PYZcb{}}\PY{l+s+s2}{\PYZdq{}}\PY{o}{.}\PY{n}{format}\PY{p}{(}\PY{n+nb}{float}\PY{p}{(}\PY{n}{std\PYZus{}starsky}\PY{o}{.}\PY{n}{area}\PY{p}{(}\PY{p}{)}\PY{p}{)}\PY{p}{)}\PY{p}{)}
             \PY{n+nb}{print} \PY{p}{(}\PY{l+s+s1}{\PYZsq{}}\PY{l+s+s1}{The total standard star flux (not ajusted for background noise) is: }\PY{l+s+s1}{\PYZsq{}}\PY{p}{,} \PY{l+s+s2}{\PYZdq{}}\PY{l+s+si}{\PYZob{}:.0f\PYZcb{}}\PY{l+s+s2}{\PYZdq{}}\PY{o}{.}\PY{n}{format}\PY{p}{(}\PY{n}{std\PYZus{}starflux}\PY{p}{)}\PY{p}{,} \PY{l+s+s1}{\PYZsq{}}\PY{l+s+s1}{counts}\PY{l+s+s1}{\PYZsq{}}\PY{p}{)}
         
         
             \PY{c+c1}{\PYZsh{}Defining the background sky area and finding its flux }
             \PY{n}{xpos1} \PY{o}{=} \PY{l+m+mi}{1900} \PY{c+c1}{\PYZsh{}x\PYZhy{}position of noise}
             \PY{n}{ypos1} \PY{o}{=} \PY{l+m+mi}{200}  \PY{c+c1}{\PYZsh{}y\PYZhy{}position of noise}
             \PY{n}{nrad}  \PY{o}{=} \PY{l+m+mi}{100}  \PY{c+c1}{\PYZsh{}radius over which to calculate background noise}
             \PY{n}{awaysky} \PY{o}{=} \PY{n}{photutils}\PY{o}{.}\PY{n}{CircularAperture}\PY{p}{(}\PY{p}{(}\PY{n}{xpos1}\PY{p}{,} \PY{n}{ypos1}\PY{p}{)}\PY{p}{,} \PY{n}{nrad}\PY{p}{)}
             \PY{n}{sky\PYZus{}circle} \PY{o}{=} \PY{n}{aperture\PYZus{}photometry}\PY{p}{(}\PY{n}{image}\PY{o}{.}\PY{n}{red}\PY{p}{,} \PY{n}{awaysky}\PY{p}{)}
             \PY{n}{skyflux} \PY{o}{=} \PY{n+nb}{float}\PY{p}{(}\PY{n}{sky\PYZus{}circle}\PY{p}{[}\PY{l+s+s1}{\PYZsq{}}\PY{l+s+s1}{aperture\PYZus{}sum}\PY{l+s+s1}{\PYZsq{}}\PY{p}{]}\PY{p}{)}
             \PY{n+nb}{print} \PY{p}{(}\PY{l+s+s1}{\PYZsq{}}\PY{l+s+s1}{The area over which background noise is calculated is: }\PY{l+s+s1}{\PYZsq{}} \PY{l+s+s2}{\PYZdq{}}\PY{l+s+si}{\PYZob{}:.0f\PYZcb{}}\PY{l+s+s2}{\PYZdq{}}\PY{o}{.}\PY{n}{format}\PY{p}{(}\PY{n+nb}{float}\PY{p}{(}\PY{n}{awaysky}\PY{o}{.}\PY{n}{area}\PY{p}{(}\PY{p}{)}\PY{p}{)}\PY{p}{)}\PY{p}{)}
             \PY{n+nb}{print} \PY{p}{(}\PY{l+s+s1}{\PYZsq{}}\PY{l+s+s1}{The total background noise in this area is: }\PY{l+s+s1}{\PYZsq{}}\PY{p}{,} \PY{l+s+s2}{\PYZdq{}}\PY{l+s+si}{\PYZob{}:.0f\PYZcb{}}\PY{l+s+s2}{\PYZdq{}}\PY{o}{.}\PY{n}{format}\PY{p}{(}\PY{n}{skyflux}\PY{p}{)}\PY{p}{,} \PY{l+s+s1}{\PYZsq{}}\PY{l+s+s1}{counts}\PY{l+s+s1}{\PYZsq{}}\PY{p}{)}
         
         
             \PY{c+c1}{\PYZsh{}plotting the measuiring radii onto the image}
             \PY{n}{pylab}\PY{o}{.}\PY{n}{xlim}\PY{p}{(}\PY{l+m+mi}{1100}\PY{p}{,}\PY{l+m+mi}{2000}\PY{p}{)}
             \PY{n}{pylab}\PY{o}{.}\PY{n}{ylim}\PY{p}{(}\PY{l+m+mi}{850}\PY{p}{,}\PY{l+m+mi}{50}\PY{p}{)}
             \PY{n}{pylab}\PY{o}{.}\PY{n}{imshow}\PY{p}{(}\PY{n}{image}\PY{o}{.}\PY{n}{red}\PY{p}{,}\PY{n}{vmin}\PY{o}{=}\PY{l+m+mi}{250}\PY{p}{,}\PY{n}{vmax}\PY{o}{=}\PY{l+m+mi}{500}\PY{p}{,}\PY{n}{norm}\PY{o}{=}\PY{n}{LogNorm}\PY{p}{(}\PY{p}{)}\PY{p}{)}
             \PY{n}{pylab}\PY{o}{.}\PY{n}{title}\PY{p}{(}\PY{l+s+s1}{\PYZsq{}}\PY{l+s+s1}{Cleaned Data Image}\PY{l+s+s1}{\PYZsq{}}\PY{p}{)}
             \PY{n}{starsky}\PY{o}{.}\PY{n}{plot}\PY{p}{(}\PY{n}{ec}\PY{o}{=}\PY{l+s+s1}{\PYZsq{}}\PY{l+s+s1}{white}\PY{l+s+s1}{\PYZsq{}}\PY{p}{)} \PY{c+c1}{\PYZsh{}the star measuring circle is plotted in white}
             \PY{n}{awaysky}\PY{o}{.}\PY{n}{plot}\PY{p}{(}\PY{n}{ec}\PY{o}{=}\PY{l+s+s1}{\PYZsq{}}\PY{l+s+s1}{red}\PY{l+s+s1}{\PYZsq{}}\PY{p}{)}   \PY{c+c1}{\PYZsh{}the background noise measuring circle is plotted in red}
             \PY{n}{std\PYZus{}starsky}\PY{o}{.}\PY{n}{plot}\PY{p}{(}\PY{n}{ec}\PY{o}{=}\PY{l+s+s1}{\PYZsq{}}\PY{l+s+s1}{blue}\PY{l+s+s1}{\PYZsq{}}\PY{p}{)} \PY{c+c1}{\PYZsh{}the standard star measuring circle is plotted in white}
         
             \PY{c+c1}{\PYZsh{}Calculating flux of just the star}
             \PY{c+c1}{\PYZsh{}1st calculate background noise per pixel}
         
             \PY{n}{f1} \PY{o}{=} \PY{n+nb}{float}\PY{p}{(}\PY{n}{sky\PYZus{}circle}\PY{p}{[}\PY{l+s+s1}{\PYZsq{}}\PY{l+s+s1}{aperture\PYZus{}sum}\PY{l+s+s1}{\PYZsq{}}\PY{p}{]}\PY{p}{)}\PY{o}{/}\PY{n}{awaysky}\PY{o}{.}\PY{n}{area}\PY{p}{(}\PY{p}{)} \PY{c+c1}{\PYZsh{}Skycircle flux per pixel for away sky dimensions}
             \PY{n}{f2} \PY{o}{=} \PY{n}{f1}\PY{o}{*}\PY{n}{starsky}\PY{o}{.}\PY{n}{area}\PY{p}{(}\PY{p}{)} \PY{c+c1}{\PYZsh{}background noise per pixel multiplied by pixel area for star}
         
             \PY{c+c1}{\PYZsh{}Star flux is the flux measured in starsky \PYZhy{} (background pixel noise per area * starsky area)}
             \PY{n+nb}{print} \PY{p}{(}\PY{l+s+s1}{\PYZsq{}}\PY{l+s+s1}{Background noise per pixel is:}\PY{l+s+s1}{\PYZsq{}}\PY{p}{,} \PY{l+s+s2}{\PYZdq{}}\PY{l+s+si}{\PYZob{}:.0f\PYZcb{}}\PY{l+s+s2}{\PYZdq{}}\PY{o}{.}\PY{n}{format}\PY{p}{(}\PY{n}{f1}\PY{p}{)}\PY{p}{,} \PY{l+s+s1}{\PYZsq{}}\PY{l+s+s1}{counts}\PY{l+s+s1}{\PYZsq{}}\PY{p}{)}
             \PY{n}{starflux\PYZus{}adjusted} \PY{o}{=} \PY{n+nb}{float}\PY{p}{(}\PY{n}{star\PYZus{}circle}\PY{p}{[}\PY{l+s+s1}{\PYZsq{}}\PY{l+s+s1}{aperture\PYZus{}sum}\PY{l+s+s1}{\PYZsq{}}\PY{p}{]}\PY{p}{)} \PY{o}{\PYZhy{}} \PY{n}{f1}\PY{o}{*}\PY{n}{starsky}\PY{o}{.}\PY{n}{area}\PY{p}{(}\PY{p}{)}
         
             \PY{n}{std\PYZus{}error} \PY{o}{=} \PY{n}{np}\PY{o}{.}\PY{n}{sqrt}\PY{p}{(}\PY{n}{sky\PYZus{}circle}\PY{p}{[}\PY{l+s+s1}{\PYZsq{}}\PY{l+s+s1}{aperture\PYZus{}sum}\PY{l+s+s1}{\PYZsq{}}\PY{p}{]}\PY{o}{/}\PY{p}{(}\PY{n+nb}{float}\PY{p}{(}\PY{n}{awaysky}\PY{o}{.}\PY{n}{area}\PY{p}{(}\PY{p}{)}\PY{p}{)}\PY{p}{)}\PY{p}{)} \PY{c+c1}{\PYZsh{}calculating standard error}
             \PY{n}{sigma} \PY{o}{=} \PY{n}{np}\PY{o}{.}\PY{n}{sqrt}\PY{p}{(}\PY{n}{star\PYZus{}circle}\PY{p}{[}\PY{l+s+s1}{\PYZsq{}}\PY{l+s+s1}{aperture\PYZus{}sum}\PY{l+s+s1}{\PYZsq{}}\PY{p}{]} \PY{o}{+} \PY{n}{std\PYZus{}error}\PY{o}{*}\PY{o}{*}\PY{l+m+mi}{2}\PY{p}{)}
             \PY{n}{sigmaf} \PY{o}{=} \PY{n+nb}{float}\PY{p}{(}\PY{n}{sigma}\PY{p}{)}
             \PY{n+nb}{print} \PY{p}{(}\PY{l+s+s1}{\PYZsq{}}\PY{l+s+s1}{The flux emitted by the star (noise adjusted) is: }\PY{l+s+s1}{\PYZsq{}}\PY{p}{,} \PY{l+s+s2}{\PYZdq{}}\PY{l+s+si}{\PYZob{}:.0f\PYZcb{}}\PY{l+s+s2}{\PYZdq{}}\PY{o}{.}\PY{n}{format}\PY{p}{(}\PY{n}{starflux\PYZus{}adjusted}\PY{p}{)}\PY{p}{,}\PY{l+s+s1}{\PYZsq{}}\PY{l+s+s1}{+/\PYZhy{}}\PY{l+s+s1}{\PYZsq{}}\PY{p}{,} \PY{l+s+s2}{\PYZdq{}}\PY{l+s+si}{\PYZob{}:.1g\PYZcb{}}\PY{l+s+s2}{\PYZdq{}}\PY{o}{.}\PY{n}{format}\PY{p}{(}\PY{n}{sigmaf}\PY{p}{)}\PY{p}{,}\PY{l+s+s1}{\PYZsq{}}\PY{l+s+s1}{counts}\PY{l+s+s1}{\PYZsq{}}\PY{p}{)}
         
         
             \PY{c+c1}{\PYZsh{}Calculating flux of just the standard star}
             \PY{c+c1}{\PYZsh{}1st calculate background noise per pixel}
         
             \PY{n}{std\PYZus{}f1} \PY{o}{=} \PY{n+nb}{float}\PY{p}{(}\PY{n}{sky\PYZus{}circle}\PY{p}{[}\PY{l+s+s1}{\PYZsq{}}\PY{l+s+s1}{aperture\PYZus{}sum}\PY{l+s+s1}{\PYZsq{}}\PY{p}{]}\PY{p}{)}\PY{o}{/}\PY{n}{awaysky}\PY{o}{.}\PY{n}{area}\PY{p}{(}\PY{p}{)} \PY{c+c1}{\PYZsh{}Skycircle flux per pixel for away sky dimensions}
             \PY{n}{std\PYZus{}f2} \PY{o}{=} \PY{n}{f1}\PY{o}{*}\PY{n}{std\PYZus{}starsky}\PY{o}{.}\PY{n}{area}\PY{p}{(}\PY{p}{)} \PY{c+c1}{\PYZsh{}background noise per pixel multiplied by pixel area for star}
         
             \PY{c+c1}{\PYZsh{}Standard Star flux is the flux measured in std\PYZus{}starsky \PYZhy{} (background pixel noise per area * std\PYZus{}starsky area)}
             \PY{n}{std\PYZus{}starflux\PYZus{}adjusted} \PY{o}{=} \PY{n+nb}{float}\PY{p}{(}\PY{n}{std\PYZus{}star\PYZus{}circle}\PY{p}{[}\PY{l+s+s1}{\PYZsq{}}\PY{l+s+s1}{aperture\PYZus{}sum}\PY{l+s+s1}{\PYZsq{}}\PY{p}{]}\PY{p}{)} \PY{o}{\PYZhy{}} \PY{n}{f1}\PY{o}{*}\PY{n}{std\PYZus{}starsky}\PY{o}{.}\PY{n}{area}\PY{p}{(}\PY{p}{)}
             \PY{n}{std\PYZus{}sigma} \PY{o}{=} \PY{n}{np}\PY{o}{.}\PY{n}{sqrt}\PY{p}{(}\PY{n}{std\PYZus{}star\PYZus{}circle}\PY{p}{[}\PY{l+s+s1}{\PYZsq{}}\PY{l+s+s1}{aperture\PYZus{}sum}\PY{l+s+s1}{\PYZsq{}}\PY{p}{]} \PY{o}{+} \PY{n}{std\PYZus{}error}\PY{o}{*}\PY{o}{*}\PY{l+m+mi}{2}\PY{p}{)} 
             \PY{n}{std\PYZus{}sigmaf} \PY{o}{=} \PY{n+nb}{float}\PY{p}{(}\PY{n}{std\PYZus{}sigma}\PY{p}{)}
             \PY{n+nb}{print} \PY{p}{(}\PY{l+s+s1}{\PYZsq{}}\PY{l+s+s1}{The flux emitted by the standard star (noise adjusted) is: }\PY{l+s+s1}{\PYZsq{}}\PY{p}{,} \PY{l+s+s2}{\PYZdq{}}\PY{l+s+si}{\PYZob{}:.0f\PYZcb{}}\PY{l+s+s2}{\PYZdq{}}\PY{o}{.}\PY{n}{format}\PY{p}{(}\PY{n}{std\PYZus{}starflux\PYZus{}adjusted}\PY{p}{)}\PY{p}{,}\PY{l+s+s1}{\PYZsq{}}\PY{l+s+s1}{+/\PYZhy{}}\PY{l+s+s1}{\PYZsq{}}\PY{p}{,} \PY{l+s+s2}{\PYZdq{}}\PY{l+s+si}{\PYZob{}:.1g\PYZcb{}}\PY{l+s+s2}{\PYZdq{}}\PY{o}{.}\PY{n}{format}\PY{p}{(}\PY{n}{sigmaf}\PY{p}{)}\PY{p}{,}\PY{l+s+s1}{\PYZsq{}}\PY{l+s+s1}{counts}\PY{l+s+s1}{\PYZsq{}}\PY{p}{)}
         
             \PY{c+c1}{\PYZsh{}Zeropoint calculation}
             \PY{c+c1}{\PYZsh{}Zeropoint is the magnitude that corresponds to a source with a flux of 1 cts/sec}
             \PY{n}{zeropoint} \PY{o}{=}  \PY{l+m+mf}{19.50} \PY{o}{+} \PY{l+m+mf}{2.5}\PY{o}{*}\PY{n}{np}\PY{o}{.}\PY{n}{log}\PY{p}{(}\PY{p}{(}\PY{n}{std\PYZus{}starflux\PYZus{}adjusted}\PY{p}{)}\PY{p}{)}
             \PY{n+nb}{print} \PY{p}{(}\PY{l+s+s1}{\PYZsq{}}\PY{l+s+s1}{The magnitude of the zero point is: }\PY{l+s+s1}{\PYZsq{}} \PY{l+s+s2}{\PYZdq{}}\PY{l+s+si}{\PYZob{}:.4g\PYZcb{}}\PY{l+s+s2}{\PYZdq{}}\PY{o}{.}\PY{n}{format}\PY{p}{(}\PY{n}{zeropoint}\PY{p}{)}\PY{p}{)}
             
             \PY{c+c1}{\PYZsh{}calculating the magnitude of the measured star}
             \PY{n}{Mag\PYZus{}star} \PY{o}{=} \PY{n}{zeropoint} \PY{o}{\PYZhy{}} \PY{l+m+mf}{2.5}\PY{o}{*}\PY{n}{np}\PY{o}{.}\PY{n}{log}\PY{p}{(}\PY{n}{starflux\PYZus{}adjusted}\PY{p}{)}
             \PY{n}{Mag\PYZus{}star\PYZus{}err} \PY{o}{=} \PY{n}{Mag\PYZus{}star}\PY{o}{*}\PY{n}{np}\PY{o}{.}\PY{n}{sqrt}\PY{p}{(}\PY{p}{(}\PY{n}{sigmaf}\PY{o}{/}\PY{n+nb}{float}\PY{p}{(}\PY{n}{star\PYZus{}circle}\PY{p}{[}\PY{l+s+s1}{\PYZsq{}}\PY{l+s+s1}{aperture\PYZus{}sum}\PY{l+s+s1}{\PYZsq{}}\PY{p}{]}\PY{p}{)}\PY{p}{)}\PY{o}{*}\PY{o}{*}\PY{l+m+mi}{2} \PY{o}{+} \PY{p}{(}\PY{n}{std\PYZus{}sigmaf}\PY{o}{/}\PY{n+nb}{float}\PY{p}{(}\PY{n}{std\PYZus{}star\PYZus{}circle}\PY{p}{[}\PY{l+s+s1}{\PYZsq{}}\PY{l+s+s1}{aperture\PYZus{}sum}\PY{l+s+s1}{\PYZsq{}}\PY{p}{]}\PY{p}{)}\PY{p}{)}\PY{p}{)}
             \PY{n+nb}{print} \PY{p}{(}\PY{l+s+s1}{\PYZsq{}}\PY{l+s+s1}{The magnitude of the star is: }\PY{l+s+s1}{\PYZsq{}}\PY{p}{,} \PY{l+s+s2}{\PYZdq{}}\PY{l+s+si}{\PYZob{}:.4g\PYZcb{}}\PY{l+s+s2}{\PYZdq{}}\PY{o}{.}\PY{n}{format}\PY{p}{(}\PY{n}{Mag\PYZus{}star}\PY{p}{)}\PY{p}{,} \PY{l+s+s1}{\PYZsq{}}\PY{l+s+s1}{+/\PYZhy{}}\PY{l+s+s1}{\PYZsq{}}\PY{p}{,} \PY{l+s+s2}{\PYZdq{}}\PY{l+s+si}{\PYZob{}:.2g\PYZcb{}}\PY{l+s+s2}{\PYZdq{}}\PY{o}{.}\PY{n}{format}\PY{p}{(}\PY{n}{Mag\PYZus{}star\PYZus{}err}\PY{p}{)}\PY{p}{)}
         
             
         \PY{n}{star\PYZus{}photometry}\PY{p}{(}\PY{p}{)}
\end{Verbatim}


    \begin{Verbatim}[commandchars=\\\{\}]
The area over which the stars flux is measured is: 1257
The total star flux (not ajusted for background noise) is:  804313 counts
The area over which the standard stars flux is measured is: 1257
The total standard star flux (not ajusted for background noise) is:  631952 counts
The area over which background noise is calculated is: 31416
The total background noise in this area is:  12679688 counts
Background noise per pixel is: 404 counts
The flux emitted by the star (noise adjusted) is:  297125 +/- 9e+02 counts
The flux emitted by the standard star (noise adjusted) is:  124765 +/- 9e+02 counts
The magnitude of the zero point is: 48.84
The magnitude of the star is:  17.33 +/- 0.62

    \end{Verbatim}

    \begin{center}
    \adjustimage{max size={0.9\linewidth}{0.9\paperheight}}{output_18_1.png}
    \end{center}
    { \hspace*{\fill} \\}
    
    \subsubsection{2.2 Data Reduction - Kepler
Lightcurve}\label{data-reduction---kepler-lightcurve}

    Two sets of data were provided by Kepler, each corresponding to separate
stellar systems. For the purposes of this paper, only the dataset for -
'Object 1' will be considered. This dataset contains continuous flux and
time observations of the designated star system. As such, lightcurves
were constructed to identify the transits of exoplanets. The data was
provided in the form of many separate lightcurve files, which are listed
below.

Before undertaking analysis of the entire dataset listed above, a subset
of the data was initially looked at. The subset was plotted with on a
graph of time in days against flux. Looking at the first graph below we
can see that the stars flux randomly fluctuates with time between around
41500 and 41700 counts. However, we can clearly see large periodic dips
in the flux emitted by the star over a short amount of time. These dips
in the lightcurve are attributed to the transiting of an extrasolar
planet in front of the face of the star. Now that we have observed a
planet transiting the face of the star in the plane of observation, we
can start to data analysis on the object to calculate its period,
radius, semi major axis and density and whether or not the planet is
habitable.

    \begin{Verbatim}[commandchars=\\\{\}]
{\color{incolor}In [{\color{incolor}13}]:} \PY{c+c1}{\PYZsh{}Fits files are open like this:}
         \PY{n}{object\PYZus{}1} \PY{o}{=} \PY{n}{fits}\PY{o}{.}\PY{n}{open}\PY{p}{(}\PY{l+s+s1}{\PYZsq{}}\PY{l+s+s1}{Data/Object}\PY{l+s+si}{\PYZpc{}s}\PY{l+s+s1}{lc/kplr}\PY{l+s+si}{\PYZpc{}s}\PY{l+s+s1}{\PYZus{}1.fits}\PY{l+s+s1}{\PYZsq{}}\PY{o}{\PYZpc{}}\PY{p}{(}\PY{n}{mykepler}\PY{p}{,} \PY{n}{mykepler}\PY{p}{)}\PY{p}{)}
         \PY{n}{object\PYZus{}1}\PY{o}{.}\PY{n}{info}\PY{p}{(}\PY{p}{)}
         
         \PY{c+c1}{\PYZsh{}the data in the extension can be stored like this:}
         \PY{n}{object\PYZus{}1}\PY{p}{[}\PY{l+m+mi}{0}\PY{p}{]}\PY{o}{.}\PY{n}{data}
         \PY{c+c1}{\PYZsh{}the first extension is empty, it just has the information}
         
         \PY{c+c1}{\PYZsh{}The last extension holds a small image of the source:}
         \PY{n}{object\PYZus{}1}\PY{p}{[}\PY{l+m+mi}{2}\PY{p}{]}\PY{o}{.}\PY{n}{header}
         \PY{n}{pylab}\PY{o}{.}\PY{n}{imshow}\PY{p}{(}\PY{n}{object\PYZus{}1}\PY{p}{[}\PY{l+m+mi}{2}\PY{p}{]}\PY{o}{.}\PY{n}{data}\PY{p}{)}
         
         \PY{c+c1}{\PYZsh{}lets have a look at the second extension}
         \PY{n}{object\PYZus{}1}\PY{p}{[}\PY{l+m+mi}{1}\PY{p}{]}
         \PY{c+c1}{\PYZsh{}This is a table extension}
         
         \PY{c+c1}{\PYZsh{}Here the beginning has a description of columns in the table, including the units}
         \PY{n}{object\PYZus{}1}\PY{p}{[}\PY{l+m+mi}{1}\PY{p}{]}\PY{o}{.}\PY{n}{header}
\end{Verbatim}


    \begin{Verbatim}[commandchars=\\\{\}]
Filename: Data/Object1lc/kplr1\_1.fits
No.    Name      Ver    Type      Cards   Dimensions   Format
  0  PRIMARY       1 PrimaryHDU      59   ()      
  1  LIGHTCURVE    1 BinTableHDU    162   4370R x 20C   [D, E, J, E, E, E, E, E, E, J, D, E, D, E, D, E, D, E, E, E]   
  2  APERTURE      1 ImageHDU        49   (6, 6)   int32   

    \end{Verbatim}

\begin{Verbatim}[commandchars=\\\{\}]
{\color{outcolor}Out[{\color{outcolor}13}]:} XTENSION= 'BINTABLE'           / marks the beginning of a new HDU               
         BITPIX  =                    8 / array data type                                
         NAXIS   =                    2 / number of array dimensions                     
         NAXIS1  =                  100 / length of first array dimension                
         NAXIS2  =                 4370 / length of second array dimension               
         PCOUNT  =                    0 / group parameter count (not used)               
         GCOUNT  =                    1 / group count (not used)                         
         TFIELDS =                   20 / number of table fields                         
         TTYPE1  = 'TIME    '           / column title: data time stamps                 
         TFORM1  = 'D       '           / column format: 64-bit floating point           
         TUNIT1  = 'BJD - 2454833'      / column units: barycenter corrected JD          
         TDISP1  = 'D14.7   '           / column display format                          
         TTYPE2  = 'TIMECORR'           / column title: barycenter - timeslice correction
         TFORM2  = 'E       '           / column format: 32-bit floating point           
         TUNIT2  = 'd       '           / column units: day                              
         TDISP2  = 'E13.6   '           / column display format                          
         TTYPE3  = 'CADENCENO'          / column title: unique cadence number            
         TFORM3  = 'J       '           / column format: signed 32-bit integer           
         TDISP3  = 'I10     '           / column display format                          
         TTYPE4  = 'SAP\_FLUX'           / column title: aperture photometry flux         
         TFORM4  = 'E       '           / column format: 32-bit floating point           
         TUNIT4  = 'e-/s    '           / column units: electrons per second             
         TDISP4  = 'E14.7   '           / column display format                          
         TTYPE5  = 'SAP\_FLUX\_ERR'       / column title: aperture phot. flux error        
         TFORM5  = 'E       '           / column format: 32-bit floating point           
         TUNIT5  = 'e-/s    '           / column units: electrons per second (1-sigma)   
         TDISP5  = 'E14.7   '           / column display format                          
         TTYPE6  = 'SAP\_BKG '           / column title: aperture phot. background flux   
         TFORM6  = 'E       '           / column format: 32-bit floating point           
         TUNIT6  = 'e-/s    '           / column units: electrons per second             
         TDISP6  = 'E14.7   '           / column display format                          
         TTYPE7  = 'SAP\_BKG\_ERR'        / column title: ap. phot. background flux error  
         TFORM7  = 'E       '           / column format: 32-bit floating point           
         TUNIT7  = 'e-/s    '           / column units: electrons per second (1-sigma)   
         TDISP7  = 'E14.7   '           / column display format                          
         TTYPE8  = 'PDCSAP\_FLUX'        / column title: aperture phot. PDC flux          
         TFORM8  = 'E       '           / column format: 32-bit floating point           
         TUNIT8  = 'e-/s    '           / column units: electrons per second             
         TDISP8  = 'E14.7   '           / column display format                          
         TTYPE9  = 'PDCSAP\_FLUX\_ERR'    / column title: ap. phot. PDC flux error         
         TFORM9  = 'E       '           / column format: 32-bit floating point           
         TUNIT9  = 'e-/s    '           / column units: electrons per second (1-sigma)   
         TDISP9  = 'E14.7   '           / column display format                          
         TTYPE10 = 'SAP\_QUALITY'        / column title: aperture photometry quality flag 
         TFORM10 = 'J       '           / column format: signed 32-bit integer           
         TDISP10 = 'B16.16  '           / column display format                          
         TTYPE11 = 'PSF\_CENTR1'         / column title: PSF-fitted column centroid       
         TFORM11 = 'D       '           / column format: 64-bit floating point           
         TUNIT11 = 'pixel   '           / column units: pixel                            
         TDISP11 = 'F10.5   '           / column display format                          
         TTYPE12 = 'PSF\_CENTR1\_ERR'     / column title: PSF-fitted column error          
         TFORM12 = 'E       '           / column format: 32-bit floating point           
         TUNIT12 = 'pixel   '           / column units: pixel (1-sigma)                  
         TDISP12 = 'E14.7   '           / column display format                          
         TTYPE13 = 'PSF\_CENTR2'         / column title: PSF-fitted row centroid          
         TFORM13 = 'D       '           / column format: 64-bit floating point           
         TUNIT13 = 'pixel   '           / column units: pixel                            
         TDISP13 = 'F10.5   '           / column display format                          
         TTYPE14 = 'PSF\_CENTR2\_ERR'     / column title: PSF-fitted row error             
         TFORM14 = 'E       '           / column format: 32-bit floating point           
         TUNIT14 = 'pixel   '           / column units: pixel (1-sigma)                  
         TDISP14 = 'E14.7   '           / column display format                          
         TTYPE15 = 'MOM\_CENTR1'         / column title: moment-derived column centroid   
         TFORM15 = 'D       '           / column format: 64-bit floating point           
         TUNIT15 = 'pixel   '           / column units: pixel                            
         TDISP15 = 'F10.5   '           / column display format                          
         TTYPE16 = 'MOM\_CENTR1\_ERR'     / column title: moment-derived column error      
         TFORM16 = 'E       '           / column format: 32-bit floating point           
         TUNIT16 = 'pixel   '           / column units: pixel (1-sigma)                  
         TDISP16 = 'E14.7   '           / column display format                          
         TTYPE17 = 'MOM\_CENTR2'         / column title: moment-derived row centroid      
         TFORM17 = 'D       '           / column format: 64-bit floating point           
         TUNIT17 = 'pixel   '           / column units: pixel                            
         TDISP17 = 'F10.5   '           / column display format                          
         TTYPE18 = 'MOM\_CENTR2\_ERR'     / column title: moment-derived row error         
         TFORM18 = 'E       '           / column format: 32-bit floating point           
         TUNIT18 = 'pixel   '           / column units: pixel (1-sigma)                  
         TDISP18 = 'E14.7   '           / column display format                          
         TTYPE19 = 'POS\_CORR1'          / column title: column position correction       
         TFORM19 = 'E       '           / column format: 32-bit floating point           
         TUNIT19 = 'pixels  '           / column units: pixel                            
         TDISP19 = 'E14.7   '           / column display format                          
         TTYPE20 = 'POS\_CORR2'          / column title: row position correction          
         TFORM20 = 'E       '           / column format: 32-bit floating point           
         TUNIT20 = 'pixels  '           / column units: pixel                            
         TDISP20 = 'E14.7   '           / column display format                          
         INHERIT =                    T / inherit the primary header                     
         EXTNAME = 'LIGHTCURVE'         / name of extension                              
         EXTVER  =                    1 / extension version number (not format version)  
         TELESCOP= 'Kepler  '           / telescope                                      
         INSTRUME= 'Kepler Photometer'  / detector type                                  
         OBJECT  = 'None of your business' / string version of target id                 
         KEPLERID= 'None of your business' / unique Kepler target identifier             
         RADESYS = 'ICRS    '           / reference frame of celestial coordinates       
         RA\_OBJ  = 'None of your business' / [deg] right ascension                       
         DEC\_OBJ = 'None of your business' / [deg] declination                           
         EQUINOX =               2000.0 / equinox of celestial coordinate system         
         EXPOSURE=          82.20427266 / [d] time on source                             
         TIMEREF = 'SOLARSYSTEM'        / barycentric correction applied to times        
         TASSIGN = 'SPACECRAFT'         / where time is assigned                         
         TIMESYS = 'TDB     '           / time system is barycentric JD                  
         BJDREFI =              2454833 / integer part of BJD reference date             
         BJDREFF =           0.00000000 / fraction of the day in BJD reference date      
         TIMEUNIT= 'd       '           / time unit for TIME, TSTART and TSTOP           
         TELAPSE =          89.29084789 / [d] TSTOP - TSTART                             
         LIVETIME=          82.20427266 / [d] TELAPSE multiplied by DEADC                
         TSTART  =         260.21442006 / observation start time in BJD-BJDREF           
         TSTOP   =         349.50526794 / observation stop time in BJD-BJDREF            
         LC\_START=       55092.72220972 / mid point of first cadence in MJD              
         LC\_END  =       55181.99659822 / mid point of last cadence in MJD               
         DEADC   =           0.92063492 / deadtime correction                            
         TIMEPIXR=                  0.5 / bin time beginning=0 middle=0.5 end=1          
         TIERRELA=             5.78E-07 / [d] relative time error                        
         TIERABSO=                      / [d] absolute time error                        
         INT\_TIME=       6.019802903270 / [s] photon accumulation time per frame         
         READTIME=       0.518948526144 / [s] readout time per frame                     
         FRAMETIM=       6.538751429414 / [s] frame time (INT\_TIME + READTIME)           
         NUM\_FRM =                  270 / number of frames per time stamp                
         TIMEDEL =     0.02043359821692 / [d] time resolution of data                    
         DATE-OBS= '2009-09-18T17:05:16.190Z' / TSTART as UTC calendar date              
         DATE-END= '2009-12-17T00:09:48.815Z' / TSTOP as UTC calendar date               
         BACKAPP =                    T / background is subtracted                       
         DEADAPP =                    T / deadtime applied                               
         VIGNAPP =                    T / vignetting or collimator correction applied    
         GAIN    =               105.23 / [electrons/count] channel gain                 
         READNOIS=            84.594397 / [electrons] read noise                         
         NREADOUT=                  270 / number of read per cadence                     
         TIMSLICE=                    3 / time-slice readout sequence section            
         MEANBLCK=                  733 / [count] FSW mean black level                   
         LCFXDOFF=               419400 / long cadence fixed offset                      
         SCFXDOFF=               219400 / short cadence fixed offset                     
         CDPP3\_0 =   101.72696685791016 / [ppm] RMS CDPP on 3.0-hr time scales           
         CDPP6\_0 =     125.456787109375 / [ppm] RMS CDPP on 6.0-hr time scales           
         CDPP12\_0=   110.96907806396484 / [ppm] RMS CDPP on 12.0-hr time scales          
         CROWDSAP=               0.9992 / Ratio of target flux to total flux in op. ap.  
         FLFRCSAP=               0.7252 / Frac. of target flux w/in the op. aperture     
         NSPSDDET=                    0 / Number of SPSDs detected                       
         NSPSDCOR=                    0 / Number of SPSDs corrected                      
         PDCVAR  =   1.8069709539413452 / Target variability                             
         PDCMETHD= 'multiScaleMap'      / PDC algorithm used for target                  
         NUMBAND =                    3 / Number of scale bands                          
         FITTYPE1= 'robust  '           / Fit type used for band 1                       
         PR\_GOOD1=                  0.0 / Prior goodness for band 1                      
         PR\_WGHT1=                  0.0 / Prior weight for band 1                        
         FITTYPE2= 'prior   '           / Fit type used for band 2                       
         PR\_GOOD2=  0.07292769849300385 / Prior goodness for band 2                      
         PR\_WGHT2=   11.492071151733398 / Prior weight for band 2                        
         FITTYPE3= 'prior   '           / Fit type used for band 3                       
         PR\_GOOD3=    0.516323983669281 / Prior goodness for band 3                      
         PR\_WGHT3=    81.36321258544922 / Prior weight for band 3                        
         PDC\_TOT =   0.7119521498680115 / PDC total goodness metric for target           
         PDC\_TOTP=   1.5745707750320435 / PDC\_TOT percentile compared to mod/out         
         PDC\_COR =   0.9627798199653625 / PDC correlation goodness metric for target     
         PDC\_CORP=    9.833599090576172 / PDC\_COR percentile compared to mod/out         
         PDC\_VAR =   0.9373531937599182 / PDC variability goodness metric for target     
         PDC\_VARP=      7.6667160987854 / PDC\_VAR percentile compared to mod/out         
         PDC\_NOI =   0.7827537655830383 / PDC noise goodness metric for target           
         PDC\_NOIP=    8.062134742736816 / PDC\_NOI percentile compared to mod/out         
         PDC\_EPT =                  1.0 / PDC earth point goodness metric for target     
         PDC\_EPTP=    54.63613510131836 / PDC\_EPT percentile compared to mod/out         
         CHECKSUM= 'hhO8jhM6hhM6hhM6'   / HDU checksum updated 2019-09-17T17:12:24       
         DATASUM = '3847933012'         / data unit checksum updated 2019-09-17T17:12:24 
\end{Verbatim}
            
    \begin{center}
    \adjustimage{max size={0.9\linewidth}{0.9\paperheight}}{output_21_2.png}
    \end{center}
    { \hspace*{\fill} \\}
    
    \begin{Verbatim}[commandchars=\\\{\}]
{\color{incolor}In [{\color{incolor}14}]:} \PY{c+c1}{\PYZsh{}Plotting the lightcurves as a function of time and flux:}
         \PY{c+c1}{\PYZsh{}This plots the barycentre time and flux in e/s}
         \PY{c+c1}{\PYZsh{}Measurements errors are also included, you can plot them as follows}
         
         \PY{c+c1}{\PYZsh{}Plotting time and flux to see how the system behaves over time and if there are any transits}
         \PY{n}{pylab}\PY{o}{.}\PY{n}{plot}\PY{p}{(}\PY{n}{object\PYZus{}1}\PY{p}{[}\PY{l+m+mi}{1}\PY{p}{]}\PY{o}{.}\PY{n}{data}\PY{p}{[}\PY{l+s+s1}{\PYZsq{}}\PY{l+s+s1}{TIME}\PY{l+s+s1}{\PYZsq{}}\PY{p}{]}\PY{p}{,} \PY{n}{object\PYZus{}1}\PY{p}{[}\PY{l+m+mi}{1}\PY{p}{]}\PY{o}{.}\PY{n}{data}\PY{p}{[}\PY{l+s+s1}{\PYZsq{}}\PY{l+s+s1}{PDCSAP\PYZus{}FLUX}\PY{l+s+s1}{\PYZsq{}}\PY{p}{]}\PY{p}{,} \PY{n}{ls}\PY{o}{=}\PY{l+s+s1}{\PYZsq{}}\PY{l+s+s1}{None}\PY{l+s+s1}{\PYZsq{}}\PY{p}{,} \PY{n}{marker}\PY{o}{=}\PY{l+s+s1}{\PYZsq{}}\PY{l+s+s1}{.}\PY{l+s+s1}{\PYZsq{}}\PY{p}{)}
         \PY{n}{pylab}\PY{o}{.}\PY{n}{errorbar}\PY{p}{(}\PY{n}{object\PYZus{}1}\PY{p}{[}\PY{l+m+mi}{1}\PY{p}{]}\PY{o}{.}\PY{n}{data}\PY{p}{[}\PY{l+s+s1}{\PYZsq{}}\PY{l+s+s1}{TIME}\PY{l+s+s1}{\PYZsq{}}\PY{p}{]}\PY{p}{,} \PY{n}{object\PYZus{}1}\PY{p}{[}\PY{l+m+mi}{1}\PY{p}{]}\PY{o}{.}\PY{n}{data}\PY{p}{[}\PY{l+s+s1}{\PYZsq{}}\PY{l+s+s1}{PDCSAP\PYZus{}FLUX}\PY{l+s+s1}{\PYZsq{}}\PY{p}{]}\PY{p}{,} \PY{n}{object\PYZus{}1}\PY{p}{[}\PY{l+m+mi}{1}\PY{p}{]}\PY{o}{.}\PY{n}{data}\PY{p}{[}\PY{l+s+s1}{\PYZsq{}}\PY{l+s+s1}{PDCSAP\PYZus{}FLUX\PYZus{}ERR}\PY{l+s+s1}{\PYZsq{}}\PY{p}{]}\PY{p}{,} \PY{n}{ls}\PY{o}{=}\PY{l+s+s1}{\PYZsq{}}\PY{l+s+s1}{None}\PY{l+s+s1}{\PYZsq{}}\PY{p}{,} \PY{n}{marker}\PY{o}{=}\PY{l+s+s1}{\PYZsq{}}\PY{l+s+s1}{.}\PY{l+s+s1}{\PYZsq{}}\PY{p}{)}
         \PY{n}{pylab}\PY{o}{.}\PY{n}{xlabel}\PY{p}{(}\PY{l+s+s1}{\PYZsq{}}\PY{l+s+s1}{Time [days]}\PY{l+s+s1}{\PYZsq{}}\PY{p}{)}
         \PY{n}{pylab}\PY{o}{.}\PY{n}{ylabel}\PY{p}{(}\PY{l+s+s1}{\PYZsq{}}\PY{l+s+s1}{Flux}\PY{l+s+s1}{\PYZsq{}}\PY{p}{)}
         \PY{c+c1}{\PYZsh{}The error bars are small, so we need to zoom in to see them}
\end{Verbatim}


\begin{Verbatim}[commandchars=\\\{\}]
{\color{outcolor}Out[{\color{outcolor}14}]:} Text(0,0.5,'Flux')
\end{Verbatim}
            
    \begin{center}
    \adjustimage{max size={0.9\linewidth}{0.9\paperheight}}{output_22_1.png}
    \end{center}
    { \hspace*{\fill} \\}
    
    \begin{Verbatim}[commandchars=\\\{\}]
{\color{incolor}In [{\color{incolor}15}]:} \PY{c+c1}{\PYZsh{}Plotting the lightcurve again}
         \PY{n}{pylab}\PY{o}{.}\PY{n}{plot}\PY{p}{(}\PY{n}{object\PYZus{}1}\PY{p}{[}\PY{l+m+mi}{1}\PY{p}{]}\PY{o}{.}\PY{n}{data}\PY{p}{[}\PY{l+s+s1}{\PYZsq{}}\PY{l+s+s1}{TIME}\PY{l+s+s1}{\PYZsq{}}\PY{p}{]}\PY{p}{,} \PY{n}{object\PYZus{}1}\PY{p}{[}\PY{l+m+mi}{1}\PY{p}{]}\PY{o}{.}\PY{n}{data}\PY{p}{[}\PY{l+s+s1}{\PYZsq{}}\PY{l+s+s1}{PDCSAP\PYZus{}FLUX}\PY{l+s+s1}{\PYZsq{}}\PY{p}{]}\PY{p}{,} \PY{n}{ls}\PY{o}{=}\PY{l+s+s1}{\PYZsq{}}\PY{l+s+s1}{None}\PY{l+s+s1}{\PYZsq{}}\PY{p}{,} \PY{n}{marker}\PY{o}{=}\PY{l+s+s1}{\PYZsq{}}\PY{l+s+s1}{.}\PY{l+s+s1}{\PYZsq{}}\PY{p}{)}
         
         \PY{c+c1}{\PYZsh{}Measurements errors are also included, you can plot them as follows}
         \PY{n}{pylab}\PY{o}{.}\PY{n}{errorbar}\PY{p}{(}\PY{n}{object\PYZus{}1}\PY{p}{[}\PY{l+m+mi}{1}\PY{p}{]}\PY{o}{.}\PY{n}{data}\PY{p}{[}\PY{l+s+s1}{\PYZsq{}}\PY{l+s+s1}{TIME}\PY{l+s+s1}{\PYZsq{}}\PY{p}{]}\PY{p}{,} \PY{n}{object\PYZus{}1}\PY{p}{[}\PY{l+m+mi}{1}\PY{p}{]}\PY{o}{.}\PY{n}{data}\PY{p}{[}\PY{l+s+s1}{\PYZsq{}}\PY{l+s+s1}{PDCSAP\PYZus{}FLUX}\PY{l+s+s1}{\PYZsq{}}\PY{p}{]}\PY{p}{,} \PY{n}{object\PYZus{}1}\PY{p}{[}\PY{l+m+mi}{1}\PY{p}{]}\PY{o}{.}\PY{n}{data}\PY{p}{[}\PY{l+s+s1}{\PYZsq{}}\PY{l+s+s1}{PDCSAP\PYZus{}FLUX\PYZus{}ERR}\PY{l+s+s1}{\PYZsq{}}\PY{p}{]}\PY{p}{,} \PY{n}{ls}\PY{o}{=}\PY{l+s+s1}{\PYZsq{}}\PY{l+s+s1}{None}\PY{l+s+s1}{\PYZsq{}}\PY{p}{,} \PY{n}{marker}\PY{o}{=}\PY{l+s+s1}{\PYZsq{}}\PY{l+s+s1}{.}\PY{l+s+s1}{\PYZsq{}}\PY{p}{)}
         
         \PY{c+c1}{\PYZsh{}We\PYZsq{}ll need to zoom in to see them:}
         \PY{c+c1}{\PYZsh{}Zoom in over a transit}
         \PY{n}{pylab}\PY{o}{.}\PY{n}{xlim}\PY{p}{(}\PY{l+m+mi}{270}\PY{p}{,} \PY{l+m+mi}{290}\PY{p}{)}
         \PY{n}{pylab}\PY{o}{.}\PY{n}{ylim}\PY{p}{(}\PY{l+m+mi}{41200}\PY{p}{,} \PY{l+m+mi}{41700}\PY{p}{)}
         \PY{n}{pylab}\PY{o}{.}\PY{n}{xlabel}\PY{p}{(}\PY{l+s+s1}{\PYZsq{}}\PY{l+s+s1}{Time [days]}\PY{l+s+s1}{\PYZsq{}}\PY{p}{)}
         \PY{n}{pylab}\PY{o}{.}\PY{n}{ylabel}\PY{p}{(}\PY{l+s+s1}{\PYZsq{}}\PY{l+s+s1}{Flux}\PY{l+s+s1}{\PYZsq{}}\PY{p}{)}
\end{Verbatim}


\begin{Verbatim}[commandchars=\\\{\}]
{\color{outcolor}Out[{\color{outcolor}15}]:} Text(0,0.5,'Flux')
\end{Verbatim}
            
    \begin{center}
    \adjustimage{max size={0.9\linewidth}{0.9\paperheight}}{output_23_1.png}
    \end{center}
    { \hspace*{\fill} \\}
    
    for lcfile in glob.glob('Data/Object\%slc/kplr*.fits' \%(mykepler)): tmp
= fits.open(lcfile) tmptime = (tmp{[}1{]}.data{[}'TIME'{]}) tmpflux =
(tmp{[}1{]}.data{[}'PDCSAP\_FLUX'{]}) tmperror =
(tmp{[}1{]}.data{[}'PDCSAP\_FLUX\_ERR'{]}) pylab.plot(tmptime, tmpflux,
c='k')

    Starting with just the one fits file, the data reduction techniques were
tested, these techniques were then applied to the enitre Kepler data
set.

When inspecting the data, there are places that are missing data or
filled with values that don't make sense, therefore the data needs to be
reduced. It is common practice to fill missing data points with inf
values and nan value. Some functions in the `numpy' and `scipy' python
libraries don't work when there and inf/nan values in the array, so we
need to apply a filter to remove them. Applying an inf/nan mask will
make it such that the when the array is passed into functions the
inf/nan points will be passed over by the function.

    \begin{Verbatim}[commandchars=\\\{\}]
{\color{incolor}In [{\color{incolor}16}]:} \PY{c+c1}{\PYZsh{}Numpy/Scipy functions do not work with Inf and Nan values}
         \PY{c+c1}{\PYZsh{}Lets create some new variables to which we can apply an Inf/Nan filter to}
         
         \PY{n}{t\PYZus{}nofilter} \PY{o}{=} \PY{n}{object\PYZus{}1}\PY{p}{[}\PY{l+m+mi}{1}\PY{p}{]}\PY{o}{.}\PY{n}{data}\PY{p}{[}\PY{l+s+s1}{\PYZsq{}}\PY{l+s+s1}{TIME}\PY{l+s+s1}{\PYZsq{}}\PY{p}{]} \PY{c+c1}{\PYZsh{}time column of fits file}
         \PY{n}{f\PYZus{}nofilter} \PY{o}{=} \PY{n}{object\PYZus{}1}\PY{p}{[}\PY{l+m+mi}{1}\PY{p}{]}\PY{o}{.}\PY{n}{data}\PY{p}{[}\PY{l+s+s1}{\PYZsq{}}\PY{l+s+s1}{PDCSAP\PYZus{}FLUX}\PY{l+s+s1}{\PYZsq{}}\PY{p}{]} \PY{c+c1}{\PYZsh{}point source flux data from the fits file}
         \PY{n}{e\PYZus{}nofilter} \PY{o}{=} \PY{n}{object\PYZus{}1}\PY{p}{[}\PY{l+m+mi}{1}\PY{p}{]}\PY{o}{.}\PY{n}{data}\PY{p}{[}\PY{l+s+s1}{\PYZsq{}}\PY{l+s+s1}{PDCSAP\PYZus{}FLUX\PYZus{}ERR}\PY{l+s+s1}{\PYZsq{}}\PY{p}{]} \PY{c+c1}{\PYZsh{}point source flux error from the fits file}
         
         \PY{c+c1}{\PYZsh{}Applying a mask filter to remove nan points}
         
         \PY{n}{nonan} \PY{o}{=} \PY{o}{\PYZti{}}\PY{n}{np}\PY{o}{.}\PY{n}{isnan}\PY{p}{(}\PY{n}{f\PYZus{}nofilter}\PY{p}{)}
         \PY{n}{t1} \PY{o}{=} \PY{n}{t\PYZus{}nofilter}\PY{p}{[}\PY{n}{nonan}\PY{p}{]}
         \PY{n}{f1} \PY{o}{=} \PY{n}{f\PYZus{}nofilter}\PY{p}{[}\PY{n}{nonan}\PY{p}{]}
         \PY{n}{e1} \PY{o}{=} \PY{n}{e\PYZus{}nofilter}\PY{p}{[}\PY{n}{nonan}\PY{p}{]}
\end{Verbatim}


    Now that our data has been cleaned of these inf/nan data points. We can
normalise the flux emitted by the star. This will give us a value for
the average flux emitted by the star. We have three options of filters
to use, a Savtisky-Galoy function, a spline filter or and median filter.
The best filter is the one that fits lightcurve of the star whilst
ignoring the transits of the planet. At low kernel sizes the
Savitsky-Galoy and spline filter both matched the lightcurve of the star
well, but they could not exclude the exoplanet dips. In order to exclude
the dips from the data, a high kernel size had to be used with both
filters, this outputted a normalisation lightcurve that had large
fluctuations in the stars flux. The spline filter is also a demanding
function to execute and takes a long time to complete. The median filter
was preferred over the other two filter functions as it was able to
match the lightcurve and avoid the planet transits whilst also being
relatively quick to execute.

The medium filter works by finding the median value between two limits
in the data. It then moves in the positive x direction by the difference
between the two limits. The function iterates over the whole array,
creating a median point value for every iteration. The limits correspond
to a kernel size, i.e. the amount of data points over which the median
is determined. The filter outputs these values into an array which can
then be used to normalise the data by dividing the flux array by this
array. The kernel size was chosen to be 51, this was chosen by looking
at the normalisation output graph and trying different values to get the
flattest normalisation lightcurve with the strongest planet dips.

We now have a flat lightcurve of fluxes with only dips from exoplanet
transits. Another use of normalising the curve is it shows us the
percentage dip in flux caused by the exoplanet as a decimal value.

The normalised lightcurve is important, not just because it removes the
star activity, but because in order to create a periodogram for the
star, we need a flat lightcurve. The periodogram will pick up the random
flux fluctuations and plot them as periods, we don't want this in our
data because this will lead to false conclusions on the number of
exoplanets in the extrasolar system. Therefore, by using a normalised
lightcurve most random flux variations are removed and the periodogram
will be easier to interpret.

    \begin{Verbatim}[commandchars=\\\{\}]
{\color{incolor}In [{\color{incolor}17}]:} \PY{c+c1}{\PYZsh{}kernel\PYZus{}size is the size of the window over which the median is calculated }
         
         \PY{n}{medfil} \PY{o}{=} \PY{n}{medfilt}\PY{p}{(}\PY{n}{f1}\PY{p}{,} \PY{n}{kernel\PYZus{}size}\PY{o}{=}\PY{l+m+mi}{21}\PY{p}{)} \PY{c+c1}{\PYZsh{}talk about why i chose median filter and the kernel size}
         
         \PY{n}{pylab}\PY{o}{.}\PY{n}{plot}\PY{p}{(}\PY{n}{t1}\PY{p}{,} \PY{n}{f1}\PY{p}{,} \PY{n}{c}\PY{o}{=}\PY{l+s+s1}{\PYZsq{}}\PY{l+s+s1}{k}\PY{l+s+s1}{\PYZsq{}}\PY{p}{,} \PY{n}{ls}\PY{o}{=}\PY{l+s+s1}{\PYZsq{}}\PY{l+s+s1}{None}\PY{l+s+s1}{\PYZsq{}}\PY{p}{,} \PY{n}{marker}\PY{o}{=}\PY{l+s+s1}{\PYZsq{}}\PY{l+s+s1}{.}\PY{l+s+s1}{\PYZsq{}}\PY{p}{,} \PY{n}{label}\PY{o}{=}\PY{l+s+s1}{\PYZsq{}}\PY{l+s+s1}{Data}\PY{l+s+s1}{\PYZsq{}}\PY{p}{)}
         \PY{n}{pylab}\PY{o}{.}\PY{n}{plot}\PY{p}{(}\PY{n}{t1}\PY{p}{,} \PY{n}{medfil}\PY{p}{,} \PY{n}{ls}\PY{o}{=}\PY{l+s+s1}{\PYZsq{}}\PY{l+s+s1}{\PYZhy{}\PYZhy{}}\PY{l+s+s1}{\PYZsq{}}\PY{p}{,} \PY{n}{c}\PY{o}{=}\PY{l+s+s1}{\PYZsq{}}\PY{l+s+s1}{r}\PY{l+s+s1}{\PYZsq{}}\PY{p}{,} \PY{n}{label}\PY{o}{=}\PY{l+s+s1}{\PYZsq{}}\PY{l+s+s1}{Median Filter}\PY{l+s+s1}{\PYZsq{}}\PY{p}{)}
         \PY{n}{pylab}\PY{o}{.}\PY{n}{xlim}\PY{p}{(}\PY{l+m+mi}{259}\PY{p}{,} \PY{l+m+mi}{351}\PY{p}{)}
         \PY{n}{pylab}\PY{o}{.}\PY{n}{ylim}\PY{p}{(}\PY{l+m+mi}{41200}\PY{p}{,} \PY{l+m+mi}{42100}\PY{p}{)}
         \PY{n}{pylab}\PY{o}{.}\PY{n}{xlabel}\PY{p}{(}\PY{l+s+s1}{\PYZsq{}}\PY{l+s+s1}{Time [days]}\PY{l+s+s1}{\PYZsq{}}\PY{p}{)}
         \PY{n}{pylab}\PY{o}{.}\PY{n}{ylabel}\PY{p}{(}\PY{l+s+s1}{\PYZsq{}}\PY{l+s+s1}{Flux}\PY{l+s+s1}{\PYZsq{}}\PY{p}{)}
\end{Verbatim}


\begin{Verbatim}[commandchars=\\\{\}]
{\color{outcolor}Out[{\color{outcolor}17}]:} Text(0,0.5,'Flux')
\end{Verbatim}
            
    \begin{center}
    \adjustimage{max size={0.9\linewidth}{0.9\paperheight}}{output_28_1.png}
    \end{center}
    { \hspace*{\fill} \\}
    
    \begin{Verbatim}[commandchars=\\\{\}]
{\color{incolor}In [{\color{incolor}18}]:} \PY{c+c1}{\PYZsh{}plotting normalised median}
         
         \PY{n}{flux\PYZus{}normalised} \PY{o}{=} \PY{n}{f1}\PY{o}{/}\PY{n}{medfil} \PY{c+c1}{\PYZsh{}Normalise flux by dividing by the median filter}
         \PY{n}{flux\PYZus{}error\PYZus{}normalised} \PY{o}{=} \PY{n}{e1}\PY{o}{/}\PY{n}{medfil} \PY{c+c1}{\PYZsh{}Normalise flux error by dividing by the median filter}
         
         \PY{n}{pylab}\PY{o}{.}\PY{n}{plot}\PY{p}{(}\PY{n}{t1}\PY{p}{,} \PY{n}{flux\PYZus{}normalised}\PY{p}{,} \PY{n}{ls}\PY{o}{=}\PY{l+s+s1}{\PYZsq{}}\PY{l+s+s1}{None}\PY{l+s+s1}{\PYZsq{}}\PY{p}{,} \PY{n}{marker}\PY{o}{=}\PY{l+s+s1}{\PYZsq{}}\PY{l+s+s1}{.}\PY{l+s+s1}{\PYZsq{}}\PY{p}{,} \PY{n}{c}\PY{o}{=}\PY{l+s+s1}{\PYZsq{}}\PY{l+s+s1}{grey}\PY{l+s+s1}{\PYZsq{}}\PY{p}{,} \PY{n}{label}\PY{o}{=}\PY{l+s+s1}{\PYZsq{}}\PY{l+s+s1}{Data}\PY{l+s+s1}{\PYZsq{}}\PY{p}{)}
         \PY{n}{pylab}\PY{o}{.}\PY{n}{xlabel}\PY{p}{(}\PY{l+s+s1}{\PYZsq{}}\PY{l+s+s1}{Time [days]}\PY{l+s+s1}{\PYZsq{}}\PY{p}{)}
         \PY{n}{pylab}\PY{o}{.}\PY{n}{ylabel}\PY{p}{(}\PY{l+s+s1}{\PYZsq{}}\PY{l+s+s1}{Normalised Flux}\PY{l+s+s1}{\PYZsq{}}\PY{p}{)}
\end{Verbatim}


\begin{Verbatim}[commandchars=\\\{\}]
{\color{outcolor}Out[{\color{outcolor}18}]:} Text(0,0.5,'Normalised Flux')
\end{Verbatim}
            
    \begin{center}
    \adjustimage{max size={0.9\linewidth}{0.9\paperheight}}{output_29_1.png}
    \end{center}
    { \hspace*{\fill} \\}
    
    \subsubsection{3. Results - Modelling of Kepler
lightcurve}\label{results---modelling-of-kepler-lightcurve}

    Now that the data has been reduced, the periods of any exoplanets
orbiting around the host star can be determined. Each periodic dip in
the flux in the lightcurve corresponds to an exoplanet transiting the
face of the star. Therefore, by applying the lombscargle periodogram
function to the lightcurve data set, one would expect an output of the
periods in days of each transiting planet. The lombscargle function acts
similarly to a Fourier transform from the time domain to the frequency
domain. The first application of the function produces a plot, but there
is a high level of aliasing. This was reduced by reapplying the
lombscargle function a second time to the data. This gave a plot of lots
of periods, inspecting the graph one can see the periods of two planets
orbiting the star. Planet 1 has a period of 19.237 days and Planet 2 has
a period of 38.98 days. These periods are displayed clearly by the black
vertical lines. A second transit from Planet 1 can be seen next to the
period of Planet 2. There is still aliasing in the plot, but it occurs
at periodic intervals, the second planet is distinguished from the
low-level aliasing because it does not follow the same pattern
behaviours as the low-level peaks.

    \begin{Verbatim}[commandchars=\\\{\}]
{\color{incolor}In [{\color{incolor}19}]:} \PY{c+c1}{\PYZsh{}Finding the period of exoplanets using periodgrams}
         
         \PY{n}{freq} \PY{o}{=} \PY{n}{np}\PY{o}{.}\PY{n}{linspace}\PY{p}{(}\PY{l+m+mi}{1}\PY{o}{/}\PY{l+m+mi}{130}\PY{p}{,}\PY{l+m+mi}{130}\PY{p}{,} \PY{l+m+mi}{13000}\PY{p}{)} \PY{c+c1}{\PYZsh{}defining the minimum amd maximum sampling time in the frequency space}
         
         \PY{n}{lomb} \PY{o}{=} \PY{n}{scipy}\PY{o}{.}\PY{n}{signal}\PY{o}{.}\PY{n}{lombscargle}\PY{p}{(}\PY{n}{t1}\PY{p}{,} \PY{n}{flux\PYZus{}normalised}\PY{p}{,} \PY{n}{freq}\PY{p}{,} \PY{n}{precenter}\PY{o}{=} \PY{k+kc}{True}\PY{p}{)} 
         
         \PY{c+c1}{\PYZsh{}lomb is the first transform from the time domain to the frequency domain}
         
         \PY{n}{pylab}\PY{o}{.}\PY{n}{plot}\PY{p}{(}\PY{n}{freq}\PY{p}{,} \PY{n}{lomb}\PY{p}{)}
         \PY{n}{pyplot}\PY{o}{.}\PY{n}{xlabel}\PY{p}{(}\PY{l+s+s1}{\PYZsq{}}\PY{l+s+s1}{Time (days)}\PY{l+s+s1}{\PYZsq{}}\PY{p}{)}
\end{Verbatim}


\begin{Verbatim}[commandchars=\\\{\}]
{\color{outcolor}Out[{\color{outcolor}19}]:} Text(0.5,0,'Time (days)')
\end{Verbatim}
            
    \begin{center}
    \adjustimage{max size={0.9\linewidth}{0.9\paperheight}}{output_32_1.png}
    \end{center}
    { \hspace*{\fill} \\}
    
    \begin{Verbatim}[commandchars=\\\{\}]
{\color{incolor}In [{\color{incolor}20}]:} \PY{c+c1}{\PYZsh{}Second application of the Lomscargle function gives the periods}
         
         \PY{n}{lomb\PYZus{}1} \PY{o}{=} \PY{n}{scipy}\PY{o}{.}\PY{n}{signal}\PY{o}{.}\PY{n}{lombscargle}\PY{p}{(}\PY{n}{freq}\PY{p}{,} \PY{n}{lomb}\PY{p}{,} \PY{n}{freq}\PY{p}{,} \PY{n}{precenter}\PY{o}{=}\PY{k+kc}{True}\PY{p}{)}
         \PY{n}{pylab}\PY{o}{.}\PY{n}{plot}\PY{p}{(}\PY{n}{freq}\PY{p}{,} \PY{n}{lomb\PYZus{}1}\PY{p}{)}
         \PY{n}{pyplot}\PY{o}{.}\PY{n}{xlabel}\PY{p}{(}\PY{l+s+s1}{\PYZsq{}}\PY{l+s+s1}{Time (days)}\PY{l+s+s1}{\PYZsq{}}\PY{p}{)}
\end{Verbatim}


\begin{Verbatim}[commandchars=\\\{\}]
{\color{outcolor}Out[{\color{outcolor}20}]:} Text(0.5,0,'Time (days)')
\end{Verbatim}
            
    \begin{center}
    \adjustimage{max size={0.9\linewidth}{0.9\paperheight}}{output_33_1.png}
    \end{center}
    { \hspace*{\fill} \\}
    
    \begin{Verbatim}[commandchars=\\\{\}]
{\color{incolor}In [{\color{incolor}21}]:} \PY{n}{pylab}\PY{o}{.}\PY{n}{plot}\PY{p}{(}\PY{n}{freq}\PY{p}{,} \PY{n}{lomb\PYZus{}1}\PY{p}{)}
         \PY{n}{pylab}\PY{o}{.}\PY{n}{xlim}\PY{p}{(}\PY{o}{\PYZhy{}}\PY{l+m+mf}{0.10}\PY{p}{,}\PY{l+m+mi}{60}\PY{p}{)}
         \PY{n}{pylab}\PY{o}{.}\PY{n}{ylim}\PY{p}{(}\PY{l+m+mf}{0.0}\PY{p}{,}\PY{l+m+mf}{0.0000000006}\PY{p}{)}
         \PY{n}{pyplot}\PY{o}{.}\PY{n}{xlabel}\PY{p}{(}\PY{l+s+s1}{\PYZsq{}}\PY{l+s+s1}{Time (days)}\PY{l+s+s1}{\PYZsq{}}\PY{p}{)}
\end{Verbatim}


\begin{Verbatim}[commandchars=\\\{\}]
{\color{outcolor}Out[{\color{outcolor}21}]:} Text(0.5,0,'Time (days)')
\end{Verbatim}
            
    \begin{center}
    \adjustimage{max size={0.9\linewidth}{0.9\paperheight}}{output_34_1.png}
    \end{center}
    { \hspace*{\fill} \\}
    
    \begin{Verbatim}[commandchars=\\\{\}]
{\color{incolor}In [{\color{incolor}22}]:} \PY{n}{pylab}\PY{o}{.}\PY{n}{plot}\PY{p}{(}\PY{n}{freq}\PY{p}{,} \PY{n}{lomb\PYZus{}1}\PY{p}{)}
         \PY{n}{pylab}\PY{o}{.}\PY{n}{xlim}\PY{p}{(}\PY{l+m+mi}{10}\PY{p}{,}\PY{l+m+mi}{50}\PY{p}{)}
         \PY{n}{pylab}\PY{o}{.}\PY{n}{ylim}\PY{p}{(}\PY{l+m+mf}{0.0}\PY{p}{,}\PY{l+m+mf}{0.0000000002}\PY{p}{)}
         \PY{n}{pyplot}\PY{o}{.}\PY{n}{xlabel}\PY{p}{(}\PY{l+s+s1}{\PYZsq{}}\PY{l+s+s1}{Time (days)}\PY{l+s+s1}{\PYZsq{}}\PY{p}{)}
         \PY{n}{pyplot}\PY{o}{.}\PY{n}{title}\PY{p}{(}\PY{l+s+s1}{\PYZsq{}}\PY{l+s+s1}{Planet 1 period}\PY{l+s+s1}{\PYZsq{}}\PY{p}{)}
         \PY{n}{actper} \PY{o}{=} \PY{l+m+mf}{19.237} \PY{c+c1}{\PYZsh{}\PYZsh{} the actual period}
         \PY{n}{pylab}\PY{o}{.}\PY{n}{axvline}\PY{p}{(}\PY{n}{actper}\PY{p}{,} \PY{n}{c}\PY{o}{=}\PY{l+s+s1}{\PYZsq{}}\PY{l+s+s1}{r}\PY{l+s+s1}{\PYZsq{}}\PY{p}{)} \PY{c+c1}{\PYZsh{}\PYZsh{} indicating the actual period}
         \PY{k}{for} \PY{n}{i} \PY{o+ow}{in} \PY{p}{[}\PY{l+m+mi}{1}\PY{p}{,} \PY{l+m+mi}{2}\PY{p}{]}\PY{p}{:}
             \PY{n}{pylab}\PY{o}{.}\PY{n}{axvline}\PY{p}{(}\PY{n}{actper}\PY{o}{*}\PY{n}{i}\PY{p}{,} \PY{n}{c}\PY{o}{=}\PY{l+s+s1}{\PYZsq{}}\PY{l+s+s1}{k}\PY{l+s+s1}{\PYZsq{}}\PY{p}{)}
             
             
\end{Verbatim}


    \begin{center}
    \adjustimage{max size={0.9\linewidth}{0.9\paperheight}}{output_35_0.png}
    \end{center}
    { \hspace*{\fill} \\}
    
    \begin{Verbatim}[commandchars=\\\{\}]
{\color{incolor}In [{\color{incolor}23}]:} \PY{n}{pylab}\PY{o}{.}\PY{n}{plot}\PY{p}{(}\PY{n}{freq}\PY{p}{,} \PY{n}{lomb\PYZus{}1}\PY{p}{)}
         \PY{n}{pylab}\PY{o}{.}\PY{n}{xlim}\PY{p}{(}\PY{l+m+mi}{15}\PY{p}{,}\PY{l+m+mi}{45}\PY{p}{)}
         \PY{n}{pylab}\PY{o}{.}\PY{n}{ylim}\PY{p}{(}\PY{l+m+mf}{0.0}\PY{p}{,}\PY{l+m+mf}{0.0000000004}\PY{p}{)}
         \PY{n}{pyplot}\PY{o}{.}\PY{n}{xlabel}\PY{p}{(}\PY{l+s+s1}{\PYZsq{}}\PY{l+s+s1}{Time (days)}\PY{l+s+s1}{\PYZsq{}}\PY{p}{)}
         \PY{n}{pyplot}\PY{o}{.}\PY{n}{title}\PY{p}{(}\PY{l+s+s1}{\PYZsq{}}\PY{l+s+s1}{Planet 2 period}\PY{l+s+s1}{\PYZsq{}}\PY{p}{)}
         \PY{c+c1}{\PYZsh{}actper = 19.237 \PYZsh{}\PYZsh{} the actual period}
         \PY{c+c1}{\PYZsh{}pylab.axvline(actper, c=\PYZsq{}y\PYZsq{}) \PYZsh{}\PYZsh{} indicating the actual period}
         \PY{c+c1}{\PYZsh{}for i in [1, 1.5, 2, 2.5, 3.0]:}
         \PY{c+c1}{\PYZsh{}    pylab.axvline(actper*i, c=\PYZsq{}k\PYZsq{})}
             
         \PY{n}{actper1} \PY{o}{=} \PY{l+m+mf}{38.98} \PY{c+c1}{\PYZsh{}\PYZsh{} the actual period}
         \PY{n}{pylab}\PY{o}{.}\PY{n}{axvline}\PY{p}{(}\PY{n}{actper1}\PY{p}{,} \PY{n}{c}\PY{o}{=}\PY{l+s+s1}{\PYZsq{}}\PY{l+s+s1}{g}\PY{l+s+s1}{\PYZsq{}}\PY{p}{)} \PY{c+c1}{\PYZsh{}\PYZsh{} indicating the actual period}
         \PY{k}{for} \PY{n}{i} \PY{o+ow}{in} \PY{p}{[}\PY{l+m+mf}{0.5}\PY{p}{,} \PY{l+m+mf}{1.00}\PY{p}{,} \PY{l+m+mf}{1.5}\PY{p}{,} \PY{l+m+mf}{2.0}\PY{p}{]}\PY{p}{:}
             \PY{n}{pylab}\PY{o}{.}\PY{n}{axvline}\PY{p}{(}\PY{n}{actper1}\PY{o}{*}\PY{n}{i}\PY{p}{,} \PY{n}{c}\PY{o}{=}\PY{l+s+s1}{\PYZsq{}}\PY{l+s+s1}{k}\PY{l+s+s1}{\PYZsq{}}\PY{p}{)}
\end{Verbatim}


    \begin{center}
    \adjustimage{max size={0.9\linewidth}{0.9\paperheight}}{output_36_0.png}
    \end{center}
    { \hspace*{\fill} \\}
    
    Knowing the periods of the planets from one lightcurve file, one can
start working with the entire list of lightcurves provided by the Kepler
space telescope. The program starts by entering the directory of where
the lightcurve data is saved. A full plot of all the lightcurves is then
made. We can clearly see that there is gap between each sampling taken
by Kepler. We also see a large range of flux values for each lightcurve
file. Each individual lightcurve needs to be normalised and then
combined into one shorter lightcurve. The purpose of folding the data,
i.e. superimposing each lightcurve on top of one another, is that it
makes a stronger trend in the transits of the exoplanets. First the data
was normalised using the same median filter as before and then saved to
a new `.csv' file so that it could be folded. A graphical output of the
normalised unfolded data can be seen below. As one can see the unfolded
lightcurve is 1600 days long and features many transits. The program
folds the normalised data into a shorter phase diagram, with phase from
1 to 3. We can see from this phase diagram a strong trend each time an
exoplanet transits the star.

    \begin{Verbatim}[commandchars=\\\{\}]
{\color{incolor}In [{\color{incolor}24}]:} \PY{c+c1}{\PYZsh{}\PYZsh{} Working over the entire data set}
         \PY{n}{glob}\PY{o}{.}\PY{n}{glob}\PY{p}{(}\PY{l+s+s1}{\PYZsq{}}\PY{l+s+s1}{Data/Object}\PY{l+s+si}{\PYZpc{}s}\PY{l+s+s1}{lc/kplr*.fits}\PY{l+s+s1}{\PYZsq{}}\PY{o}{\PYZpc{}}\PY{p}{(}\PY{n}{mykepler}\PY{p}{)}\PY{p}{)} \PY{c+c1}{\PYZsh{}Lists all the lightcurve files .fits in the directory}
\end{Verbatim}


\begin{Verbatim}[commandchars=\\\{\}]
{\color{outcolor}Out[{\color{outcolor}24}]:} ['Data/Object1lc/kplr1\_4.fits',
          'Data/Object1lc/kplr1\_14.fits',
          'Data/Object1lc/kplr1\_7.fits',
          'Data/Object1lc/kplr1\_11.fits',
          'Data/Object1lc/kplr1\_1.fits',
          'Data/Object1lc/kplr1\_3.fits',
          'Data/Object1lc/kplr1\_5.fits',
          'Data/Object1lc/kplr1\_2.fits',
          'Data/Object1lc/kplr1\_17.fits',
          'Data/Object1lc/kplr1\_16.fits',
          'Data/Object1lc/kplr1\_8.fits',
          'Data/Object1lc/kplr1\_6.fits',
          'Data/Object1lc/kplr1\_12.fits',
          'Data/Object1lc/kplr1\_13.fits',
          'Data/Object1lc/kplr1\_10.fits',
          'Data/Object1lc/kplr1\_9.fits',
          'Data/Object1lc/kplr1\_15.fits']
\end{Verbatim}
            
    \begin{Verbatim}[commandchars=\\\{\}]
{\color{incolor}In [{\color{incolor}25}]:} \PY{c+c1}{\PYZsh{}Lets plot all our lightcurves together onto one graph}
         
         \PY{k}{for} \PY{n}{lcfile} \PY{o+ow}{in} \PY{n}{glob}\PY{o}{.}\PY{n}{glob}\PY{p}{(}\PY{l+s+s1}{\PYZsq{}}\PY{l+s+s1}{Data/Object}\PY{l+s+si}{\PYZpc{}s}\PY{l+s+s1}{lc/kplr*.fits}\PY{l+s+s1}{\PYZsq{}} \PY{o}{\PYZpc{}}\PY{p}{(}\PY{n}{mykepler}\PY{p}{)}\PY{p}{)}\PY{p}{:}
             \PY{n}{tmp} \PY{o}{=} \PY{n}{fits}\PY{o}{.}\PY{n}{open}\PY{p}{(}\PY{n}{lcfile}\PY{p}{)}
             \PY{n}{tmptime} \PY{o}{=} \PY{p}{(}\PY{n}{tmp}\PY{p}{[}\PY{l+m+mi}{1}\PY{p}{]}\PY{o}{.}\PY{n}{data}\PY{p}{[}\PY{l+s+s1}{\PYZsq{}}\PY{l+s+s1}{TIME}\PY{l+s+s1}{\PYZsq{}}\PY{p}{]}\PY{p}{)}
             \PY{n}{tmpflux} \PY{o}{=} \PY{p}{(}\PY{n}{tmp}\PY{p}{[}\PY{l+m+mi}{1}\PY{p}{]}\PY{o}{.}\PY{n}{data}\PY{p}{[}\PY{l+s+s1}{\PYZsq{}}\PY{l+s+s1}{PDCSAP\PYZus{}FLUX}\PY{l+s+s1}{\PYZsq{}}\PY{p}{]}\PY{p}{)}
             \PY{n}{tmperror} \PY{o}{=} \PY{p}{(}\PY{n}{tmp}\PY{p}{[}\PY{l+m+mi}{1}\PY{p}{]}\PY{o}{.}\PY{n}{data}\PY{p}{[}\PY{l+s+s1}{\PYZsq{}}\PY{l+s+s1}{PDCSAP\PYZus{}FLUX\PYZus{}ERR}\PY{l+s+s1}{\PYZsq{}}\PY{p}{]}\PY{p}{)}
             \PY{n}{pylab}\PY{o}{.}\PY{n}{plot}\PY{p}{(}\PY{n}{tmptime}\PY{p}{,} \PY{n}{tmpflux}\PY{p}{,} \PY{n}{c}\PY{o}{=}\PY{l+s+s1}{\PYZsq{}}\PY{l+s+s1}{k}\PY{l+s+s1}{\PYZsq{}}\PY{p}{)}
\end{Verbatim}


    \begin{center}
    \adjustimage{max size={0.9\linewidth}{0.9\paperheight}}{output_39_0.png}
    \end{center}
    { \hspace*{\fill} \\}
    
    \begin{Verbatim}[commandchars=\\\{\}]
{\color{incolor}In [{\color{incolor}26}]:} \PY{c+c1}{\PYZsh{}Applying the same data reduction techniques used on one lightcurve}
         \PY{c+c1}{\PYZsh{}to the entire lightcurve dataset}
         
         \PY{n}{full\PYZus{}dataset\PYZus{}time}\PY{o}{=}\PY{p}{[}\PY{p}{]}    \PY{c+c1}{\PYZsh{} define new set arrays for our entire dataset (stop conflicting variables)}
         \PY{n}{full\PYZus{}dataset\PYZus{}flux}\PY{o}{=}\PY{p}{[}\PY{p}{]}    
         \PY{n}{full\PYZus{}dataset\PYZus{}flux\PYZus{}error}\PY{o}{=}\PY{p}{[}\PY{p}{]}
         
         \PY{c+c1}{\PYZsh{}Opening and reading all the fits files in the object 1 directory}
         
         \PY{k}{for} \PY{n}{lcfile} \PY{o+ow}{in} \PY{n}{glob}\PY{o}{.}\PY{n}{glob}\PY{p}{(}\PY{l+s+s1}{\PYZsq{}}\PY{l+s+s1}{Data/Object}\PY{l+s+si}{\PYZpc{}s}\PY{l+s+s1}{lc/kplr*.fits}\PY{l+s+s1}{\PYZsq{}} \PY{o}{\PYZpc{}}\PY{p}{(}\PY{n}{mykepler}\PY{p}{)}\PY{p}{)}\PY{p}{:} 
             \PY{n}{tmp} \PY{o}{=} \PY{n}{fits}\PY{o}{.}\PY{n}{open}\PY{p}{(}\PY{n}{lcfile}\PY{p}{)}
             \PY{n}{t} \PY{o}{=} \PY{p}{(}\PY{n}{tmp}\PY{p}{[}\PY{l+m+mi}{1}\PY{p}{]}\PY{o}{.}\PY{n}{data}\PY{p}{[}\PY{l+s+s1}{\PYZsq{}}\PY{l+s+s1}{TIME}\PY{l+s+s1}{\PYZsq{}}\PY{p}{]}\PY{p}{)}
             \PY{n}{full\PYZus{}flux} \PY{o}{=} \PY{p}{(}\PY{n}{tmp}\PY{p}{[}\PY{l+m+mi}{1}\PY{p}{]}\PY{o}{.}\PY{n}{data}\PY{p}{[}\PY{l+s+s1}{\PYZsq{}}\PY{l+s+s1}{PDCSAP\PYZus{}FLUX}\PY{l+s+s1}{\PYZsq{}}\PY{p}{]}\PY{p}{)}
             \PY{n}{error} \PY{o}{=} \PY{p}{(}\PY{n}{tmp}\PY{p}{[}\PY{l+m+mi}{1}\PY{p}{]}\PY{o}{.}\PY{n}{data}\PY{p}{[}\PY{l+s+s1}{\PYZsq{}}\PY{l+s+s1}{PDCSAP\PYZus{}FLUX\PYZus{}ERR}\PY{l+s+s1}{\PYZsq{}}\PY{p}{]}\PY{p}{)}
             
             
         \PY{c+c1}{\PYZsh{}Plotting a normalised lightcurve for all the fits files}
         
             \PY{n}{medfilter} \PY{o}{=} \PY{n}{medfilt}\PY{p}{(}\PY{n}{full\PYZus{}flux}\PY{p}{,} \PY{n}{kernel\PYZus{}size} \PY{o}{=} \PY{l+m+mi}{21}\PY{p}{)} \PY{c+c1}{\PYZsh{}Calling the same median filter function as before}
             \PY{n}{normalised\PYZus{}flux} \PY{o}{=} \PY{n}{full\PYZus{}flux}\PY{o}{/}\PY{n}{medfilter}            \PY{c+c1}{\PYZsh{}Normalise flux error by dividing by the median filter as before}
             \PY{n}{normalised\PYZus{}error} \PY{o}{=} \PY{n}{error}\PY{o}{/}\PY{n}{medfilter}               \PY{c+c1}{\PYZsh{}Normalise flux error by dividing by the median filter as before}
             
             \PY{n}{pylab}\PY{o}{.}\PY{n}{plot}\PY{p}{(}\PY{n}{t}\PY{p}{,} \PY{n}{normalised\PYZus{}flux}\PY{p}{,} \PY{n}{ls}\PY{o}{=}\PY{l+s+s1}{\PYZsq{}}\PY{l+s+s1}{None}\PY{l+s+s1}{\PYZsq{}}\PY{p}{,} \PY{n}{marker}\PY{o}{=}\PY{l+s+s1}{\PYZsq{}}\PY{l+s+s1}{.}\PY{l+s+s1}{\PYZsq{}}\PY{p}{,} \PY{n}{c}\PY{o}{=}\PY{l+s+s1}{\PYZsq{}}\PY{l+s+s1}{grey}\PY{l+s+s1}{\PYZsq{}}\PY{p}{,} \PY{n}{label}\PY{o}{=}\PY{l+s+s1}{\PYZsq{}}\PY{l+s+s1}{Data}\PY{l+s+s1}{\PYZsq{}}\PY{p}{)} \PY{c+c1}{\PYZsh{}plots the normalised fits data}
             
         \PY{c+c1}{\PYZsh{}Adds the noramlised fits data onto the end of each array}
             
             \PY{n}{full\PYZus{}dataset\PYZus{}time}\PY{o}{.}\PY{n}{extend}\PY{p}{(}\PY{n}{t}\PY{p}{)}
             \PY{n}{full\PYZus{}dataset\PYZus{}flux}\PY{o}{.}\PY{n}{extend}\PY{p}{(}\PY{n}{normalised\PYZus{}flux}\PY{p}{)}
             \PY{n}{full\PYZus{}dataset\PYZus{}flux\PYZus{}error}\PY{o}{.}\PY{n}{extend}\PY{p}{(}\PY{n}{normalised\PYZus{}error}\PY{p}{)}
\end{Verbatim}


    \begin{center}
    \adjustimage{max size={0.9\linewidth}{0.9\paperheight}}{output_40_0.png}
    \end{center}
    { \hspace*{\fill} \\}
    
    \begin{Verbatim}[commandchars=\\\{\}]
{\color{incolor}In [{\color{incolor}27}]:} \PY{c+c1}{\PYZsh{}In order to work with the normalised dataset, we need to save it.}
         \PY{c+c1}{\PYZsh{}Saves the arrays to a new file}
         \PY{c+c1}{\PYZsh{}We\PYZsq{}ll first need to write our data to file.}
         
         \PY{n}{np}\PY{o}{.}\PY{n}{savetxt}\PY{p}{(}\PY{l+s+s1}{\PYZsq{}}\PY{l+s+s1}{normalised\PYZus{}dataset.csv}\PY{l+s+s1}{\PYZsq{}}\PY{p}{,} \PY{n}{np}\PY{o}{.}\PY{n}{array}\PY{p}{(}\PY{p}{[}\PY{n}{t}\PY{p}{,} \PY{n}{normalised\PYZus{}flux}\PY{p}{,} \PY{n}{normalised\PYZus{}error}\PY{p}{]}\PY{p}{)}\PY{o}{.}\PY{n}{transpose}\PY{p}{(}\PY{p}{)}\PY{p}{,} \PY{n}{header}\PY{o}{=}\PY{l+s+s1}{\PYZsq{}}\PY{l+s+s1}{\PYZsh{}JD, mag, error}\PY{l+s+s1}{\PYZsq{}}\PY{p}{,} \PY{n}{delimiter}\PY{o}{=}\PY{l+s+s1}{\PYZsq{}}\PY{l+s+s1}{,}\PY{l+s+s1}{\PYZsq{}}\PY{p}{)}
\end{Verbatim}


    \begin{Verbatim}[commandchars=\\\{\}]
{\color{incolor}In [{\color{incolor}28}]:} \PY{c+c1}{\PYZsh{}We want to combine all the transits together to get a stronger data trend}
         \PY{c+c1}{\PYZsh{}To do this we apply a fold to the dataset so that they overlap}
         
         \PY{k}{def} \PY{n+nf}{fold\PYZus{}lightcurve}\PY{p}{(}\PY{n}{filename}\PY{p}{,} \PY{n}{period}\PY{p}{,} \PY{o}{*}\PY{n}{args}\PY{p}{,} \PY{o}{*}\PY{o}{*}\PY{n}{kwargs}\PY{p}{)}\PY{p}{:}
             \PY{n}{obj\PYZus{}name} \PY{o}{=} \PY{n}{kwargs}\PY{o}{.}\PY{n}{get}\PY{p}{(}\PY{l+s+s1}{\PYZsq{}}\PY{l+s+s1}{obj\PYZus{}name}\PY{l+s+s1}{\PYZsq{}}\PY{p}{,} \PY{k+kc}{None}\PY{p}{)}
             \PY{n}{outdata} \PY{o}{=} \PY{n}{kwargs}\PY{o}{.}\PY{n}{get}\PY{p}{(}\PY{l+s+s1}{\PYZsq{}}\PY{l+s+s1}{output\PYZus{}file}\PY{l+s+s1}{\PYZsq{}}\PY{p}{,} \PY{l+s+s1}{\PYZsq{}}\PY{l+s+s1}{folded\PYZus{}lc\PYZus{}data.csv}\PY{l+s+s1}{\PYZsq{}}\PY{p}{)}
             \PY{n}{plotname} \PY{o}{=} \PY{n}{kwargs}\PY{o}{.}\PY{n}{get}\PY{p}{(}\PY{l+s+s1}{\PYZsq{}}\PY{l+s+s1}{plot\PYZus{}file}\PY{l+s+s1}{\PYZsq{}}\PY{p}{,} \PY{l+s+s1}{\PYZsq{}}\PY{l+s+s1}{folded\PYZus{}lc.pdf}\PY{l+s+s1}{\PYZsq{}}\PY{p}{)}
             
             \PY{c+c1}{\PYZsh{}\PYZsh{} Read in the data. Should be comma separated, header row (if present) should have \PYZsh{} at the start.}
             \PY{n}{data} \PY{o}{=} \PY{n}{pd}\PY{o}{.}\PY{n}{read\PYZus{}csv}\PY{p}{(}\PY{n}{filename}\PY{p}{,} \PY{n}{usecols}\PY{o}{=}\PY{p}{[}\PY{l+m+mi}{0}\PY{p}{,}\PY{l+m+mi}{1}\PY{p}{,}\PY{l+m+mi}{2}\PY{p}{]}\PY{p}{,} \PY{n}{names}\PY{o}{=}\PY{p}{(}\PY{l+s+s1}{\PYZsq{}}\PY{l+s+s1}{JD}\PY{l+s+s1}{\PYZsq{}}\PY{p}{,} \PY{l+s+s1}{\PYZsq{}}\PY{l+s+s1}{mag}\PY{l+s+s1}{\PYZsq{}}\PY{p}{,} \PY{l+s+s1}{\PYZsq{}}\PY{l+s+s1}{error}\PY{l+s+s1}{\PYZsq{}}\PY{p}{)}\PY{p}{,} \PY{n}{comment}\PY{o}{=}\PY{l+s+s1}{\PYZsq{}}\PY{l+s+s1}{\PYZsh{}}\PY{l+s+s1}{\PYZsq{}}\PY{p}{)}
             \PY{k}{if} \PY{n+nb}{len}\PY{p}{(}\PY{n}{data}\PY{o}{.}\PY{n}{columns}\PY{p}{)} \PY{o}{\PYZlt{}} \PY{l+m+mi}{3}\PY{p}{:}
                 \PY{n+nb}{print}\PY{p}{(}\PY{l+s+s2}{\PYZdq{}}\PY{l+s+s2}{File format should be }\PY{l+s+se}{\PYZbs{}n}\PY{l+s+se}{\PYZbs{}}
         \PY{l+s+s2}{              (M)JD, magnitude, uncertainty}\PY{l+s+se}{\PYZbs{}n}\PY{l+s+s2}{\PYZdq{}}\PY{p}{)}
                 \PY{n}{exit}\PY{p}{(}\PY{l+m+mi}{1}\PY{p}{)}
             \PY{c+c1}{\PYZsh{}\PYZsh{} Folding the lightcurve:}
             \PY{c+c1}{\PYZsh{}\PYZsh{} Phase = JD/period \PYZhy{} floor(JD/period)}
             \PY{c+c1}{\PYZsh{}\PYZsh{} The floor function is there to make sure that the phase is between 0 and 1.}
             
             \PY{n}{data}\PY{p}{[}\PY{l+s+s1}{\PYZsq{}}\PY{l+s+s1}{Phase}\PY{l+s+s1}{\PYZsq{}}\PY{p}{]} \PY{o}{=} \PY{n}{data}\PY{o}{.}\PY{n}{apply}\PY{p}{(}\PY{k}{lambda} \PY{n}{x}\PY{p}{:} \PY{p}{(}\PY{p}{(}\PY{n}{x}\PY{o}{.}\PY{n}{JD}\PY{o}{/} \PY{n}{period}\PY{p}{)} \PY{o}{\PYZhy{}} \PY{n}{np}\PY{o}{.}\PY{n}{floor}\PY{p}{(}\PY{n}{x}\PY{o}{.}\PY{n}{JD} \PY{o}{/} \PY{n}{period}\PY{p}{)}\PY{p}{)}\PY{p}{,} \PY{n}{axis}\PY{o}{=}\PY{l+m+mi}{1}\PY{p}{)}
             
               
             \PY{c+c1}{\PYZsh{}\PYZsh{} Now make the plot}
             
             \PY{n}{pyplot}\PY{o}{.}\PY{n}{clf}\PY{p}{(}\PY{p}{)}
             \PY{n}{pyplot}\PY{o}{.}\PY{n}{figure}\PY{p}{(}\PY{n}{figsize}\PY{o}{=}\PY{p}{(}\PY{l+m+mi}{10}\PY{p}{,}\PY{l+m+mi}{5}\PY{p}{)}\PY{p}{)}
             
             \PY{c+c1}{\PYZsh{}\PYZsh{} concatenating the arrays to make phase \PYZhy{}\PYZgt{} 0 \PYZhy{} 3}
             \PY{c+c1}{\PYZsh{}\PYZsh{} This makes it easier to see if periodic lightcurves join up as expected}
             
             \PY{n}{phase\PYZus{}long} \PY{o}{=} \PY{n}{np}\PY{o}{.}\PY{n}{concatenate}\PY{p}{(}\PY{p}{(}\PY{n}{data}\PY{o}{.}\PY{n}{Phase}\PY{p}{,} \PY{n}{data}\PY{o}{.}\PY{n}{Phase} \PY{o}{+} \PY{l+m+mf}{1.0}\PY{p}{,} \PY{n}{data}\PY{o}{.}\PY{n}{Phase} \PY{o}{+} \PY{l+m+mf}{2.0}\PY{p}{)}\PY{p}{)}
             \PY{n}{mag\PYZus{}long} \PY{o}{=} \PY{n}{np}\PY{o}{.}\PY{n}{concatenate}\PY{p}{(}\PY{p}{(}\PY{n}{data}\PY{o}{.}\PY{n}{mag}\PY{p}{,} \PY{n}{data}\PY{o}{.}\PY{n}{mag}\PY{p}{,} \PY{n}{data}\PY{o}{.}\PY{n}{mag}\PY{p}{)}\PY{p}{)}
             \PY{n}{err\PYZus{}long} \PY{o}{=} \PY{n}{np}\PY{o}{.}\PY{n}{concatenate}\PY{p}{(}\PY{p}{(}\PY{n}{data}\PY{o}{.}\PY{n}{error}\PY{p}{,} \PY{n}{data}\PY{o}{.}\PY{n}{error}\PY{p}{,} \PY{n}{data}\PY{o}{.}\PY{n}{error}\PY{p}{)}\PY{p}{)}
             
             \PY{n}{pyplot}\PY{o}{.}\PY{n}{errorbar}\PY{p}{(}\PY{n}{phase\PYZus{}long}\PY{p}{,} \PY{n}{mag\PYZus{}long}\PY{p}{,} \PY{n}{yerr}\PY{o}{=}\PY{n}{err\PYZus{}long}\PY{p}{,} \PY{n}{marker}\PY{o}{=}\PY{l+s+s1}{\PYZsq{}}\PY{l+s+s1}{o}\PY{l+s+s1}{\PYZsq{}}\PY{p}{,} \PY{n}{ls}\PY{o}{=}\PY{l+s+s1}{\PYZsq{}}\PY{l+s+s1}{None}\PY{l+s+s1}{\PYZsq{}}\PY{p}{,} \PY{n}{zorder}\PY{o}{=}\PY{l+m+mi}{4}\PY{p}{,} \PY{n}{label}\PY{o}{=}\PY{l+s+s1}{\PYZsq{}}\PY{l+s+s1}{\PYZus{}nolegend\PYZus{}}\PY{l+s+s1}{\PYZsq{}}\PY{p}{,} \PY{n}{mec}\PY{o}{=}\PY{l+s+s1}{\PYZsq{}}\PY{l+s+s1}{Grey}\PY{l+s+s1}{\PYZsq{}}\PY{p}{)}
             \PY{c+c1}{\PYZsh{}\PYZsh{} Inverting the y axis because magnitudes}
             \PY{c+c1}{\PYZsh{}pyplot.gca().invert\PYZus{}yaxis()}
             
             \PY{n}{pyplot}\PY{o}{.}\PY{n}{xlabel}\PY{p}{(}\PY{l+s+s1}{\PYZsq{}}\PY{l+s+s1}{Phase (\PYZdl{}}\PY{l+s+s1}{\PYZbs{}}\PY{l+s+s1}{phi\PYZdl{})}\PY{l+s+s1}{\PYZsq{}}\PY{p}{)}
             \PY{n}{pyplot}\PY{o}{.}\PY{n}{ylabel}\PY{p}{(}\PY{l+s+s1}{\PYZsq{}}\PY{l+s+s1}{Magnitude}\PY{l+s+s1}{\PYZsq{}}\PY{p}{)}
             
             \PY{c+c1}{\PYZsh{}\PYZsh{} making the plot title to include object and period. }
             \PY{c+c1}{\PYZsh{}\PYZsh{} If no object name given in kwargs then title is just period}
             \PY{k}{if} \PY{n}{obj\PYZus{}name} \PY{o}{!=} \PY{k+kc}{None}\PY{p}{:}
                 \PY{n}{namestring} \PY{o}{=} \PY{n+nb}{str}\PY{p}{(}\PY{n}{obj\PYZus{}name}\PY{p}{)} \PY{o}{+} \PY{l+s+s1}{\PYZsq{}}\PY{l+s+s1}{, }\PY{l+s+s1}{\PYZsq{}}
             \PY{k}{else}\PY{p}{:}
                 \PY{n}{namestring} \PY{o}{=} \PY{l+s+s1}{\PYZsq{}}\PY{l+s+s1}{\PYZsq{}}
             \PY{n}{titletext} \PY{o}{=} \PY{n}{namestring} \PY{o}{+} \PY{l+s+s1}{\PYZsq{}}\PY{l+s+s1}{P = }\PY{l+s+s1}{\PYZsq{}} \PY{o}{+} \PY{n+nb}{str}\PY{p}{(}\PY{n}{np}\PY{o}{.}\PY{n}{around}\PY{p}{(}\PY{n}{period}\PY{p}{,} \PY{n}{decimals}\PY{o}{=}\PY{l+m+mi}{4}\PY{p}{)}\PY{p}{)} \PY{o}{+} \PY{l+s+s1}{\PYZsq{}}\PY{l+s+s1}{d}\PY{l+s+s1}{\PYZsq{}}
             \PY{n}{pyplot}\PY{o}{.}\PY{n}{suptitle}\PY{p}{(}\PY{n}{titletext}\PY{p}{)}
             
             \PY{n}{pyplot}\PY{o}{.}\PY{n}{savefig}\PY{p}{(}\PY{n}{plotname}\PY{p}{)}
             
             \PY{c+c1}{\PYZsh{}\PYZsh{} Printing the phased data to a csv file.}
             \PY{c+c1}{\PYZsh{}\PYZsh{} If filename not given in fold\PYZus{}lightcurve arguments default filename is folded\PYZus{}lc\PYZus{}data.csv}
             
             \PY{n}{data}\PY{o}{.}\PY{n}{to\PYZus{}csv}\PY{p}{(}\PY{n}{outdata}\PY{p}{,} \PY{n}{header}\PY{o}{=}\PY{k+kc}{True}\PY{p}{,} \PY{n}{index}\PY{o}{=}\PY{k+kc}{False}\PY{p}{,} \PY{n}{sep}\PY{o}{=}\PY{l+s+s1}{\PYZsq{}}\PY{l+s+s1}{,}\PY{l+s+s1}{\PYZsq{}}\PY{p}{)}
         
             
             \PY{k}{return} \PY{n}{data}
\end{Verbatim}


    \begin{Verbatim}[commandchars=\\\{\}]
{\color{incolor}In [{\color{incolor}29}]:} \PY{c+c1}{\PYZsh{}Lets try folding on the period we found from the single lightcurve}
         \PY{n}{fold\PYZus{}lightcurve}\PY{p}{(}\PY{l+s+s1}{\PYZsq{}}\PY{l+s+s1}{normalised\PYZus{}dataset.csv}\PY{l+s+s1}{\PYZsq{}}\PY{p}{,} \PY{l+m+mf}{19.237}\PY{p}{)}
         \PY{c+c1}{\PYZsh{}Phase goes from 0\PYZhy{}1, we can clear see a nicely folded transit for this phase}
\end{Verbatim}


\begin{Verbatim}[commandchars=\\\{\}]
{\color{outcolor}Out[{\color{outcolor}29}]:}                JD       mag     error     Phase
         0     1001.208319       NaN       NaN  0.045970
         1     1001.228752  1.000680  0.000174  0.047032
         2     1001.249184  1.000214  0.000174  0.048094
         3     1001.269617  0.999926  0.000174  0.049156
         4     1001.290050  0.999382  0.000174  0.050218
         5     1001.310483  0.999693  0.000174  0.051280
         6     1001.330916  0.999686  0.000174  0.052343
         7     1001.351348  0.999767  0.000174  0.053405
         8     1001.371781  0.999928  0.000174  0.054467
         9     1001.392214  1.000000  0.000174  0.055529
         10    1001.412647  0.999709  0.000174  0.056591
         11    1001.433080  1.000072  0.000174  0.057653
         12    1001.453512  0.999861  0.000174  0.058716
         13    1001.473945  1.000000  0.000174  0.059778
         14    1001.494378  1.000001  0.000174  0.060840
         15    1001.514811  1.000195  0.000174  0.061902
         16    1001.535244  1.000133  0.000173  0.062964
         17    1001.555676  0.999920  0.000174  0.064026
         18    1001.576109  1.000000  0.000174  0.065089
         19    1001.596542  0.999978  0.000173  0.066151
         20    1001.616975  1.000000  0.000174  0.067213
         21    1001.637407  0.999935  0.000174  0.068275
         22    1001.657840  0.999985  0.000174  0.069337
         23    1001.678273  0.999929  0.000173  0.070399
         24    1001.698706  0.999931  0.000173  0.071462
         25    1001.719139  0.999994  0.000173  0.072524
         26    1001.739571  1.000000  0.000173  0.073586
         27    1001.760004  1.000347  0.000173  0.074648
         28    1001.780437  1.000000  0.000173  0.075710
         29    1001.800870  1.000105  0.000173  0.076772
         {\ldots}           {\ldots}       {\ldots}       {\ldots}       {\ldots}
         4724  1097.732445  1.000357  0.000177  0.063599
         4725  1097.752878  0.999982  0.000177  0.064661
         4726  1097.773311  1.000204  0.000177  0.065723
         4727  1097.793744  1.000056  0.000177  0.066785
         4728  1097.814177  1.000064  0.000177  0.067847
         4729  1097.834610  0.999976  0.000177  0.068909
         4730  1097.855043  1.000000  0.000177  0.069972
         4731  1097.875476  0.999988  0.000177  0.071034
         4732  1097.895910  1.000036  0.000177  0.072096
         4733  1097.916343  0.999954  0.000177  0.073158
         4734  1097.936776  1.000056  0.000177  0.074220
         4735  1097.957209  0.999920  0.000177  0.075282
         4736  1097.977642  0.999957  0.000177  0.076345
         4737  1097.998075  0.999666  0.000177  0.077407
         4738  1098.018508  0.999408  0.000177  0.078469
         4739  1098.038941  0.999618  0.000177  0.079531
         4740  1098.059374  0.999883  0.000177  0.080593
         4741  1098.079807  0.999812  0.000177  0.081656
         4742  1098.100240  1.000097  0.000177  0.082718
         4743  1098.120673  1.000000  0.000177  0.083780
         4744  1098.141107  1.000398  0.000177  0.084842
         4745  1098.161540  0.999827  0.000177  0.085904
         4746  1098.181973  1.000802  0.000177  0.086966
         4747  1098.202406  1.000268  0.000177  0.088029
         4748  1098.222839  1.000445  0.000177  0.089091
         4749  1098.243272  1.000107  0.000177  0.090153
         4750  1098.263705  1.000000  0.000177  0.091215
         4751  1098.284138  0.999888  0.000177  0.092277
         4752  1098.304571  0.999980  0.000177  0.093339
         4753  1098.325004  1.000132  0.000177  0.094402
         
         [4754 rows x 4 columns]
\end{Verbatim}
            
    
    \begin{verbatim}
<Figure size 432x288 with 0 Axes>
    \end{verbatim}

    
    \begin{center}
    \adjustimage{max size={0.9\linewidth}{0.9\paperheight}}{output_43_2.png}
    \end{center}
    { \hspace*{\fill} \\}
    
    \begin{Verbatim}[commandchars=\\\{\}]
{\color{incolor}In [{\color{incolor}30}]:} \PY{n}{fold\PYZus{}lightcurve}\PY{p}{(}\PY{l+s+s1}{\PYZsq{}}\PY{l+s+s1}{normalised\PYZus{}dataset.csv}\PY{l+s+s1}{\PYZsq{}}\PY{p}{,} \PY{l+m+mf}{19.237}\PY{p}{)}
         \PY{n}{pylab}\PY{o}{.}\PY{n}{ylim}\PY{p}{(}\PY{l+m+mf}{0.992}\PY{p}{,}\PY{l+m+mf}{1.002}\PY{p}{)}
         \PY{n}{pylab}\PY{o}{.}\PY{n}{xlim}\PY{p}{(}\PY{l+m+mi}{0}\PY{p}{,}\PY{l+m+mi}{3}\PY{p}{)}
\end{Verbatim}


\begin{Verbatim}[commandchars=\\\{\}]
{\color{outcolor}Out[{\color{outcolor}30}]:} (0, 3)
\end{Verbatim}
            
    
    \begin{verbatim}
<Figure size 432x288 with 0 Axes>
    \end{verbatim}

    
    \begin{center}
    \adjustimage{max size={0.9\linewidth}{0.9\paperheight}}{output_44_2.png}
    \end{center}
    { \hspace*{\fill} \\}
    
    \begin{Verbatim}[commandchars=\\\{\}]
{\color{incolor}In [{\color{incolor}31}]:} \PY{n}{fold\PYZus{}lightcurve}\PY{p}{(}\PY{l+s+s1}{\PYZsq{}}\PY{l+s+s1}{normalised\PYZus{}dataset.csv}\PY{l+s+s1}{\PYZsq{}}\PY{p}{,} \PY{l+m+mf}{19.237}\PY{p}{)}
         \PY{n}{pylab}\PY{o}{.}\PY{n}{ylim}\PY{p}{(}\PY{l+m+mf}{0.992}\PY{p}{,}\PY{l+m+mf}{1.002}\PY{p}{)}
         \PY{n}{pylab}\PY{o}{.}\PY{n}{xlim}\PY{p}{(}\PY{l+m+mf}{0.53}\PY{p}{,}\PY{l+m+mf}{0.61}\PY{p}{)}
\end{Verbatim}


\begin{Verbatim}[commandchars=\\\{\}]
{\color{outcolor}Out[{\color{outcolor}31}]:} (0.53, 0.61)
\end{Verbatim}
            
    
    \begin{verbatim}
<Figure size 432x288 with 0 Axes>
    \end{verbatim}

    
    \begin{center}
    \adjustimage{max size={0.9\linewidth}{0.9\paperheight}}{output_45_2.png}
    \end{center}
    { \hspace*{\fill} \\}
    
    In order to determine the depth of the transit caused by each exoplanet,
a parametric model needs to be fit to the lightcurve. Below one can see
how a basic function was fit to the folded normalised data. The program
uses the `curve\_fit' function with a function of, n, parameters defined
by the user. The function also sets limits on the where to begin the fit
on the data. One limitation of the function is that it doesn't account
for limb darkening and therefore the parametric function won't fit to
the bottom of the transit. However, for the case of this paper, a
rudimentary calculation of the depth will suffice. Using the transit
depth one can determine the planets radius.

    \begin{Verbatim}[commandchars=\\\{\}]
{\color{incolor}In [{\color{incolor}32}]:} \PY{c+c1}{\PYZsh{}Modelling the lightcurve of Planet 1}
         \PY{n}{transits} \PY{o}{=} \PY{n}{pd}\PY{o}{.}\PY{n}{read\PYZus{}csv}\PY{p}{(}\PY{l+s+s1}{\PYZsq{}}\PY{l+s+s1}{folded\PYZus{}lc\PYZus{}data.csv}\PY{l+s+s1}{\PYZsq{}}\PY{p}{,} \PY{n}{error\PYZus{}bad\PYZus{}lines}\PY{o}{=}\PY{k+kc}{False}\PY{p}{)} \PY{c+c1}{\PYZsh{}reads the normalised dataset}
         \PY{n}{nanmask} \PY{o}{=} \PY{o}{\PYZti{}}\PY{n}{np}\PY{o}{.}\PY{n}{isnan}\PY{p}{(}\PY{n}{Y}\PY{p}{)}                                               \PY{c+c1}{\PYZsh{}flux nan mask}
         \PY{n}{Y} \PY{o}{=} \PY{n}{transits}\PY{p}{[}\PY{l+s+s1}{\PYZsq{}}\PY{l+s+s1}{mag}\PY{l+s+s1}{\PYZsq{}}\PY{p}{]}\PY{o}{.}\PY{n}{values}                                          \PY{c+c1}{\PYZsh{}New variable definitions for the plotting function}
         \PY{n}{X} \PY{o}{=} \PY{n}{transits}\PY{p}{[}\PY{l+s+s1}{\PYZsq{}}\PY{l+s+s1}{Phase}\PY{l+s+s1}{\PYZsq{}}\PY{p}{]}\PY{o}{.}\PY{n}{values}
         \PY{n}{yerror} \PY{o}{=} \PY{n}{transits}\PY{p}{[}\PY{l+s+s1}{\PYZsq{}}\PY{l+s+s1}{error}\PY{l+s+s1}{\PYZsq{}}\PY{p}{]}\PY{o}{.}\PY{n}{values}                   
         
         \PY{c+c1}{\PYZsh{}Defining the functions operating coniditions(baseline, lower limit, upper limit)}
         
         \PY{k}{def} \PY{n+nf}{func}\PY{p}{(}\PY{n}{X}\PY{p}{,} \PY{n}{a}\PY{p}{,} \PY{n}{b}\PY{p}{,} \PY{n}{fmax}\PY{p}{,} \PY{n}{fmin}\PY{p}{)}\PY{p}{:}
             
             \PY{n}{y}\PY{o}{=} \PY{p}{[}\PY{p}{]} 
             \PY{k}{for} \PY{n}{x} \PY{o+ow}{in} \PY{n}{X}\PY{p}{:} 
                 \PY{k}{if} \PY{n}{x}\PY{o}{\PYZlt{}}\PY{n}{a}\PY{p}{:}
                     \PY{n}{y}\PY{o}{.}\PY{n}{append}\PY{p}{(}\PY{n}{fmax}\PY{p}{)}
                 \PY{k}{if} \PY{n}{x}\PY{o}{\PYZgt{}}\PY{n}{a} \PY{o+ow}{and} \PY{n}{x}\PY{o}{\PYZlt{}}\PY{n}{b}\PY{p}{:}
                     \PY{n}{y}\PY{o}{.}\PY{n}{append}\PY{p}{(}\PY{n}{fmin}\PY{p}{)}
                 \PY{k}{if} \PY{n}{x}\PY{o}{\PYZgt{}}\PY{n}{b}\PY{p}{:}
                     \PY{n}{y}\PY{o}{.}\PY{n}{append}\PY{p}{(}\PY{n}{fmax}\PY{p}{)}
             \PY{k}{return} \PY{n}{y}
         
         \PY{c+c1}{\PYZsh{}Findind the transit depth function}
         
         \PY{n}{popt1}\PY{p}{,} \PY{n}{pcov1} \PY{o}{=} \PY{n}{curve\PYZus{}fit}\PY{p}{(}\PY{n}{func}\PY{p}{,} \PY{n}{X}\PY{p}{[}\PY{n}{nanmask}\PY{p}{]}\PY{p}{,} \PY{n}{Y}\PY{p}{[}\PY{n}{nanmask}\PY{p}{]}\PY{p}{,} \PY{n}{p0} \PY{o}{=} \PY{p}{[}\PY{l+m+mf}{0.543}\PY{p}{,} \PY{l+m+mf}{0.552}\PY{p}{,} \PY{l+m+mf}{1.00}\PY{p}{,} \PY{l+m+mf}{0.992}\PY{p}{]}\PY{p}{)} \PY{c+c1}{\PYZsh{} using curve\PYZus{}fit function}
         \PY{n}{delta\PYZus{}flux\PYZus{}1}\PY{o}{=} \PY{n}{popt1}\PY{p}{[}\PY{l+m+mi}{2}\PY{p}{]}\PY{o}{\PYZhy{}}\PY{n}{popt1}\PY{p}{[}\PY{l+m+mi}{3}\PY{p}{]}                                                            \PY{c+c1}{\PYZsh{} calculating the change in flux}
\end{Verbatim}


    \begin{Verbatim}[commandchars=\\\{\}]

        ---------------------------------------------------------------------------

        NameError                                 Traceback (most recent call last)

        <ipython-input-32-cb48c07b8193> in <module>()
          1 \#Modelling the lightcurve of Planet 1
          2 transits = pd.read\_csv('folded\_lc\_data.csv', error\_bad\_lines=False) \#reads the normalised dataset
    ----> 3 nanmask = \textasciitilde{}np.isnan(Y)                                               \#flux nan mask
          4 Y = transits['mag'].values                                          \#New variable definitions for the plotting function
          5 X = transits['Phase'].values


        NameError: name 'Y' is not defined

    \end{Verbatim}

    \begin{Verbatim}[commandchars=\\\{\}]
{\color{incolor}In [{\color{incolor} }]:} \PY{n}{pylab}\PY{o}{.}\PY{n}{scatter}\PY{p}{(}\PY{n}{X}\PY{p}{,} \PY{n}{Y}\PY{p}{)}
        \PY{n}{xdata} \PY{o}{=} \PY{n}{np}\PY{o}{.}\PY{n}{linspace}\PY{p}{(}\PY{n+nb}{min}\PY{p}{(}\PY{n}{X}\PY{p}{)}\PY{p}{,} \PY{n+nb}{max}\PY{p}{(}\PY{n}{X}\PY{p}{)}\PY{p}{,} \PY{l+m+mi}{1000}\PY{p}{)}
        \PY{n}{pylab}\PY{o}{.}\PY{n}{plot}\PY{p}{(}\PY{n}{xdata}\PY{p}{,} \PY{n}{func}\PY{p}{(}\PY{n}{xdata}\PY{p}{,} \PY{o}{*}\PY{n}{popt1}\PY{p}{)}\PY{p}{,} \PY{l+s+s1}{\PYZsq{}}\PY{l+s+s1}{g}\PY{l+s+s1}{\PYZsq{}}\PY{p}{,} \PY{n}{label}\PY{o}{=}\PY{l+s+s1}{\PYZsq{}}\PY{l+s+s1}{fit: a=}\PY{l+s+si}{\PYZpc{}5.3f}\PY{l+s+s1}{, b=}\PY{l+s+si}{\PYZpc{}5.3f}\PY{l+s+s1}{, c=}\PY{l+s+si}{\PYZpc{}5.3f}\PY{l+s+s1}{, d=}\PY{l+s+si}{\PYZpc{}5.3f}\PY{l+s+s1}{\PYZsq{}} \PY{o}{\PYZpc{}} \PY{n+nb}{tuple}\PY{p}{(}\PY{n}{popt1}\PY{p}{)}\PY{p}{)}
        \PY{n}{pylab}\PY{o}{.}\PY{n}{ylim}\PY{p}{(}\PY{l+m+mf}{0.992}\PY{p}{,} \PY{l+m+mf}{1.002}\PY{p}{)}
        \PY{n}{pylab}\PY{o}{.}\PY{n}{xlim}\PY{p}{(}\PY{l+m+mf}{0.525}\PY{p}{,}\PY{l+m+mf}{0.575}\PY{p}{)}
        \PY{n}{pylab}\PY{o}{.}\PY{n}{title}\PY{p}{(}\PY{l+s+s1}{\PYZsq{}}\PY{l+s+s1}{Planet 1 transit depth}\PY{l+s+s1}{\PYZsq{}}\PY{p}{)}
        \PY{n+nb}{print}\PY{p}{(}\PY{l+s+s1}{\PYZsq{}}\PY{l+s+s1}{Planet 1 transit depth is: }\PY{l+s+s1}{\PYZsq{}}\PY{p}{,} \PY{l+s+s2}{\PYZdq{}}\PY{l+s+si}{\PYZpc{}.6f}\PY{l+s+s2}{\PYZdq{}} \PY{o}{\PYZpc{}} \PY{n}{delta\PYZus{}flux\PYZus{}1}\PY{p}{)}    \PY{c+c1}{\PYZsh{} printing the value of the change in flux}
\end{Verbatim}


    \begin{Verbatim}[commandchars=\\\{\}]
{\color{incolor}In [{\color{incolor} }]:} \PY{c+c1}{\PYZsh{}Modelling the lightcurve of Planet 2}
        
        \PY{n}{popt}\PY{p}{,} \PY{n}{pcov2} \PY{o}{=} \PY{n}{curve\PYZus{}fit}\PY{p}{(}\PY{n}{func}\PY{p}{,} \PY{n}{X}\PY{p}{[}\PY{n}{nanmask}\PY{p}{]}\PY{p}{,} \PY{n}{Y}\PY{p}{[}\PY{n}{nanmask}\PY{p}{]}\PY{p}{,} \PY{n}{p0} \PY{o}{=} \PY{p}{[}\PY{l+m+mf}{0.59}\PY{p}{,} \PY{l+m+mf}{0.60}\PY{p}{,} \PY{l+m+mf}{1.00}\PY{p}{,} \PY{l+m+mf}{0.992}\PY{p}{]}\PY{p}{)}
        
        \PY{n}{pylab}\PY{o}{.}\PY{n}{scatter}\PY{p}{(}\PY{n}{X}\PY{p}{,} \PY{n}{Y}\PY{p}{)}
        \PY{n}{xdata} \PY{o}{=} \PY{n}{np}\PY{o}{.}\PY{n}{linspace}\PY{p}{(}\PY{n+nb}{min}\PY{p}{(}\PY{n}{X}\PY{p}{)}\PY{p}{,} \PY{n+nb}{max}\PY{p}{(}\PY{n}{X}\PY{p}{)}\PY{p}{,} \PY{l+m+mi}{1000}\PY{p}{)}
        
        \PY{n}{pylab}\PY{o}{.}\PY{n}{plot}\PY{p}{(}\PY{n}{xdata}\PY{p}{,} \PY{n}{func}\PY{p}{(}\PY{n}{xdata}\PY{p}{,} \PY{o}{*}\PY{n}{popt}\PY{p}{)}\PY{p}{,} \PY{l+s+s1}{\PYZsq{}}\PY{l+s+s1}{g}\PY{l+s+s1}{\PYZsq{}}\PY{p}{,} \PY{n}{label}\PY{o}{=}\PY{l+s+s1}{\PYZsq{}}\PY{l+s+s1}{fit: a=}\PY{l+s+si}{\PYZpc{}5.3f}\PY{l+s+s1}{, b=}\PY{l+s+si}{\PYZpc{}5.3f}\PY{l+s+s1}{, c=}\PY{l+s+si}{\PYZpc{}5.3f}\PY{l+s+s1}{, d=}\PY{l+s+si}{\PYZpc{}5.3f}\PY{l+s+s1}{\PYZsq{}} \PY{o}{\PYZpc{}} \PY{n+nb}{tuple}\PY{p}{(}\PY{n}{popt}\PY{p}{)}\PY{p}{)}
        \PY{n}{pylab}\PY{o}{.}\PY{n}{title}\PY{p}{(}\PY{l+s+s1}{\PYZsq{}}\PY{l+s+s1}{Planet 2 transit dept}\PY{l+s+s1}{\PYZsq{}}\PY{p}{)}
        \PY{n}{pylab}\PY{o}{.}\PY{n}{ylim}\PY{p}{(}\PY{l+m+mf}{0.992}\PY{p}{,} \PY{l+m+mf}{1.002}\PY{p}{)}
        \PY{n}{pylab}\PY{o}{.}\PY{n}{xlim}\PY{p}{(}\PY{l+m+mf}{0.585}\PY{p}{,}\PY{l+m+mf}{0.605}\PY{p}{)}
\end{Verbatim}


    Both of the curve\_fit functions produce a poor fits to the data. The
function fails for Planet 1 to reach the bottom of the transit because
of the limb darkening effect. Limb darkening occurs when a planet starts
or ends the transit. The effect is caused by only part of the planet
blocking the stars flux. The function fails to find the correct transit
depth for Planet 2 not because of the limb darkening effect, but rather
because when the data is folded, due to the time gap between observation
days the transits appears in different places. This has the effect of
shifting some of the transits out of place. It can clearly be seen that
there has been data shifted across to where the transit happens for
Planet 2 from the figure titled 'Planet 2 transit depth.' Different
function parameters were tried but positive effect could be made to
create a better fit for either of the lightcurves.

A better estimate to the transit depth can be estimated by reading off
the graph where the average depth point is for each transit. For Planet
1, the transit depth is estimated to be 0.006 +/- 0.001 and for Planet
2, 0.005 +/- 0.001. The uncertainty for these values can be read off
from the graph as half the range of reasonable values that the transit
depth could take. For Planet 1 the range is between 0.993 to 0.995 and
for Planet 2 0.994 to 0.996.

Knowing an estimate for the transit depths, the radius of the planets
can now be calculated using equation 2. Uncertainty on the radius was
calculated from the percentage uncertainty on the read-out measurements
of the transit depths for each exoplanet. The semi major axis and
densities of the planets were then derived from the calculated radii.
The mass of the exoplanets was provided by the assessor, as was the
stars mass and radius. The uncertainty for the semi major axis and
densities of the exoplanets were derived from the uncertainty in the
radii.

    \begin{Verbatim}[commandchars=\\\{\}]
{\color{incolor}In [{\color{incolor} }]:} \PY{c+c1}{\PYZsh{}Astronomical constants required for further data analysis}
        
        \PY{n}{day} \PY{o}{=} \PY{l+m+mi}{86400}             \PY{c+c1}{\PYZsh{}Number of seconds in one earth day}
        \PY{n}{G} \PY{o}{=} \PY{l+m+mf}{6.67e\PYZhy{}11}            \PY{c+c1}{\PYZsh{}The Gravitational constant}
        \PY{n}{Au} \PY{o}{=} \PY{l+m+mf}{1.5e11}             \PY{c+c1}{\PYZsh{}One astronomical unit in metres}
        
        \PY{c+c1}{\PYZsh{}Radii}
        \PY{n}{R\PYZus{}Solar} \PY{o}{=} \PY{l+m+mf}{695e6}         \PY{c+c1}{\PYZsh{}Radius of the sun in metres}
        \PY{n}{R\PYZus{}Jupiter} \PY{o}{=} \PY{l+m+mf}{699e5}       \PY{c+c1}{\PYZsh{}Radius of Jupiter in metres}
        
        \PY{c+c1}{\PYZsh{}Masses}
        \PY{n}{M\PYZus{}Solar} \PY{o}{=} \PY{l+m+mf}{1.989e30}      \PY{c+c1}{\PYZsh{}Mass of the sun in kg}
        \PY{n}{M\PYZus{}Jupiter} \PY{o}{=} \PY{l+m+mf}{1.898e27}    \PY{c+c1}{\PYZsh{}Mass of Jupiter mass in kg}
        \PY{n}{M\PYZus{}Planet\PYZus{}1} \PY{o}{=} \PY{l+m+mf}{0.137} \PY{o}{*} \PY{n}{M\PYZus{}Jupiter}
        \PY{n}{M\PYZus{}Planet\PYZus{}2} \PY{o}{=} \PY{l+m+mf}{0.227} \PY{o}{*} \PY{n}{M\PYZus{}Jupiter}
        
        \PY{c+c1}{\PYZsh{}Periods in seconds}
        
        \PY{n}{P\PYZus{}Planet\PYZus{}1} \PY{o}{=} \PY{l+m+mf}{19.237} \PY{o}{*} \PY{n}{day}
        \PY{n}{P\PYZus{}Planet\PYZus{}2} \PY{o}{=} \PY{l+m+mf}{38.98} \PY{o}{*} \PY{n}{day}
        
        \PY{c+c1}{\PYZsh{}Densities}
        \PY{n}{D\PYZus{}Jupiter} \PY{o}{=} \PY{l+m+mi}{1330}         \PY{c+c1}{\PYZsh{}Density of Jupiter, in units of kg/m\PYZca{}3}
        
        \PY{c+c1}{\PYZsh{}Variables given by assessor}
        \PY{n}{kepler\PYZus{}star\PYZus{}properties} \PY{o}{=} \PY{p}{[}\PY{l+m+mf}{1.022}\PY{o}{*}\PY{n}{M\PYZus{}Solar}\PY{p}{,} \PY{l+m+mf}{0.958}\PY{o}{*}\PY{n}{R\PYZus{}Solar}\PY{p}{]} 
        \PY{c+c1}{\PYZsh{}1st value is the mass in solar masses, 2nd is the stars radius in solar radii}
\end{Verbatim}


    \begin{Verbatim}[commandchars=\\\{\}]
{\color{incolor}In [{\color{incolor} }]:} \PY{c+c1}{\PYZsh{}Estimating the radius of each exoplanet in the Kepler star system}
        
        \PY{n}{delta\PYZus{}flux\PYZus{}planet\PYZus{}1} \PY{o}{=} \PY{l+m+mi}{1} \PY{o}{\PYZhy{}} \PY{l+m+mf}{0.994} \PY{c+c1}{\PYZsh{}read off estimate of the change in flux for planet 1}
        \PY{n}{delta\PYZus{}flux\PYZus{}planet\PYZus{}2} \PY{o}{=} \PY{l+m+mi}{1} \PY{o}{\PYZhy{}} \PY{l+m+mf}{0.995} \PY{c+c1}{\PYZsh{}read off estimate of the change in flux for planet 1}
        
        \PY{c+c1}{\PYZsh{}Planet 1}
        \PY{n}{radius\PYZus{}planet\PYZus{}1} \PY{o}{=} \PY{p}{(}\PY{n}{np}\PY{o}{.}\PY{n}{sqrt}\PY{p}{(}\PY{n}{delta\PYZus{}flux\PYZus{}planet\PYZus{}1}\PY{o}{*}\PY{p}{(}\PY{n}{kepler\PYZus{}star\PYZus{}properties}\PY{p}{[}\PY{l+m+mi}{1}\PY{p}{]}\PY{p}{)}\PY{o}{*}\PY{o}{*}\PY{l+m+mi}{2}\PY{p}{)}\PY{p}{)} \PY{o}{/} \PY{n}{R\PYZus{}Jupiter}  
        \PY{n}{radius\PYZus{}1\PYZus{}error}  \PY{o}{=} \PY{l+m+mf}{0.001}\PY{o}{/}\PY{l+m+mf}{0.006} \PY{o}{*} \PY{n}{radius\PYZus{}planet\PYZus{}1}
        \PY{n+nb}{print}\PY{p}{(}\PY{l+s+s1}{\PYZsq{}}\PY{l+s+s1}{The radius of Planet 1 is: }\PY{l+s+s1}{\PYZsq{}}\PY{p}{,} \PY{l+s+s2}{\PYZdq{}}\PY{l+s+si}{\PYZpc{}.3f}\PY{l+s+s2}{ Jupiter radii}\PY{l+s+s2}{\PYZdq{}} \PY{o}{\PYZpc{}} \PY{n}{radius\PYZus{}planet\PYZus{}1}\PY{p}{)}
        \PY{n+nb}{print}\PY{p}{(}\PY{l+s+s1}{\PYZsq{}}\PY{l+s+s1}{The uncertainty in the radius of Planet 1 is: }\PY{l+s+s1}{\PYZsq{}}\PY{p}{,} \PY{l+s+s2}{\PYZdq{}}\PY{l+s+si}{\PYZpc{}.3f}\PY{l+s+s2}{ Jupiter radii}\PY{l+s+s2}{\PYZdq{}} \PY{o}{\PYZpc{}} \PY{n}{radius\PYZus{}1\PYZus{}error}\PY{p}{)}
         
        \PY{c+c1}{\PYZsh{}Planet 2}
        \PY{n}{radius\PYZus{}planet\PYZus{}2} \PY{o}{=} \PY{p}{(}\PY{n}{np}\PY{o}{.}\PY{n}{sqrt}\PY{p}{(}\PY{n}{delta\PYZus{}flux\PYZus{}planet\PYZus{}2}\PY{o}{*}\PY{p}{(}\PY{n}{kepler\PYZus{}star\PYZus{}properties}\PY{p}{[}\PY{l+m+mi}{1}\PY{p}{]}\PY{p}{)}\PY{o}{*}\PY{o}{*}\PY{l+m+mi}{2}\PY{p}{)}\PY{p}{)} \PY{o}{/} \PY{n}{R\PYZus{}Jupiter}  
        \PY{n}{radius\PYZus{}2\PYZus{}error}  \PY{o}{=} \PY{l+m+mf}{0.001}\PY{o}{/}\PY{l+m+mf}{0.005} \PY{o}{*} \PY{n}{radius\PYZus{}planet\PYZus{}2}
        \PY{n+nb}{print}\PY{p}{(}\PY{l+s+s1}{\PYZsq{}}\PY{l+s+s1}{The radius of Planet 2 is: }\PY{l+s+s1}{\PYZsq{}}\PY{p}{,} \PY{l+s+s2}{\PYZdq{}}\PY{l+s+si}{\PYZpc{}.3f}\PY{l+s+s2}{ Jupiter radii}\PY{l+s+s2}{\PYZdq{}} \PY{o}{\PYZpc{}} \PY{n}{radius\PYZus{}planet\PYZus{}2}\PY{p}{)}
        \PY{n+nb}{print}\PY{p}{(}\PY{l+s+s1}{\PYZsq{}}\PY{l+s+s1}{The uncertainty in the radius of Planet 2 is: }\PY{l+s+s1}{\PYZsq{}}\PY{p}{,} \PY{l+s+s2}{\PYZdq{}}\PY{l+s+si}{\PYZpc{}.3f}\PY{l+s+s2}{ Jupiter radii}\PY{l+s+s2}{\PYZdq{}} \PY{o}{\PYZpc{}} \PY{n}{radius\PYZus{}2\PYZus{}error}\PY{p}{)}
\end{Verbatim}


    \begin{Verbatim}[commandchars=\\\{\}]
{\color{incolor}In [{\color{incolor} }]:} \PY{c+c1}{\PYZsh{}Estimating the semi\PYZhy{}major axis for both planets using equation (3)}
        \PY{c+c1}{\PYZsh{}Planet 1}
        \PY{n}{semi\PYZus{}mjr\PYZus{}p1} \PY{o}{=}\PY{p}{(}\PY{p}{(}\PY{p}{(}\PY{p}{(}\PY{n}{P\PYZus{}Planet\PYZus{}1}\PY{o}{*}\PY{o}{*}\PY{l+m+mi}{2}\PY{p}{)}\PY{o}{*}\PY{n}{G}\PY{o}{*}\PY{p}{(}\PY{n}{kepler\PYZus{}star\PYZus{}properties}\PY{p}{[}\PY{l+m+mi}{0}\PY{p}{]} \PY{o}{+} \PY{n}{M\PYZus{}Planet\PYZus{}1}\PY{p}{)}\PY{p}{)}\PY{o}{/}\PY{p}{(}\PY{l+m+mi}{4}\PY{o}{*}\PY{p}{(}\PY{n}{np}\PY{o}{.}\PY{n}{pi}\PY{o}{*}\PY{o}{*}\PY{l+m+mi}{2}\PY{p}{)}\PY{p}{)}\PY{p}{)}\PY{o}{*}\PY{o}{*}\PY{p}{(}\PY{l+m+mi}{1}\PY{o}{/}\PY{l+m+mi}{3}\PY{p}{)}\PY{p}{)} \PY{o}{/}\PY{n}{Au} 
        \PY{n}{semi\PYZus{}mjr\PYZus{}error1} \PY{o}{=} \PY{l+m+mi}{2}\PY{o}{*}\PY{p}{(}\PY{l+m+mf}{0.001}\PY{o}{/}\PY{l+m+mf}{0.006}\PY{p}{)} \PY{o}{*} \PY{n}{semi\PYZus{}mjr\PYZus{}p1}
        \PY{n+nb}{print} \PY{p}{(}\PY{l+s+s1}{\PYZsq{}}\PY{l+s+s1}{The semi\PYZhy{}major axis of Planet 1, in Astronomical Units, is: }\PY{l+s+s1}{\PYZsq{}}\PY{p}{,} \PY{l+s+s2}{\PYZdq{}}\PY{l+s+si}{\PYZpc{}.3f}\PY{l+s+s2}{ AU}\PY{l+s+s2}{\PYZdq{}} \PY{o}{\PYZpc{}} \PY{n}{semi\PYZus{}mjr\PYZus{}p1}\PY{p}{)}
        \PY{n+nb}{print}\PY{p}{(}\PY{l+s+s1}{\PYZsq{}}\PY{l+s+s1}{The uncertainty in the semi major axis of Planet 1 is: }\PY{l+s+s1}{\PYZsq{}}\PY{p}{,} \PY{l+s+s2}{\PYZdq{}}\PY{l+s+si}{\PYZpc{}.3f}\PY{l+s+s2}{ Jupiter radii}\PY{l+s+s2}{\PYZdq{}} \PY{o}{\PYZpc{}} \PY{n}{semi\PYZus{}mjr\PYZus{}error1}\PY{p}{)}
        
        \PY{c+c1}{\PYZsh{}Planet 2}
        \PY{n}{semi\PYZus{}mjr\PYZus{}p2} \PY{o}{=} \PY{p}{(}\PY{p}{(}\PY{p}{(}\PY{p}{(}\PY{n}{P\PYZus{}Planet\PYZus{}2}\PY{o}{*}\PY{o}{*}\PY{l+m+mi}{2}\PY{p}{)}\PY{o}{*}\PY{n}{G}\PY{o}{*}\PY{p}{(}\PY{n}{kepler\PYZus{}star\PYZus{}properties}\PY{p}{[}\PY{l+m+mi}{0}\PY{p}{]} \PY{o}{+} \PY{n}{M\PYZus{}Planet\PYZus{}2}\PY{p}{)}\PY{p}{)}\PY{o}{/}\PY{p}{(}\PY{l+m+mi}{4}\PY{o}{*}\PY{p}{(}\PY{n}{np}\PY{o}{.}\PY{n}{pi}\PY{o}{*}\PY{o}{*}\PY{l+m+mi}{2}\PY{p}{)}\PY{p}{)}\PY{p}{)}\PY{o}{*}\PY{o}{*}\PY{p}{(}\PY{l+m+mi}{1}\PY{o}{/}\PY{l+m+mi}{3}\PY{p}{)}\PY{p}{)}\PY{o}{/}\PY{n}{Au} 
        \PY{n}{semi\PYZus{}mjr\PYZus{}error2} \PY{o}{=} \PY{l+m+mi}{2}\PY{o}{*}\PY{p}{(}\PY{l+m+mf}{0.001}\PY{o}{/}\PY{l+m+mf}{0.005}\PY{p}{)} \PY{o}{*} \PY{n}{semi\PYZus{}mjr\PYZus{}p2}
        \PY{n+nb}{print} \PY{p}{(}\PY{l+s+s1}{\PYZsq{}}\PY{l+s+s1}{The semi\PYZhy{}major axis of Planet 2, in Astronomical Units, is: }\PY{l+s+s1}{\PYZsq{}}\PY{p}{,} \PY{l+s+s2}{\PYZdq{}}\PY{l+s+si}{\PYZpc{}.3f}\PY{l+s+s2}{ AU}\PY{l+s+s2}{\PYZdq{}} \PY{o}{\PYZpc{}} \PY{n}{semi\PYZus{}mjr\PYZus{}p2}\PY{p}{)}
        \PY{n+nb}{print}\PY{p}{(}\PY{l+s+s1}{\PYZsq{}}\PY{l+s+s1}{The uncertainty in the semi major axis of Planet 2 is: }\PY{l+s+s1}{\PYZsq{}}\PY{p}{,} \PY{l+s+s2}{\PYZdq{}}\PY{l+s+si}{\PYZpc{}.3f}\PY{l+s+s2}{ Jupiter radii}\PY{l+s+s2}{\PYZdq{}} \PY{o}{\PYZpc{}} \PY{n}{semi\PYZus{}mjr\PYZus{}error2}\PY{p}{)}
\end{Verbatim}


    \begin{Verbatim}[commandchars=\\\{\}]
{\color{incolor}In [{\color{incolor} }]:} \PY{c+c1}{\PYZsh{}Calculating the planets densities}
        \PY{c+c1}{\PYZsh{}Density  = planet mass / 4/3*pi*planet radius\PYZca{}3}
        \PY{c+c1}{\PYZsh{}Planet 1}
        \PY{n}{Density\PYZus{}Planet\PYZus{}1} \PY{o}{=} \PY{n}{M\PYZus{}Planet\PYZus{}1} \PY{o}{/} \PY{p}{(}\PY{p}{(}\PY{l+m+mi}{4}\PY{o}{/}\PY{l+m+mi}{3}\PY{p}{)}\PY{o}{*}\PY{n}{np}\PY{o}{.}\PY{n}{pi}\PY{o}{*}\PY{p}{(}\PY{n}{radius\PYZus{}planet\PYZus{}1}\PY{o}{*}\PY{n}{R\PYZus{}Jupiter}\PY{p}{)}\PY{o}{*}\PY{o}{*}\PY{l+m+mi}{3}\PY{p}{)}
        \PY{n}{Density\PYZus{}Planet\PYZus{}1J} \PY{o}{=} \PY{n}{Density\PYZus{}Planet\PYZus{}1} \PY{o}{/} \PY{n}{D\PYZus{}Jupiter}
        \PY{n}{Density\PYZus{}error\PYZus{}1} \PY{o}{=} \PY{l+m+mi}{3}\PY{o}{*}\PY{p}{(}\PY{l+m+mf}{0.001}\PY{o}{/}\PY{l+m+mf}{0.006}\PY{p}{)} \PY{o}{*} \PY{n}{Density\PYZus{}Planet\PYZus{}1}
        \PY{n}{Density\PYZus{}error\PYZus{}1J} \PY{o}{=} \PY{l+m+mi}{3}\PY{o}{*}\PY{p}{(}\PY{l+m+mf}{0.001}\PY{o}{/}\PY{l+m+mf}{0.006}\PY{p}{)} \PY{o}{*} \PY{n}{Density\PYZus{}Planet\PYZus{}1J}
        \PY{n+nb}{print}\PY{p}{(}\PY{l+s+s1}{\PYZsq{}}\PY{l+s+s1}{The density of Planet 1 is: }\PY{l+s+s1}{\PYZsq{}}\PY{p}{,} \PY{l+s+s2}{\PYZdq{}}\PY{l+s+si}{\PYZpc{}.3d}\PY{l+s+s2}{ kg/m\PYZca{}3}\PY{l+s+s2}{\PYZdq{}} \PY{o}{\PYZpc{}} \PY{n}{Density\PYZus{}Planet\PYZus{}1}\PY{p}{,} \PY{l+s+s1}{\PYZsq{}}\PY{l+s+s1}{ OR }\PY{l+s+s1}{\PYZsq{}}\PY{p}{,} \PY{l+s+s2}{\PYZdq{}}\PY{l+s+si}{\PYZpc{}.3f}\PY{l+s+s2}{ Jupiter densities}\PY{l+s+s2}{\PYZdq{}} \PY{o}{\PYZpc{}} \PY{n}{Density\PYZus{}Planet\PYZus{}1J}\PY{p}{)}
        \PY{n+nb}{print}\PY{p}{(}\PY{l+s+s1}{\PYZsq{}}\PY{l+s+s1}{The density uncertainty for Planet 1 is: }\PY{l+s+s1}{\PYZsq{}}\PY{p}{,} \PY{l+s+s2}{\PYZdq{}}\PY{l+s+si}{\PYZpc{}.3d}\PY{l+s+s2}{ kg/m\PYZca{}3}\PY{l+s+s2}{\PYZdq{}} \PY{o}{\PYZpc{}} \PY{n}{Density\PYZus{}error\PYZus{}1}\PY{p}{,} \PY{l+s+s1}{\PYZsq{}}\PY{l+s+s1}{ OR }\PY{l+s+s1}{\PYZsq{}}\PY{p}{,} \PY{l+s+s2}{\PYZdq{}}\PY{l+s+si}{\PYZpc{}.3f}\PY{l+s+s2}{ Jupiter densities}\PY{l+s+s2}{\PYZdq{}} \PY{o}{\PYZpc{}} \PY{n}{Density\PYZus{}error\PYZus{}1J}\PY{p}{)}
        
        \PY{c+c1}{\PYZsh{}Planet 2}
        \PY{n}{Density\PYZus{}Planet\PYZus{}2} \PY{o}{=} \PY{n}{M\PYZus{}Planet\PYZus{}2} \PY{o}{/} \PY{p}{(}\PY{p}{(}\PY{l+m+mi}{4}\PY{o}{/}\PY{l+m+mi}{3}\PY{p}{)}\PY{o}{*}\PY{n}{np}\PY{o}{.}\PY{n}{pi}\PY{o}{*}\PY{p}{(}\PY{n}{radius\PYZus{}planet\PYZus{}2}\PY{o}{*}\PY{n}{R\PYZus{}Jupiter}\PY{p}{)}\PY{o}{*}\PY{o}{*}\PY{l+m+mi}{3}\PY{p}{)}
        \PY{n}{Density\PYZus{}Planet\PYZus{}2J} \PY{o}{=} \PY{n}{Density\PYZus{}Planet\PYZus{}2} \PY{o}{/} \PY{n}{D\PYZus{}Jupiter}
        \PY{n}{Density\PYZus{}error\PYZus{}2} \PY{o}{=} \PY{l+m+mi}{3}\PY{o}{*}\PY{p}{(}\PY{l+m+mf}{0.001}\PY{o}{/}\PY{l+m+mf}{0.005}\PY{p}{)} \PY{o}{*} \PY{n}{Density\PYZus{}Planet\PYZus{}2}
        \PY{n}{Density\PYZus{}error\PYZus{}2J} \PY{o}{=} \PY{l+m+mi}{3}\PY{o}{*}\PY{p}{(}\PY{l+m+mf}{0.001}\PY{o}{/}\PY{l+m+mf}{0.005}\PY{p}{)} \PY{o}{*} \PY{n}{Density\PYZus{}Planet\PYZus{}2J}
        \PY{n+nb}{print}\PY{p}{(}\PY{l+s+s1}{\PYZsq{}}\PY{l+s+s1}{The density of Planet 2 is: }\PY{l+s+s1}{\PYZsq{}}\PY{p}{,} \PY{l+s+s2}{\PYZdq{}}\PY{l+s+si}{\PYZpc{}.3d}\PY{l+s+s2}{ kg/m\PYZca{}3}\PY{l+s+s2}{\PYZdq{}} \PY{o}{\PYZpc{}} \PY{n}{Density\PYZus{}Planet\PYZus{}2}\PY{p}{,} \PY{l+s+s1}{\PYZsq{}}\PY{l+s+s1}{ OR }\PY{l+s+s1}{\PYZsq{}}\PY{p}{,} \PY{l+s+s2}{\PYZdq{}}\PY{l+s+si}{\PYZpc{}.3f}\PY{l+s+s2}{ Jupiter densities}\PY{l+s+s2}{\PYZdq{}} \PY{o}{\PYZpc{}} \PY{n}{Density\PYZus{}Planet\PYZus{}2J}\PY{p}{)}
        \PY{n+nb}{print}\PY{p}{(}\PY{l+s+s1}{\PYZsq{}}\PY{l+s+s1}{The density uncertainty for Planet 2 is: }\PY{l+s+s1}{\PYZsq{}}\PY{p}{,} \PY{l+s+s2}{\PYZdq{}}\PY{l+s+si}{\PYZpc{}.3d}\PY{l+s+s2}{ kg/m\PYZca{}3}\PY{l+s+s2}{\PYZdq{}} \PY{o}{\PYZpc{}} \PY{n}{Density\PYZus{}error\PYZus{}2}\PY{p}{,} \PY{l+s+s1}{\PYZsq{}}\PY{l+s+s1}{ OR }\PY{l+s+s1}{\PYZsq{}}\PY{p}{,} \PY{l+s+s2}{\PYZdq{}}\PY{l+s+si}{\PYZpc{}.3f}\PY{l+s+s2}{ Jupiter densities}\PY{l+s+s2}{\PYZdq{}} \PY{o}{\PYZpc{}} \PY{n}{Density\PYZus{}error\PYZus{}2J}\PY{p}{)}
\end{Verbatim}


    \begin{Verbatim}[commandchars=\\\{\}]
{\color{incolor}In [{\color{incolor} }]:} \PY{k}{def} \PY{n+nf}{conflevels}\PY{p}{(}\PY{n}{x}\PY{p}{,}\PY{n}{y}\PY{p}{,}\PY{n}{nbins}\PY{o}{=}\PY{l+m+mi}{1}\PY{p}{,}\PY{n}{confints}\PY{o}{=}\PY{p}{[}\PY{l+m+mf}{0.99}\PY{p}{,}\PY{l+m+mf}{0.95}\PY{p}{,}\PY{l+m+mf}{0.68}\PY{p}{]}\PY{p}{)}\PY{p}{:}
            \PY{c+c1}{\PYZsh{} Make a 2d normed histogram}
            \PY{k}{if} \PY{p}{(}\PY{n}{nbins}\PY{o}{\PYZgt{}}\PY{l+m+mi}{1}\PY{p}{)}\PY{p}{:}
                \PY{n}{H}\PY{p}{,}\PY{n}{xedges}\PY{p}{,}\PY{n}{yedges}\PY{o}{=}\PY{n}{np}\PY{o}{.}\PY{n}{histogram2d}\PY{p}{(}\PY{n}{x}\PY{p}{,}\PY{n}{y}\PY{p}{,}\PY{n}{bins}\PY{o}{=}\PY{n}{nbins}\PY{p}{,}\PY{n}{normed}\PY{o}{=}\PY{k+kc}{True}\PY{p}{)}
            \PY{k}{else}\PY{p}{:}
                \PY{n}{H}\PY{p}{,}\PY{n}{xedges}\PY{p}{,}\PY{n}{yedges}\PY{o}{=}\PY{n}{np}\PY{o}{.}\PY{n}{histogram2d}\PY{p}{(}\PY{n}{x}\PY{p}{,}\PY{n}{y}\PY{p}{,}\PY{n}{normed}\PY{o}{=}\PY{k+kc}{True}\PY{p}{)}
                
            \PY{n}{norm}\PY{o}{=}\PY{n}{H}\PY{o}{.}\PY{n}{sum}\PY{p}{(}\PY{p}{)}              \PY{c+c1}{\PYZsh{} Find the norm of the sum}
           
            \PY{n}{contour1}\PY{o}{=}\PY{l+m+mf}{0.99}             \PY{c+c1}{\PYZsh{} Set contour levels}
            \PY{n}{contour2}\PY{o}{=}\PY{l+m+mf}{0.95}
            \PY{n}{contour3}\PY{o}{=}\PY{l+m+mf}{0.68}
        
            \PY{n}{target1} \PY{o}{=} \PY{n}{norm}\PY{o}{*}\PY{n}{contour1}   \PY{c+c1}{\PYZsh{} Set target levels as percentage of norm}
            \PY{n}{target2} \PY{o}{=} \PY{n}{norm}\PY{o}{*}\PY{n}{contour2}
            \PY{n}{target3} \PY{o}{=} \PY{n}{norm}\PY{o}{*}\PY{n}{contour3}
        
            \PY{c+c1}{\PYZsh{} Take histogram bin membership as proportional to likelihood \PYZhy{} this is true when data comes from a Markovian process}
            \PY{k}{def} \PY{n+nf}{objective}\PY{p}{(}\PY{n}{limit}\PY{p}{,} \PY{n}{target}\PY{p}{)}\PY{p}{:}
                \PY{n}{w} \PY{o}{=} \PY{n}{np}\PY{o}{.}\PY{n}{where}\PY{p}{(}\PY{n}{H}\PY{o}{\PYZgt{}}\PY{n}{limit}\PY{p}{)}
                \PY{n}{count} \PY{o}{=} \PY{n}{H}\PY{p}{[}\PY{n}{w}\PY{p}{]}
                
                
                \PY{k}{return} \PY{n}{count}\PY{o}{.}\PY{n}{sum}\PY{p}{(}\PY{p}{)} \PY{o}{\PYZhy{}} \PY{n}{target}
        
            \PY{c+c1}{\PYZsh{} Find levels by summing histogram to objective}
            \PY{n}{level1}\PY{o}{=} \PY{n}{optimize}\PY{o}{.}\PY{n}{bisect}\PY{p}{(}\PY{n}{objective}\PY{p}{,} \PY{n}{H}\PY{o}{.}\PY{n}{min}\PY{p}{(}\PY{p}{)}\PY{p}{,} \PY{n}{H}\PY{o}{.}\PY{n}{max}\PY{p}{(}\PY{p}{)}\PY{p}{,} \PY{n}{args}\PY{o}{=}\PY{p}{(}\PY{n}{target1}\PY{p}{,}\PY{p}{)}\PY{p}{)}
            \PY{n}{level2}\PY{o}{=} \PY{n}{optimize}\PY{o}{.}\PY{n}{bisect}\PY{p}{(}\PY{n}{objective}\PY{p}{,} \PY{n}{H}\PY{o}{.}\PY{n}{min}\PY{p}{(}\PY{p}{)}\PY{p}{,} \PY{n}{H}\PY{o}{.}\PY{n}{max}\PY{p}{(}\PY{p}{)}\PY{p}{,} \PY{n}{args}\PY{o}{=}\PY{p}{(}\PY{n}{target2}\PY{p}{,}\PY{p}{)}\PY{p}{)}
            \PY{n}{level3}\PY{o}{=} \PY{n}{optimize}\PY{o}{.}\PY{n}{bisect}\PY{p}{(}\PY{n}{objective}\PY{p}{,} \PY{n}{H}\PY{o}{.}\PY{n}{min}\PY{p}{(}\PY{p}{)}\PY{p}{,} \PY{n}{H}\PY{o}{.}\PY{n}{max}\PY{p}{(}\PY{p}{)}\PY{p}{,} \PY{n}{args}\PY{o}{=}\PY{p}{(}\PY{n}{target3}\PY{p}{,}\PY{p}{)}\PY{p}{)}
        
            \PY{c+c1}{\PYZsh{} For nice contour shading with seaborn, define top level}
            \PY{n}{level4}\PY{o}{=}\PY{n}{H}\PY{o}{.}\PY{n}{max}\PY{p}{(}\PY{p}{)}
            \PY{n}{levels}\PY{o}{=}\PY{p}{[}\PY{n}{level1}\PY{p}{,}\PY{n}{level2}\PY{p}{,}\PY{n}{level3}\PY{p}{]}
        
            \PY{k}{return} \PY{n}{levels}
\end{Verbatim}


    \begin{Verbatim}[commandchars=\\\{\}]
{\color{incolor}In [{\color{incolor} }]:} \PY{n}{masses} \PY{o}{=} \PY{n}{np}\PY{o}{.}\PY{n}{array}\PY{p}{(}\PY{p}{[}\PY{l+m+mf}{0.137}\PY{p}{,} \PY{l+m+mf}{0.227}\PY{p}{]}\PY{p}{)}   \PY{c+c1}{\PYZsh{}Array containing Juptier masses of each planet}
        \PY{n}{periods} \PY{o}{=} \PY{n}{np}\PY{o}{.}\PY{n}{array}\PY{p}{(}\PY{p}{[}\PY{l+m+mf}{19.237}\PY{p}{,} \PY{l+m+mf}{38.98}\PY{p}{]}\PY{p}{)} \PY{c+c1}{\PYZsh{}Array containing period in days of each planet}
        
        \PY{c+c1}{\PYZsh{}Table is taken from the NASA exoplanet archive ICE Plotting tool of confirmed planets}
        \PY{c+c1}{\PYZsh{}https://exoplanetarchive.ipac.caltech.edu/cgi\PYZhy{}bin/IcePlotter/nph\PYZhy{}icePlotInit?mode=demo\PYZam{}set=confirmed}
        \PY{c+c1}{\PYZsh{}accessed 19/12/2019}
        
        \PY{n}{table1} \PY{o}{=} \PY{n}{np}\PY{o}{.}\PY{n}{loadtxt}\PY{p}{(}\PY{l+s+s1}{\PYZsq{}}\PY{l+s+s1}{Data/plot.tbl.txt}\PY{l+s+s1}{\PYZsq{}}\PY{p}{,} \PY{n}{skiprows}\PY{o}{=}\PY{l+m+mi}{3}\PY{p}{)}
        
        \PY{n}{levels} \PY{o}{=} \PY{n}{conflevels}\PY{p}{(}\PY{n}{np}\PY{o}{.}\PY{n}{log10}\PY{p}{(}\PY{n}{table1}\PY{p}{[}\PY{p}{:}\PY{p}{,}\PY{l+m+mi}{1}\PY{p}{]}\PY{p}{)}\PY{p}{,}\PY{n}{np}\PY{o}{.}\PY{n}{log10}\PY{p}{(}\PY{n}{table1}\PY{p}{[}\PY{p}{:}\PY{p}{,}\PY{l+m+mi}{2}\PY{p}{]}\PY{p}{)}\PY{p}{)}
        
        \PY{n}{pylab}\PY{o}{.}\PY{n}{xlabel}\PY{p}{(}\PY{l+s+sa}{r}\PY{l+s+s1}{\PYZsq{}}\PY{l+s+s1}{Orbital period (days)}\PY{l+s+s1}{\PYZsq{}}\PY{p}{,}\PY{n}{fontsize}\PY{o}{=}\PY{l+m+mi}{14}\PY{p}{)}
        \PY{n}{pylab}\PY{o}{.}\PY{n}{ylabel}\PY{p}{(}\PY{l+s+sa}{r}\PY{l+s+s1}{\PYZsq{}}\PY{l+s+s1}{Mass (Jupiter mass)}\PY{l+s+s1}{\PYZsq{}}\PY{p}{,}\PY{n}{fontsize}\PY{o}{=}\PY{l+m+mi}{14}\PY{p}{)}
        
        \PY{n}{pylab}\PY{o}{.}\PY{n}{scatter}\PY{p}{(}\PY{n}{np}\PY{o}{.}\PY{n}{log10}\PY{p}{(}\PY{n}{table1}\PY{p}{[}\PY{p}{:}\PY{p}{,}\PY{l+m+mi}{1}\PY{p}{]}\PY{p}{)}\PY{p}{,} \PY{n}{np}\PY{o}{.}\PY{n}{log10}\PY{p}{(}\PY{n}{table1}\PY{p}{[}\PY{p}{:}\PY{p}{,}\PY{l+m+mi}{2}\PY{p}{]}\PY{p}{)}\PY{p}{,} \PY{n}{alpha} \PY{o}{=} \PY{l+m+mf}{0.4}\PY{p}{,} \PY{n}{color}\PY{o}{=}\PY{l+s+s1}{\PYZsq{}}\PY{l+s+s1}{m}\PY{l+s+s1}{\PYZsq{}}\PY{p}{)} \PY{c+c1}{\PYZsh{} alpha determines transparency}
        \PY{n}{sns}\PY{o}{.}\PY{n}{kdeplot}\PY{p}{(}\PY{n}{np}\PY{o}{.}\PY{n}{log10}\PY{p}{(}\PY{n}{table1}\PY{p}{[}\PY{p}{:}\PY{p}{,}\PY{l+m+mi}{1}\PY{p}{]}\PY{p}{)}\PY{p}{,}\PY{n}{np}\PY{o}{.}\PY{n}{log10}\PY{p}{(}\PY{n}{table1}\PY{p}{[}\PY{p}{:}\PY{p}{,}\PY{l+m+mi}{2}\PY{p}{]}\PY{p}{)}\PY{p}{,}\PY{n}{n\PYZus{}levels}\PY{o}{=}\PY{n}{levels}\PY{p}{)}
        
        \PY{n}{pylab}\PY{o}{.}\PY{n}{scatter}\PY{p}{(}\PY{n}{np}\PY{o}{.}\PY{n}{log10}\PY{p}{(}\PY{n}{periods}\PY{p}{)}\PY{p}{,} \PY{n}{np}\PY{o}{.}\PY{n}{log10}\PY{p}{(}\PY{n}{masses}\PY{p}{)}\PY{p}{,} \PY{n}{color} \PY{o}{=} \PY{l+s+s1}{\PYZsq{}}\PY{l+s+s1}{black}\PY{l+s+s1}{\PYZsq{}}\PY{p}{,} \PY{n}{alpha} \PY{o}{=} \PY{l+m+mf}{0.7}\PY{p}{)}
\end{Verbatim}


    \subsubsection{4. Discussion}\label{discussion}

    The paper had the goal of finding planets in an extrasolar system and to
determine whether the planets are habitable. Two planets were found in
the Kepler star system of periods 19.237 (Planet 1) and 38.98 (Planet
2). The radii, semi-major axes and densities were also found for both
planets. For Planet 1 they were; 0.738 +/-0.123 Jupiter radii, 0.141 +/-
0.047 AU and 0.340 +/-0.170 Jupiter's density. For Planet 2 they were;
0.674 +/- 0.135 Jupiter radii, 0.226 +/- 0.226 Au and 0.741 +/- 0.371
Jupiter's density. Planet 2 is a smaller and denser than planet one and
orbits the host star further out than Planet 1.

Both the discovered planets have significantly large fractions of
Jupiter's radius. The largest known rocky planets have radii of
approximately two earth radii (0.2 in Jupiter radii). Both exoplanets
had radii over 3 times this value. Therefore, we can rule out with
certainty that the planets can't be rocky bodies. The masses and
densities of the exoplanets are too low to suggest them being a
migrating Hot Jupiter type of planet. Both planets have densities
indicative of a low-density gas giants, such a Saturn. Knowing this and
looking at the contour plot of the mass against the orbital period. We
can see that both our planets fall just outside the accretion planet
formation regime (the two points highlighted in black). We can therefore
conclude that both Planet 1 and Planet 2 are low-density gas giant
planets formed by accretion. Due to the being made up of primarily gas,
both exoplanets would not be habitable. Furthermore, the exoplanet's
orbits are very close to the host star. The host star is a G-type main
sequence star. Therefore, one would expect that any planet that is not
in an orbit of approximately 1 Au to be uninhabitable. By this same
reasoning we can also rule that the planets will not be ice giants as
they are too close to the host star for ice to remain solid.

IIt is clear to see that this extrasolar system is not akin to our own.
It is dominated by gaseous planets, in low orbits travelling around the
host star at high velocities. Our own solar system does not have any
gaseous planets within 5 Au of the sun whereas in this system both
planets orbit within 0.25 Au. We can also conclude that in the early
stages of the extrasolar systems life, there must have been of gas and
dust orbiting near to the star because lots of gas and dust is required
for gas giants to form by accretion.

Further measurements of spectroscopy could be carried out on these two
exoplanets to determine their atmospheric composition. This would allow
for a greater understanding of composition of the planets mass. This
could give more insight into how the exoplanets formed.

    \subsubsection{5. Summary}\label{summary}

\(\bullet\) Data reduction techniques were applied to the GROND dataset.

\(\bullet\) The techniques demonstrated were bias-frame subtractions and
flat fielding.

\(\bullet\) An aperture photometry function was written to determine the
magnitude of a star in the image.

\(\bullet\) The function accounted for background noise in the image and
found the uncertainties.

\(\bullet\) The function calculated the magnitude of a star from the
zero point in the image and a standard star of magnitude +19.5.

\(\bullet\) The absolute magnitude was calculated to be +17.33 +/- 0.62.

\(\bullet\) Data reduction techniques were applied to lightcurves from
the Kepler Space telescope as part of an investigation into finding
exoplanets around a host star.

\(\bullet\) The data reduction techniques implemented were an inf/nan
filter and a median filter to noramlise the lightcurve data.

\(\bullet\) A function was applied to the raw data to give the
periodicities of the exoplanets. A periodogram was plotted from this
function. From the periodogram two exoplanets were found.

\(\bullet\) Multiple transit lightcurves were folded to give a stronger
data trend. A box function was applied to the folded lightcurve to
calculate the transit depths.

\(\bullet\) From the transit depths the radius of the planets was
determined, which allowed for the calculation of each exoplanets semi
major axis and density.

\(\bullet\) Exoplanet 1's properties were; period 19.237 days, radius
0.738 +/-0.123 Jupiter radii, semi major axis 0.141 AU and density 0.340
+/-0.170 Jupiter densities.

\(\bullet\) Exoplanet 2's properties were; period 38.98 days, radius
0.674 +/-0.135 Jupiter radii, semi major axis 0.226 Au and density 0.741
+/- 0.371 Jupiter densities.

\(\bullet\) It was concluded that the planets are low density gas giants
that formed close to the host star

\(\bullet\) Data from the NASA exoplanet archive was used to plot a
graph of planet mass against period. The two discovered exoplanets were
highlights

\(\bullet\) The planets were found to lie just outside the accretion
regime of planet formation. It was concluded that the early extrasolar
system had lots of dust close to the star from which the gas giants
formed.

\(\bullet\) It was also concluded that the extrasolar system is not like
our own. It is dominated by gaseous planets, in low orbits travelling
around the host star at high velocities. Whereas in our solar system
there are only rocky bodies below 5 Au.

    \subsubsection{References:}\label{references}

{[}1{]} Ridpath, 2018, Oxford University Press, A Dictionary of
Astronmoy (3 ed.), Definition of an Exoplanet

{[}2{]} Perryman, 2011, Cambridge University Press, The Exoplanet
Handbook ,pg 104-105

{[}3{]} Asher-Johnson, 2015, Princeton University Press,How Do You Find
an Exoplanet?,pg 67-69

{[}4{]} Warner, 2016, Springer Publishing, A Practical Guide to
Lightcurve Photometry and Analysis, pg 58

This research has made use of the NASA Exoplanet Archive, which is
operated by the California Institute of Technology, under contract with
the National Aeronautics and Space Administration under the Exoplanet
Exploration Program.


    % Add a bibliography block to the postdoc
    
    
    
    \end{document}
